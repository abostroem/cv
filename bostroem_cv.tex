\documentclass[10pt]{cv}
\font\cap=cmcsc10
\usepackage{amsmath}

\topmargin 0pt
\headheight 0pt
\headsep 0pt
\textheight 674pt
\pagestyle{empty}
\parindent 0.08in
\parskip \baselineskip
\topmargin 0in
\footskip 1in
%\oddsidemargin 0in
\oddsidemargin -0.25in
%\evensidemargin 0in
\evensidemargin -0.25in

\textwidth 6.9in

\setlength{\topmargin}{-0.25in}
\setlength{\headheight}{0in}
\setlength{\headsep}{0in}
\setlength{\topskip}{0.75in}
\setlength{\textheight}{9in}
\pagestyle{myheadings}
\setlength{\parskip}{0in}

%set the number in the second set of curly braces to the number you want displayed on
%the first page
%setcounter{page}{2}
  
\usepackage{hyperref}
\usepackage{bibentry}
\usepackage{url}
\usepackage{color}
\usepackage{aas_macros}
\usepackage{hanging}
\usepackage{natbib}
\bibliographystyle{apj}
\newcommand\hangbibentry[1]{%
    \smallskip\par\hangpara{1em}{1}\bibentry{#1}\smallskip\par %{indent}{afterline}
}

%\usepackage{helvetica} % uses helvetica postscript font (download helvetica.sty)
%\usepackage{newcent}   % uses new century schoolbook postscript font  
\setlength{\topmargin}{-0.6in}  % Start text higher on the page 
\setlength{\textheight}{9.5in}  % increase textheight to fit more on a page
\setlength{\headsep}{0.2in}     % space between header and text
\setlength{\headheight}{12pt}   % make room for header
\usepackage{fancyhdr}  % use fancyhdr package to get 2-line header
\renewcommand{\headrulewidth}{0pt} % suppress line drawn by default by fancyhdr
%\lhead{\hspace*{-\sectionwidth}Justin Ely} % force lhead all the way left
%\rhead{Page \thepage}  % put page number at right
\cfoot{}  % the footer is empty
%\pagestyle{fancy} % set pagestyle for the document

\begin{document} 
\section{{\LARGE \bf{K. Azalee Bostroem}}}
{\rule{\linewidth}{0.5mm}} \\
\begin{minipage}{0.55\textwidth}
Department of Physics \\
University of California, Davis \\
Davis, CA
\end{minipage}
\begin{minipage}{0.45\textwidth}
\begin{flushright}
\vspace{0.25cm}
\color{blue}\url{kabostroem@ucdavis.edu}\\
\color{blue}\url{https://github.com/abostroem} \\
\color{blue}\url{https://abostroem.wixsite.com/home}
\end{flushright}
\end{minipage} \\
{\rule{\linewidth}{0.5mm}} 
\vspace{-.5em}
%\begin{resume}
\begin{llist}
%-------------------------------------------------------------
%-----------------EMPLOYMENT-----------------------
%-------------------------------------------------------------
\vspace{-0.1in}   
\sectiontitle{Employment}
    \textsc{Graduate Student Researcher}\hfill2015-Present \\
    University of California, Davis, CA\\
\\
\textsc{Supernova Spectroscopy Analyst}\hfill Fall 2015\\ 
University of Pittsburgh, Pittsburgh, PA\\
\\
\textsc{Research and Instrument Analyst}\hfill 2009 - 2014 \\
Space Telescope Science Institute, Baltimore, MD\\
%    \textsc{Graduate Student Researcher}\hfill2016-present \\
%    University of California, Davis, CA\\
%    \\

%-------------------------------------------------------------
%--------------------EDUCATION------------------------
%-------------------------------------------------------------
\vspace{-0.1in}   
\sectiontitle{Education}
\textsc{University of California, Davis}\hfill 2014 - Present\\
Davis, CA, USA \\
Ph.D. in Physics \\
Thesis Advisor: Stefano Valenti \\
\\
\textsc{San Diego State University}\hfill 2006 - 2009\\
San Diego, CA, USA \\
M.S. in Astronomy \\
Thesis Advisor: Douglas Leonard \\
\\
\textsc{Vassar College} \hfill 2002 - 2006 \\
Poughkeepsie, NY, USA \\
B.A. in Mathematics \\
California State and New York State Secondary Teaching Certifications in Mathematics \\
%-------------------------------------------------------------
%--------------------HONORS----------------------------
%-------------------------------------------------------------
\vspace{-0.1in}   
\sectiontitle{Honors}
NASA Hubble 25th Anniversary Commendation, HST Science Team \hfill 2016 \\ %	CHECK THIS DATE
Ray and Constance Chandler Fellowship \hfill 2014 - 2015 \\ %$5000?
NASA Group Achievement Award, HST SM4 Servicing Implementation Team \hfill2010 \\
Cliff E. Smith and Ruth Kinnell Graduate Fellowship \hfill 2007 - 2008\\ %$2000
%-------------------------------------------------------------
%----------RESEARCH EXPERIENCE--------------
%-------------------------------------------------------------
\vspace{-0.1in}   
\sectiontitle{Research\\ Interests}
Core collapse supernovae and their progenitor systems\\
Transient astronomy\\
Big data management in astronomy\\
Development and support of astronomical pipelines and archives\\ 
\vspace{-0.1in}   
\sectiontitle{Research\\ Experience} 
%\vspace{0.1in}
Supervisor: Stefano Valenti \hfill 2015-Present\\
Characterizing the properties of massive stars with observations of supernovae from the Swift Observatory, Very Large Array, Keck Observatory, the Gemini Observatory, the Las Cumbres Observatory, the ePESSTO collaboration, the Global Supernova Project, and the DLT40 Survey.\\
\\
Supervisor: Jesus Maiz-Apellaniz \hfill 2015-2018\\
Analyzed long-slit HST/STIS spectra of the 30 Doradus cluster in 
the Large Magellanic Cloud to create a family of UV-optical extinction laws. \\
\\
Supervisor: Michael Wood-Vasey \hfill Fall 2015\\
Corrected telluric lines in spectra from the Kitt Peak Observatory using
precipitable water vapor measurements from nearby GPS stations.\\
\\
Supervisors: Alex Filippenko \& Ori Fox \hfill 2013-2016\\
Searching for the companion star to supernova 1993J using UV and optical 
spectroscopy and imaging from the Hubble Space Telescope.\\
\\
Supervisors: Paul Crowther, Daniel Lennon, \& Nolan Walborn \hfill 2011-2014\\
Characterized the massive stars in the central cluster of 30 Doradus using 
UV, optical, and NIR spectroscopy from the Hubble Space Telescope.\\
\\
Supervisors: Adam Riess \& Steven Rodeny\hfill 2011-2013\\
Searched for high-redshift type Ia supernovae in the CLASH/CANDELS images.\\
\\
Supervisor: Douglas Leonard \hfill 2006-2009 \\
Calibrated the NIR Tully-Fisher Relation with a uniform sample of Cepheid variable stars and used it to find Hubble's Constant.\\
\\
Supervisor: Eric Sandquist \hfill Spring 2009\\
Observed and reduced observations of young open clusters from the Mount Laguna Observatory. \\ 
\\
Supervisor: George Smoot \hfill Summer 2005\\
Created content for the Universe Adventure website as part of the 
Pre-Service Teacher Program.\\
%-------------------------------------------------------------
%--------------------Proposals & Grants---------------
%-------------------------------------------------------------
\vspace{-0.1in}   
\sectiontitle{Proposals\\\& Grants}
\$700, Diversity and Inclusion in Physics Travel Award\hfill 2019\\
\textsc{American Astronomical Society Winter Meeting, 2020}\\
\\
14.63 hours, Gemini Observatory \hfill 2019-2020\\
\textsc{Progenitors of Type II Supernovae from Nebular Spectra}\\
\textbf{P-I: K. Azalee Bostroem}\\
\\
\$39,200, Swift General Observer Program \hfill 2019\\
Co-I: \textsc{High Cadence UV Light Curves of Extremely Young Supernova}\\
P-I: David J. Sand\\
\\
90 nights, Las Cumbres Observatory \hfill 2017 - 2019\\
Co-I: \textsc{The Global Supernova Project}\\
P-I: Stefano Valenti \\
\\
\$40,000, AIP Venture Partnership Fund \hfill 2017-2019\\
\textsc{Developing a Data Carpentry Curriculum for Astronomers and Physicists}\\
{\bf P-I: K. A. Bostroem} \& Rodolfo Montez\\
\\
\$800, FAMOUS Travel Grant \hfill2018\\ 
\textsc{.Astronomy Conference}\\
\\
25.5 hours, Gemini Observatory \hfill 2018 - 2019\\
\textsc{Progenitors of SNe type II from nebular spectra}\\
{\bf P-I: K. Azalee Bostroem} \\
\\
\newpage
14 half nights, Keck Observatory \hfill 2017 - 2019\\
Co-I: \textsc{The Global Supernova Project}\\
P-I: Stefano Valenti \\
%\\
%\$30,000, Swift General Observer Program \hfill 2016-2018 \\
%Co-I\textsc{Early Spectroscopy of Supernovae with Swift and Floyds}\\
%P-I: Stefano Valenti \\
\\
25 orbits, Hubble Space Telescope General Observer Program \hfill 2015, 2017\\ %Cycle 22, Cycle 24
Co-I: \textsc{UV Spectroscopic Signatures from Type Ia Supernovae Strongly Interacting with a Circumstellar Medium}\\
P-I: Ori Fox \\
\\
Nine orbits, Hubble Space Telescope General Observer Program \hfill 2016\\ %
Co-I: \textsc{Long-Lost Companions: A Search for the Binary Secondaries of Three Nearby Supernovae}\\
P-I: Ori Fox \\
\\
15 orbits, Hubble Space Telescope General Observer Program \hfill 2015-2016\\ %Cycle 23
\textsc{The optical-UV extinction law in 30 Doradus}\\
P-I: Jesus Maiz-Apellaniz \\
{\bf Admin P-I: K. Azalee Bostroem} \\
\\
Summer funding, Three orbits, Hubble Space Telescope General Observer Program \hfill 2015\\ %Cycle 22; 
Co-I: \textsc{Uncovering the Putative B-Star Binary Companion of the SN 1993J Progenitor}\\
P-I: Ori Fox \\
\\
Six orbits, Hubble Space Telescope Calibration Program \hfill 2013-2014\\ %%13527, Cycle 21
\textsc{COS NUV Spectroscopic Sensitivity Monitoring}\\
{\bf P-I: K. Azalee Bostroem} \\ % J. Taylor, C. R. Proffitt
\\
23 orbits, Hubble Space Telescope Calibration Program \hfill 2013-2014\\ %13520, Cycle 21
\textsc{COS FUV Spectroscopic Sensitivity Monitoring}\\
{\bf P-I: K. Azalee Bostroem} \\ %J. H. Debes, C. R. Proffitt
\\
12 orbits, Hubble Space Telescope Calibration Program \hfill  2013-2014\\ %13548, Cycle 21
Co-I: \textsc{MAMA Spectroscopic Sensitivity and Focus Monitor Cycle 21}\\ 
P-I: H. Sana \\ %, R. A. Osten, C. R. Proffitt, {\bf K. A. Bostroem}
\\
Five orbits, Hubble Space Telescope Calibration Program \hfill  2013-2014\\ %13544, Cycle 21
Co-I: \textsc{STIS/CCD Spectroscopic Sensitivity Monitor for Cycle 21} \\
P-I: H. Sana\\ %, R. A. Osten, {\bf K. A. Bostroem}, C. R. Proffitt
\\
One orbit, Hubble Space Telescope General Observer Program \hfill  2013-2014\\ %13447, Cycle 21
Co-I: \textsc{The massive monsters living deep in the Tarantula nebula: How massive are they really?} \\
P-I: S. E. de Mink\\ %, S. C. Nieves, {\bf K. A. Bostroem}, P. A. Crowther, et al.
\\
Six orbits, Hubble Space Telescope Calibration Program \hfill 2012-2013\\ %13125, Cycle 20
\textsc{COS NUV Spectroscopic Sensitivity Monitoring}\\
{\bf P-I: K. Azalee Bostroem} \\ %R. A. Osten, C. R. Proffitt
\\
33 orbits, Hubble Space Telescope Calibration Program \hfill 2012-2013\\ %13119, Cycle 20
\textsc{COS FUV Spectroscopic Sensitivity Monitoring}\\
{\bf P-I: K. Azalee Bostroem} \\ %R. A. Osten, C. R. Proffitt
\\
12 orbits, Hubble Space Telescope Calibration Program \hfill  2012-2013\\ %13145, Cycle 20 
Co-I: \textsc{MAMA Spectroscopic Sensitivity and Focus Monitor Cycle 20}\\
P-I: S. T. Holland\\ %, R. A. Osten, C. R. Proffitt, {\bf K. A. Bostroem}
\\
Five orbits, Hubble Space Telescope Calibration Program \hfill  2012-2013\\ %13141, Cycle 20
Co-I: \textsc{STIS/CCD Spectroscopic Sensitivity Monitor for Cycle 20} \\
P-I: S. T. Holland\\ %, R. A. Osten, {\bf K. A. Bostroem}, C. R. Proffitt
\\
12 orbits, Hubble Space Telescope Calibration Program \hfill 2011-2012\\ %12775, Cycle 19
\textsc{MAMA Spectroscopic Sensitivity and Focus Monitor Cycle 19}\\
{\bf P-I: K. Azalee Bostroem} \\ % R. A. Osten, C. R. Proffitt. S. T. Holland
\\
Five orbits, Hubble Space Telescope Calibration Program \hfill 2011-2012\\ %12772, Cycle 19
\textsc{STIS/CCD Spectroscopic Sensitivity Monitor for Cycle 19}\\
{\bf P-I: K. Azalee Bostroem} \\ %. A. Osten, C. R. Proffitt
\\
Eight orbits, Hubble Space Telescope Calibration Program \hfill 2011-2012\\ %12796, Cycle 19, 
Co-I: \textsc{Second COS FUV Lifetime Position: Focus Sweep Enabling Program (FENA3)} \\
P-I: C. Oliveira \\%, {\bf K. A. Bostroem}
\\
44 orbits, Hubble Space Telescope Calibration Program \hfill 2011-2012\\ %12715, Cycle 19
Co-I: \textsc{COS FUV Spectroscopic Sensitivity Monitoring} \\
P-I: R. A. Osten\\%, C. D. Keyes, D. J. Sahnow, A. Aloisi, {\bf K. A. Bostroem}
\\
12 orbits, Hubble Space Telescope Calibration Program \hfill 2010-2011\\ %12414, Cycle 18
Co-I: \textsc{MAMA Spectroscopic Sensitivity and Focus Monitor Cycle 18} \\
P-I: R. A. Osten \\%, C. R. Proffitt, {\bf K. A. Bostroem}
\\
Five orbits, Hubble Space Telescope Calibration Program \hfill 2010-2011\\ %12411, Cycle 18, 
Co-I: \textsc{STIS/CCD Spectroscopic Sensitivity Monitor for Cycle 18} \\
P-I: R. A. Osten\\%, {\bf K. A. Bostroem}, C. R. Proffitt
%Nine HST Calibration Proposals including with Bostroem as Co-I\\ % How do I include this? 15 HST Calibration Proposals including {\bf 6 with Bostroem as PI}
%-------------------------------------------------------------
%----------------TALKS-----------------------
%-------------------------------------------------------------
\vspace{-0.1in}  
\sectiontitle{Talks}
Astrophysics and Cosmology Symposium, Davis, CA \hfill September 2019\\\vspace{-0.1in}  
\\
The Extragalactic Explosive Universe, Garching, DE\hfill September 2019\\ \vspace{-0.1in}  
\\
Transients Around the Globe, Weizmann Institute of Science, Rehovot, Israel\hfill April 2019\\ \vspace{-0.1in}  %invited
\\
Astronomy and Astrophysics Seminar, Tel Aviv University, Tel Aviv, Israel \hfill April 2019\\ \vspace{-0.1in}  %invited
\\
Time-Domain Follow-up Observations with Las Cumbres Observatory, American Astronomical Society 233rd Meeting, Seattle, WA  \hfill January 2019\\ \vspace{-0.1in}  %invited
\\
American Astronomical Society 233rd Meeting,  Seattle, WA \hfill January 2019\\\vspace{-0.1in}  
\\
Astrophysics and Cosmology Symposium, Davis, CA \hfill September 2019\\\vspace{-0.1in}  
\\
Keynote; Python in Astronomy, Lorentz Center, Leiden, NL\hfill May 2017\\ \vspace{-0.1in}  %invited
\\
Northern California Graduate Physics Admissions Bootcamp at UC Davis, Davis, CA\hfill August 2017\\ \vspace{-0.1in}  %invited
\\
California State Summer School for Mathematics and Science at UC Davis, Davis, CA \hfill August 2017\\ \vspace{-0.1in}  %invited
\\
Stellar and Extragalactic Astronomy Lunch at the Goddard Space Flight Center, Greenbelt, MD \hfill June 2014\\  \vspace{-0.1in}  %invited
\\
STScI TIPS/JIM Monthly Meeting, Baltimore, MD\hfill March 2012\\\vspace{-0.1in}  
%-------------------------------------------------------------
%------------MENTORED STUDENTS---------------
%-------------------------------------------------------------
\newpage
\vspace{-0.1in}  
\sectiontitle{Mentored \\Students} 
%\vspace{0.1in}
Isabele Ye (w/ Stefano Valenti) \hfill 2018-2019\\
Monitor the data acquisition of Type II SNe observed with the Las Cumbres Observatory as part of the Global Supernova Project to ensure the complete light curve is observed.\\
\\
Gayle Zhang (w/ Stefano Valenti) \hfill 2017-2018\\
Derived the radius around saturated stars within which supernova detections in the DLT40 survey should be considered false positives.\\
\\
Martha Saladino (w/ Justin Ely)\hfill Summer 2012\\
Refined the flux calibration of the HST/COS spectroscopy by characterizing throughput as a function of time.\\
\\
Kenneth Hart \hfill 2011-2012\\
Improved the calibration of the HST/COS and HST/STIS NUV detectors by characterizing the vignetted region on each.\\
\\
Inna Bojinova \hfill Summer 2011\\
Improved the flux calibration of the HST/COS spectroscopy by characterizing throughput as a function of wavelength.\\
%-------------------------------------------------------------
%--------------------SKILLS-------------------------------
%-------------------------------------------------------------
\vspace{-0.1in}  
\sectiontitle{Skills}
Working knowledge of UNIX, IDL, Python, PyRAF, SQL, Git/GitHub, HTML\\
Experience using cluster computing resources at UC Davis\\ 
%-------------------------------------------------------------
%--------------------SOFTWARE------------------------
%-------------------------------------------------------------
%\vspace{-0.1in}   
%\sectiontitle{Software \\Contributions}
%\textsc{Photutils: Photometry tools}, Bradley, L., [7 people], {\bf Bostroem,  K. A.}, et al., 2016, ASCL \\
%
%-------------------------------------------------------------
%-------OBSERVING EXPERIENCE----------------
%-------------------------------------------------------------
\vspace{-0.1in}   
\sectiontitle{Observing\\Experience}
6 nights, Optical Spectroscopy and Imaging ({\it Lick Observatory/KAST, Nickel, ShARCS})\\
14 half nights, Optical Spectroscopy ({\it Keck/LRIS})\\
3 nights, Optical Imaging ({\it Mount Laguna Observatory/40-inch})
%
%-------------------------------------------------------------
%----------INSTRUMENT SUPPORT----------------
%-------------------------------------------------------------
\vspace{-0.1in}   
\sectiontitle{Instrument\\Support} 
\textsc{HST/COS and HST/STIS Calibration Pipeline Lead}\\
Supervised the development and testing of the HST/COS and HST/STIS pipelines coordinating between the scientists, software developers, and archive team. \\
%\begin{itemize}
%\item Organized and lead bi-weekly meetings
%\item Supervised the testing of the HST/COS and HST/STIS calibration pipelines
%\item Coordinated the HST/COS and HST/STIS calibration pipeline work with the HST/COS and HST/STIS teams and the archive and pipeline developers
%\item Prioritized HST/COS and HST/STIS pipeline development items
%\item Oversaw the implementation of improvements to the HST/COS and HST/STIS calibration pipeline
%\end{itemize}
\\
\textsc{Instrument Monitoring and Calibration}\\
Improve the quality of HST/STIS and HST/COS observations through the development of monitoring tools and the development and testing of new calibration reference files.\\
%As a member of the HST/STIS team I created and tested HST/STIS CCD Dark and Bias reference files. I also characterized the flux calibration and blaze shift correction for the HST/STIS echelle modes following Servicing Mission 4 and developed tools to monitor the HST/STIS time-dependent sensitivity. I continued this work on the HST/COS team developing tools to monitor the HST/COS FUV and NUV time-dependent sensitivity and create and test new reference files as needed. Additionally, I have created and tested throughput reference files for HST/COS and HST/STIS to be used in the Exposure Time Calculator.\\
\\
\textsc{User Support Deputy}\\
Maintained the internal and external web pages, created and delivered Space Telescope Analysis Newsletters, answered help desk questions, and tracked user support issues\\
%As user support deputy for the HST/COS and HST/STIS teams, I maintained the internal and external web pages, created and delivered Space Telescope Analysis Newsletters, and tracked user support issues within the team. I also support the Spectrographs? Help Desk, answering questions from the community about HST/COS, HST/STIS, HST/GHRS, and HST/FOS.\\
%HST/COS and HST/STIS Calibration Pipeline Lead, Community Support Deputy, and instrument monitoring and calibration
%-------------------------------------------------------------
%------------------PUBLICATIONS---------------------
%-------------------------------------------------------------

%-------------------------------------------------------------
%----------REFEREED PUBLICATIONS------------
%-------------------------------------------------------------


%

%Use this in ADS:
%\item{\textsc{%T}\\\ \n%G, %Y, %q, %V \n\color{blue}\href{%u}{(ADS link)}\color{black}}\\\\\n
%\vspace{-0.1in}  
\sectiontitle{First Author \\Refereed \\Publications\\\color{blue}\href{https://ui.adsabs.harvard.edu/public-libraries/fYopgcTYQIyT_QbQD0bV9w}{(ADS link)}\color{black}}
\vspace{-0.5in}  
\begin{revnumerate}[5]
\item{\textsc{Considering the Single and Binary Origins of SN 2017eaw}\\ 
\textbf{Bostroem, K. A.}, Zapartas, E., Koplitz, B., Williams, B., Tran, D., Dolphin, A., 2023, submitted}\\

\item{\textsc{Early Spectroscopy and Dense Circumstellar Medium Interaction in SN~2023ixf}\\ 
\textbf{Bostroem, K. A.}, Pearson, J., Shrestha, M., Sand, D. J., Valenti, S., Jha, S. W., Andrews, J. E., Smith, N., Terreran, G., Green, E., Dong, Y., Lundquist, M., Haislip, J., Hoang, E. T., Hosseinzadeh, G., Janzen, D., Jencson, J. E., Kouprianov, V., Paraskeva, E., Meza Retamal, N. E., Reichart, D. E., Arcavi, I., Bonanos, A. Z., Coughlin, M. W., Farah, J., Hawley, S., Hebb, L., Hiramatsu, D., Howell, D. A., Iijima, T., Ilyin, I., McCully, C., Moran, S., Morris, B. M., Mura, A. C., Newsome, M., Pabst, M. T., Ochner, P., Padilla Gonzalez, E., Pastorello, A., Pellegrino, C., Ravi, A. P., Reguitti, A., Salo, L., Vinko, J., Wheeler, J. C., Williams, G. G., Wyatt, S., 2023, accepted to ApJL, arXiv, 
\color{blue}\href{https://ui.adsabs.harvard.edu/abs/2023arXiv230610119B}{(ADS link)}\color{black}}\\

\item{\textsc{SN 2022acko: The First Early Far-ultraviolet Spectra of a Type IIP Supernova}\\ 
\textbf{Bostroem, K. A.}, Dessart, L., Hillier, D. J., Lundquist, M., Andrews, J. E., Sand, D. J., Dong, Y., Valenti, S., Haislip, J., Hoang, E. T., Hosseinzadeh, G., Janzen, D., Jencson, J. E., Jha, S. W., Kouprianov, V., Pearson, J., Meza Retamal, N. E., Reichart, D. E., Shrestha, M., Ashall, C., Baron, E., Brown, P. J., DerKacy, J. M., Farah, J., Galbany, L., González Hernández, J. I., Green, E., Hoeflich, P., Howell, D. A., Kwok, L. A., McCully, C., Müller-Bravo, T. E., Newsome, M., Gonzalez, E. P., Pellegrino, C., Rho, J., Rowe, M., Schwab, M., Shahbandeh, M., Smith, N., Strader, J., Terreran, G., Van Dyk, S. D., Wyatt, S., 2023, ApJL, 953 
\color{blue}\href{https://ui.adsabs.harvard.edu/abs/2023ApJ...953L..18B}{(ADS link)}\color{black}}\\

    \item{\textsc{Discovery and Rapid Follow-up Observations of the Unusual Type II SN 2018ivc in NGC 1068}\\ 
    \textbf{Bostroem, K. A.}, Valenti, S., Sand, D. J., Andrews, J. E., Van Dyk, S. D., Galbany, L., Pooley, D., Amaro, R. C., Smith, N., Yang, S., Anupama, G. C., Arcavi, I., Baron, E., Brown, P. J., Burke, J., Cartier, R., Hiramatsu, D., Dastidar, R., DerKacy, J. M., Dong, Y., Egami, E., Ertel, S., Filippenko, A. V., Fox, O. D., Haislip, J., Hosseinzadeh, G., Howell, D. A., Gangopadhyay, A., Jha, S. W., Kouprianov, V., Kumar, B., Lundquist, M., Milisavljevic, D., McCully, C., Milne, P., Misra, K., Reichart, D. E., Sahu, D. K., Sai, H., Singh, A., Smith, P. S., Vinko, J., Wang, X., Wang, Y., Wheeler, J. C., Williams, G. G., Wyatt, S., Zhang, J., Zhang, X., 2020, ApJ, 895 
    \color{blue}\href{https://ui.adsabs.harvard.edu/abs/2020ApJ...895...31B}{(ADS link)}\color{black}}\\

    \item{\textsc{Signatures of circumstellar interaction in the Type IIL supernova ASASSN-15oz}\\ 
    \textbf{Bostroem, K. A.}, Valenti, S., Horesh, A., Morozova, V., Kuin, N. P. M., Wyatt, S., Jerkstrand, A., Sand, D. J., Lundquist, M., Smith, M., Sullivan, M., Hosseinzadeh, G., Arcavi, I., Callis, E., Cartier, R., Gal-Yam, A., Galbany, L., Gutiérrez, C., Howell, D. A., Inserra, C., Kankare, E., L\'{o}pez, K. M., McCully, C., Pignata, G., Piro, A. L., Rodr\'{i}guez, \'{O}., Smartt, S. J., Smith, K. W., Yaron, O., Young, D. R., 2019, MNRAS, 485 
    \color{blue}\href{https://ui.adsabs.harvard.edu/abs/2019MNRAS.485.5120B}{(ADS link)}\color{black}}\\
\end{revnumerate}


\vspace{0.1in}
\sectiontitle{Co-Author \\ Refereed\\Publications\\\color{blue}\href{https://ui.adsabs.harvard.edu/public-libraries/fYopgcTYQIyT_QbQD0bV9w}{(ADS link)}\color{black}}

\begin{revnumerate}[67]
\item{\textsc{Ground-based and JWST Observations of SN 2022pul: I. Unusual Signatures of Carbon, Oxygen, and Circumstellar Interaction in a Peculiar Type Ia Supernova}\\ 
Siebert, M. R., Kwok, L. A., Johansson, J., Jha, S. W., Blondin, S., Dessart, L., Foley, R. J., Hillier, D. J., Larison, C., Pakmor, R., Temim, T., Andrews, J. E., Auchettl, K., Badenes, C., Barna, B., \textbf{Bostroem, K. A.}, Brenner Newman, M. J., Brink, T. G., José Bustamante-Rosell, M., Camacho-Neves, Y., Clocchiatti, A., Coulter, D. A., Davis, K. W., Deckers, M., Dimitriadis, G., Dong, Y., Farah, J., Filippenko, A. V., Flörs, A., Fox, O. D., Garnavich, P., Padilla Gonzalez, E., Graur, O., Hambsch, F.-J., Hosseinzadeh, G., Howell, D. A., Hughes, J. P., Kerzendorf, W. E., Le Saux, X. K., Maeda, K., Maguire, K., McCully, C., Mihalenko, C., Newsome, M., O'Brien, J. T., Pearson, J., Pellegrino, C., Pierel, J. D. R., Polin, A., Rest, A., Rojas-Bravo, C., Sand, D. J., Schwab, M., Shahbandeh, M., Shrestha, M., Smith, N., Strolger, L.-G., Szalai, T., Taggart, K., Terreran, G., Terwel, J. H., Tinyanont, S., Valenti, S., Vinkó, J., Wheeler, J. C., Yang, Y., Zheng, W., Ashall, C., Derkacy, J. M., Galbany, L., Hoeflich, P., Hsiao, E., De Jaeger, T., Lu, J., Maund, J., Medler, K., Morrell, N., Shappee, B. J., Stritzinger, M., Suntzeff, N., Tucker, M., Wang, L., 2023, arXiv, 
\color{blue}\href{https://ui.adsabs.harvard.edu/abs/2023arXiv230812449S}{(ADS link)}\color{black}}\\

\item{\textsc{Ground-based and JWST Observations of SN 2022pul: II. Evidence from Nebular Spectroscopy for a Violent Merger in a Peculiar Type-Ia Supernova}\\ 
Kwok, L. A., Siebert, M. R., Johansson, J., Jha, S. W., Blondin, S., Dessart, L., Foley, R. J., Hillier, D. J., Larison, C., Pakmor, R., Temim, T., Andrews, J. E., Auchettl, K., Badenes, C., Barna, B., \textbf{Bostroem, K. A.}, Brenner Newman, M. J., Brink, T. G., Bustamante-Rosell, M. J., Camacho-Neves, Y., Clocchiatti, A., Coulter, D. A., Davis, K. W., Deckers, M., Dimitriadis, G., Dong, Y., Farah, J., Filippenko, A. V., Flors, A., Fox, O. D., Garnavich, P., Padilla Gonzalez, E., Graur, O., Hambsch, F.-J., Hosseinzadeh, G., Howell, D. A., Hughes, J. P., Kerzendorf, W. E., Le Saux, X. K., Maeda, K., Maguire, K., McCully, C., Mihalenko, C., Newsome, M., O'Brien, J. T., Pearson, J., Pellegrino, C., Pierel, J. D. R., Polin, A., Rest, A., Rojas-Bravo, C., Sand, D. J., Schwab, M., Shahbandeh, M., Shrestha, M., Smith, N., Strolger, L.-G., Szalai, T., Taggart, K., Terreran, G., Terwel, J. H., Tinyanont, S., Valenti, S., Vinko, J., Wheeler, J. C., Yang, Y., Zheng, W., Ashall, C., DerKacy, J. M., Galbany, L., Hoeflich, P., de Jaeger, T., Lu, J., Maund, J., Medler, K., Morrell, N., Shappee, B. J., Stritzinger, M., Suntzeff, N., Tucker, M., Wang, L., 2023, arXiv, 
\color{blue}\href{https://ui.adsabs.harvard.edu/abs/2023arXiv230812450K}{(ADS link)}\color{black}}\\

\item{\textsc{SN 2022joj: A Potential Double Detonation with a Thin Helium shell}\\ 
Padilla Gonzalez, E., Howell, D. A., Terreran, G., McCully, C., Newsome, M., Burke, J., Farah, J., Pellegrino, C., \textbf{Bostroem, K. A.}, Hosseinzadeh, G., Pearson, J., Sand, D. J., Shrestha, M., Smith, N., Dong, Y., Meza Retamal, N., Valenti, S., Boos, S., Shen, K. J., Townsley, D., Galbany, L., Piscarreta, L., Foley, R. J., Bustamante-Rosell, M. J., Coulter, D. A., Chornock, R., Davis, K. W., Dickinson, C. B., Jones, D. O., Kutcka, J., Le Saux, X. K., Rojas-Bravo, C. R., Taggart, K., Tinyanont, S., Yang, G., Jha, S. W., Margutti, R., 2023, arXiv, 
\color{blue}\href{https://ui.adsabs.harvard.edu/abs/2023arXiv230806334P}{(ADS link)}\color{black}}\\

\item{\textsc{A comprehensive optical search for pre-explosion outbursts from the quiescent progenitor of SN~2023ixf}\\ 
Dong, Y., Sand, D. J., Valenti, S., \textbf{Bostroem, K. A.}, Andrews, J. E., Hosseinzadeh, G., Hoang, E., Janzen, D., Jencson, J. E., Lundquist, M., Meza Retamal, N. E., Pearson, J., Shrestha, M., Haislip, J., Kouprianov, V., Reichart, D. E., 2023, accepted, arXiv, 
\color{blue}\href{https://ui.adsabs.harvard.edu/abs/2023arXiv230702539D}{(ADS link)}\color{black}}\\

\item{\textsc{From Discovery to the First Month of the Type II Supernova 2023ixf: High and Variable Mass Loss in the Final Year Before Explosion}\\ 
Hiramatsu, D., Tsuna, D., Berger, E., Itagaki, K., Goldberg, J. A., Gomez, S., De, K., Hosseinzadeh, G., \textbf{Bostroem, K. A.}, Brown, P. J., Arcavi, I., Bieryla, A., Blanchard, P. K., Esquerdo, G. A., Farah, J., Howell, D. A., Matsumoto, T., McCully, C., Newsome, M., Padilla Gonzalez, E., Pellegrino, C., Rhee, J., Terreran, G., Vinkó, J., Wheeler, J. C., 2023, accepted, arXiv, 
\color{blue}\href{https://ui.adsabs.harvard.edu/abs/2023arXiv230703165H}{(ADS link)}\color{black}}\\

\item{\textsc{High resolution spectroscopy of SN~2023ixf's first week: Engulfing the Asymmetric Circumstellar Material}\\ 
Smith, N., Pearson, J., Sand, D. J., Ilyin, I., \textbf{Bostroem, K. A.}, Hosseinzadeh, G., Shrestha, M., 2023, accepted, arXiv, 
\color{blue}\href{https://ui.adsabs.harvard.edu/abs/2023arXiv230607964S}{(ADS link)}\color{black}}\\

\item{\textsc{A Low-Mass Helium Star Progenitor Model for the Type Ibn SN 2020nxt}\\ 
Wang, Q., Goel, A., Dessart, L., Fox, O. D., Shahbandeh, M., Rest, S., Rest, A., Groh, J. H., Allan, A., Fransson, C., Smith, N., Hosseinzadeh, G., Filippenko, A. V., Andrews, J., \textbf{Bostroem, K. A.}, Brink, T. G., Brown, P., Burke, J., Chevalier, R., Clayton, G. C., Dai, M., Davis, K. W., Foley, R. J., Gomez, S., Harris, C., Hiramatsu, D., Howell, D. A., Jennings, C., Jha, S. W., Kasliwal, M. M., Kelly, P. L., Kool, E. C., Liu, E., Ma, E., McCully, C., Miller, A. M., Murakami, Y., Pellegrino, C., Padilla Gonzalez, E., Perera, D., Pierel, J., Rojas-Bravo, C., Siebert, M. R., Sollerman, J., Szalai, T., Tinyanont, S., Van Dyk, S. D., Zheng, W., Chambers, K. C., Coulter, D. A., de Boer, T., Earl, N., Farias, D., Gall, C., McGill, P., Ransome, C. L., Taggart, K., Villar, V. A., 2023, arXiv, 
\color{blue}\href{https://ui.adsabs.harvard.edu/abs/2023arXiv230505015W}{(ADS link)}\color{black}}\\

    \item{\textsc{Early Lightcurves of Type Ia Supernovae are Consistent with Nondegenerate Progenitor Companions}\\ 
    Burke, J., Howell, D. A., Sand, D. J., Amaro, R. C., Brown, P. J., Andrews, J. E., \textbf{Bostroem, K. A.}, Dong, Y., Haislip, J., Hiramatsu, D., Hosseinzadeh, G., Kouprianov, V., Lundquist, M. J., McCully, C., Pellegrino, C., Reichart, D., Tartaglia, L., Valenti, S., Yang, S., 2022, arXiv, 
    \color{blue}\href{https://ui.adsabs.harvard.edu/abs/2022arXiv220707681B}{(ADS link)}\color{black}}\\

  \item{\textsc{Identifying the SN 2022acko progenitor with JWST}\\ 
Van Dyk, S. D., \textbf{Bostroem, K. A.}, Zheng, W., Brink, T. G., Fox, O. D., Andrews, J. E., Filippenko, A. V., Dong, Y., Hoang, E., Hosseinzadeh, G., Janzen, D., Jencson, J. E., Lundquist, M. J., Meza, N., Milisavljevic, D., Pearson, J., Sand, D. J., Shrestha, M., Valenti, S., Howell, D. A., 2023, MNRAS, 524 
\color{blue}\href{https://ui.adsabs.harvard.edu/abs/2023MNRAS.524.2186V}{(ADS link)}\color{black}}\\

\item{\textsc{SN 2020bio: A Double-peaked, H-poor Type IIb Supernova with Evidence of Circumstellar Interaction}\\ 
Pellegrino, C., Hiramatsu, D., Arcavi, I., Howell, D. A., \textbf{Bostroem, K. A.}, Brown, P. J., Burke, J., Elias-Rosa, N., Itagaki, K., Kaneda, H., McCully, C., Modjaz, M., Padilla Gonzalez, E., Pritchard, T. A., Yesmin, N., 2023, ApJ, 954 
\color{blue}\href{https://ui.adsabs.harvard.edu/abs/2023ApJ...954...35P}{(ADS link)}\color{black}}\\

\item{\textsc{The Early Light Curve of SN 2023bee: Constraining Type Ia Supernova Progenitors the Apian Way}\\ 
Hosseinzadeh, G., Sand, D. J., Sarbadhicary, S. K., Ryder, S. D., Jha, S. W., Dong, Y., \textbf{Bostroem, K. A.}, Andrews, J. E., Hoang, E., Janzen, D., Jencson, J. E., Lundquist, M., Meza Retamal, N. E., Pearson, J., Shrestha, M., Valenti, S., Wyatt, S., Farah, J., Howell, D. A., McCully, C., Newsome, M., Padilla Gonzalez, E., Pellegrino, C., Terreran, G., Alzaabi, M., Green, E. M., Gurney, J. L., Milne, P. A., Ridenhour, K. I., Smith, N., Robles, P. S., Kwok, L. A., Schwab, M., Gromadzki, M., Buckley, D. A. H., Itagaki, K., Hiramatsu, D., Chomiuk, L., Lundqvist, P., Haislip, J., Kouprianov, V., Reichart, D. E., 2023, ApJL, 953 
\color{blue}\href{https://ui.adsabs.harvard.edu/abs/2023ApJ...953L..15H}{(ADS link)}\color{black}}\\

\item{\textsc{AT 2021loi: A Bowen Fluorescence Flare with a Rebrightening Episode Occurring in a Previously Known AGN}\\ 
Makrygianni, L., Trakhtenbrot, B., Arcavi, I., Ricci, C., Lam, M. C., Horesh, A., Sfaradi, I., \textbf{Bostroem, K. A.,} Hosseinzadeh, G., Howell, D. A., Pellegrino, C., Fender, R., Green, D. A., Williams, D. R. A., Bright, J., 2023, ApJ, 953 
\color{blue}\href{https://ui.adsabs.harvard.edu/abs/2023ApJ...953...32M}{(ADS link)}\color{black}}\\

\item{\textsc{A Luminous Red Supergiant and Dusty Long-period Variable Progenitor for SN 2023ixf}\\ 
Jencson, J. E., Pearson, J., Beasor, E. R., Lau, R. M., Andrews, J. E., \textbf{Bostroem, K. A.}, Dong, Y., Engesser, M., Gomez, S., Guolo, M., Hoang, E., Hosseinzadeh, G., Jha, S. W., Karambelkar, V., Kasliwal, M. M., Lundquist, M., Meza Retamal, N. E., Rest, A., Sand, D. J., Shahbandeh, M., Shrestha, M., Smith, N., Strader, J., Valenti, S., Wang, Q., Zenati, Y., 2023, ApJL, 952 
\color{blue}\href{https://ui.adsabs.harvard.edu/abs/2023ApJ...952L..30J}{(ADS link)}\color{black}}\\

\item{\textsc{Shock Cooling and Possible Precursor Emission in the Early Light Curve of the Type II SN 2023ixf}\\ 
Hosseinzadeh, G., Farah, J., Shrestha, M., Sand, D. J., Dong, Y., Brown, P. J., \textbf{Bostroem, K. A.}, Valenti, S., Jha, S. W., Andrews, J. E., Arcavi, I., Haislip, J., Hiramatsu, D., Hoang, E., Howell, D. A., Janzen, D., Jencson, J. E., Kouprianov, V., Lundquist, M., McCully, C., Meza Retamal, N. E., Modjaz, M., Newsome, M., Padilla Gonzalez, E., Pearson, J., Pellegrino, C., Ravi, A. P., Reichart, D. E., Smith, N., Terreran, G., Vinkó, J., 2023, ApJL, 953 
\color{blue}\href{https://ui.adsabs.harvard.edu/abs/2023ApJ...953L..16H}{(ADS link)}\color{black}}\\

\item{\textsc{Shock Cooling and Possible Precursor Emission in the Early Light Curve of the Type II SN 2023ixf}\\ 
Hosseinzadeh, G., Farah, J., Shrestha, M., Sand, D. J., Dong, Y., Brown, P. J., \textbf{Bostroem, K. A.}, Valenti, S., Jha, S. W., Andrews, J. E., Arcavi, I., Haislip, J., Hiramatsu, D., Hoang, E., Howell, D. A., Janzen, D., Jencson, J. E., Kouprianov, V., Lundquist, M., McCully, C., Meza Retamal, N. E., Modjaz, M., Newsome, M., Padilla Gonzalez, E., Pearson, J., Pellegrino, C., Ravi, A. P., Reichart, D. E., Smith, N., Terreran, G., Vinkó, J., 2023, ApJL, 953 
\color{blue}\href{https://ui.adsabs.harvard.edu/abs/2023ApJ...953L..16H}{(ADS link)}\color{black}}\\

\item{\textsc{Observational Properties of a Bright Type lax SN 2018cni and a Faint Type Iax SN 2020kyg}\\ 
Singh, M., Sahu, D. K., Dastidar, R., Barna, B., Misra, K., Gangopadhyay, A., Howell, D. A., Jha, S. W., Im, H., Taggart, K., Andrews, J., Hiramatsu, D., Teja, R. S., Pellegrino, C., Foley, R. J., Joshi, A., Anupama, G. C., \textbf{Bostroem, K. A.}, Burke, J., Camacho-Neves, Y., Dutta, A., Kwok, L. A., McCully, C., Pan, Y.-C., Siebert, M., Srivastav, S., Szalai, T., Swift, J. J., Yang, G., Zhou, H., DiLullo, N., Scheer, J., 2023, ApJ, 953 
\color{blue}\href{https://ui.adsabs.harvard.edu/abs/2023ApJ...953...93S}{(ADS link)}\color{black}}\\

\item{\textsc{Near-infrared and Optical Observations of Type Ic SN 2021krf: Luminous Late-time Emission and Dust Formation}\\ 
Ravi, A. P., Rho, J., Park, S., Park, S. H., Yoon, S.-C., Geballe, T. R., Vinkó, J., Tinyanont, S., \textbf{Bostroem, K. A.}, Burke, J., Hiramatsu, D., Howell, D. A., McCully, C., Newsome, M., Padilla Gonzalez, E., Pellegrino, C., Cartier, R., Pritchard, T., Andersen, M., Blinnikov, S., Dong, Y., Blanchard, P., Kilpatrick, C. D., Hoeflich, P., Valenti, S., Filippenko, A. V., Suntzeff, N. B., Seok, J. Y., Könyves-Tóth, R., Foley, R. J., Siebert, M. R., Jones, D. O., 2023, ApJ, 950 
\color{blue}\href{https://ui.adsabs.harvard.edu/abs/2023ApJ...950...14R}{(ADS link)}\color{black}}\\

\item{\textsc{SN 2017fzw: A Fast-Expanding Type Ia Supernova with Transitional Features}\\ 
Huang, J., Li, Y., Zeng, X., Zheng, S., Bird, S. A., Zhang, J., Esamdin, A., Iskandar, A., \textbf{Bostroem, K. A.}, Zeng, S., Xiao, Y., Huang, Y., Howell, D. A., McCully, C., Li, W., Zhang, T., Wang, L., Hu, L., 2023, Univ, 9 
\color{blue}\href{https://ui.adsabs.harvard.edu/abs/2023Univ....9..295H}{(ADS link)}\color{black}}\\

\item{\textsc{What Does the Geometry of the H$\beta$ BLR Depend On?}\\ 
Villafaña, L., Williams, P. R., Treu, T., Brewer, B. J., Barth, A. J., U, V., Bennert, V. N., Guo, H., Bentz, M. C., Canalizo, G., Filippenko, A. V., Gates, E., Joner, M. D., Malkan, M. A., Woo, J.-H., Abolfathi, B., Bohn, T., \textbf{Bostroem, K. A.}, Brandel, A., Brink, T. G., Channa, S., Cosens, M., Donohue, E., Halevi, G., Hood, C. E., Horst, J. C., de Kouchkovsky, M., Kuhn, B., Leonard, D. C., Michel, R., B. Olaes, M. K., Park, D., Runco, J. N., Sexton, R. O., Shivvers, I., Spencer, C. L., Stahl, B. E., Stegman, S., Walsh, J. L., Zheng, W., 2023, ApJ, 948 
\color{blue}\href{https://ui.adsabs.harvard.edu/abs/2023ApJ...948...95V}{(ADS link)}\color{black}}\\

    \item{\textsc{Circumstellar Medium Interaction in SN 2018lab, A Low-Luminosity II-P Supernova observed with TESS}\\ 
    Pearson, J., Hosseinzadeh, G., Sand, D. J., Andrews, J. E., Jencson, J. E., Dong, Y., \textbf{Bostroem, K. A.}, Valenti, S., Janzen, D., Meza Retamal, N., Lundquist, M. J., Wyatt, S., Amaro, R. C., Burke, J., Howell, D. A., McCully, C., Hiramatsu, D., Jha, S. W., Smith, N., Haislip, J., Kouprianov, V., Reichart, D. E., Yang, Y., Roy, R., Rho, J., 2022, 2023, ApJ, 945 , 
    \color{blue}\href{https://ui.adsabs.harvard.edu/abs/2023ApJ...945..107P}{(ADS link)}\color{black}}\\

    \item{\textsc{Limit on Supernova Emission in the Brightest Gamma-Ray Burst, GRB 221009A}\\ 
    Shrestha, M., Sand, D. J., Alexander, K. D., \textbf{Bostroem, K. A.}, Hosseinzadeh, G., Pearson, J., Aghakhanloo, M., Vinkó, J., Andrews, J. E., Jencson, J. E., Lundquist, M. J., Wyatt, S., Howell, D. A., McCully, C., Gonzalez, E. P., Pellegrino, C., Terreran, G., Hiramatsu, D., Newsome, M., Farah, J., Jha, S. W., Smith, N., Wheeler, J. C., Martínez-Vázquez, C., Carballo-Bello, J. A., Drlica-Wagner, A., James, D. J., Mutlu-Pakdil, B., Stringfellow, G. S., Sakowska, J. D., Noël, N. E. D., Bom, C. R., Kuehn, K., 2023, ApJL, 946 
    \color{blue}\href{https://ui.adsabs.harvard.edu/abs/2023ApJ...946L..25S}{(ADS link)}\color{black}}\\
    
        \item{\textsc{The origin and evolution of the normal Type Ia SN 2018aoz with infant-phase reddening and excess emission}\\ 
    Ni, Y. Q., Moon, D.-S., Drout, M. R., Polin, A., Sand, D. J., Gonz\'{a}lez-Gait\'{a}n, S., Kim, S. C., Lee, Y., Park, H. S., Howell, D. A., Nugent, P. E., Piro, A. L., Brown, P. J., Galbany, L., Burke, J., Hiramatsu, D., Hosseinzadeh, G., Valenti, S., Afsariardchi, N., Andrews, J. E., Antoniadis, J., Beaton, R. L., \textbf{Bostroem, K. A.}, Carlberg, R. G., Cenko, S. B., Cha, S.-M., Dong, Y., Gal-Yam, A., Haislip, J., Holoien, T. W.-S., Johnson, S. D., Kouprianov, V., Lee, Y., Matzner, C. D., Morrell, N., Mccully, C., Pignata, G., Reichart, D. E., Rich, J., Ryder, S. D., Smith, N., Wyatt, S., Yang, S., 2022, 2023, ApJ, 946, 
    \color{blue}\href{https://ui.adsabs.harvard.edu/abs/2023ApJ...946....7N}{(ADS link)}\color{black}}\\

    \item{\textsc{JWST Low-resolution MIRI Spectral Observations of SN 2021aefx: High-density Burning in a Type Ia Supernova}\\ 
    DerKacy, J. M., Ashall, C., Hoeflich, P., Baron, E., Shappee, B. J., Baade, D., Andrews, J., \textbf{Bostroem, K. A.}, Brown, P. J., Burns, C. R., Burrow, A., Cikota, A., de Jaeger, T., Do, A., Dong, Y., Dominguez, I., Galbany, L., Hsiao, E. Y., Karamehmetoglu, E., Krisciunas, K., Kumar, S., Lu, J., Evans, T. B. M., Maund, J. R., Mazzali, P., Medler, K., Morrell, N., Patat, F., Phillips, M. M., Shahbandeh, M., Stangl, S., Stevens, C. P., Stritzinger, M. D., Suntzeff, N. B., Telesco, C. M., Tucker, M. A., Valenti, S., Wang, L., Yang, Y., Jha, S. W., Kwok, L. A., 2023, ApJL, 945 
    \color{blue}\href{https://ui.adsabs.harvard.edu/abs/2023ApJ...945L...2D}{(ADS link)}\color{black}}\\

    \item{\textsc{Forbidden hugs in pandemic times. IV. Panchromatic evolution of three luminous red novae}\\ 
    Pastorello, A., Valerin, G., Fraser, M., Reguitti, A., Elias-Rosa, N., Filippenko, A. V., Rojas-Bravo, C., Tartaglia, L., Reynolds, T. M., Valenti, S., Andrews, J. E., Ashall, C., \textbf{Bostroem, K. A.}, Brink, T. G., Burke, J., Cai, Y.-Z., Cappellaro, E., Coulter, D. A., Dastidar, R., Davis, K. W., Dimitriadis, G., Fiore, A., Foley, R. J., Fugazza, D., Galbany, L., Gangopadhyay, A., Geier, S., Gutierrez, C. P., Haislip, J., Hiramatsu, D., Holmbo, S., Howell, D. A., Hsiao, E. Y., Hung, T., Jha, S. W., Kankare, E., Karamehmetoglu, E., Kilpatrick, C. D., Kotak, R., Kouprianov, V., Kravtsov, T., Kumar, S., Li, Z.-T., Lundquist, M. J., Lundqvist, P., Matilainen, K., Mazzali, P. A., McCully, C., Misra, K., Morales-Garoffolo, A., Moran, S., Morrell, N., Newsome, M., Padilla Gonzalez, E., Pan, Y.-C., Pellegrino, C., Phillips, M. M., Pignata, G., Piro, A. L., Reichart, D. E., Rest, A., Salmaso, I., Sand, D. J., Siebert, M. R., Smartt, S. J., Smith, K. W., Srivastav, S., Stritzinger, M. D., Taggart, K., Tinyanont, S., Yan, S.-Y., Wang, L., Wang, X.-F., Williams, S. C., Wyatt, S., Zhang, T.-M., de Boer, T., Chambers, K., Gao, H., Magnier, E., 2022, 2023, A\&A, 671 
    \color{blue}\href{https://ui.adsabs.harvard.edu/abs/2023A&A...671A.158P}{(ADS link)}\color{black}}\\

    \item{\textsc{A JWST Near- and Mid-infrared Nebular Spectrum of the Type Ia Supernova 2021aefx}\\ 
    Kwok, L. A., Jha, S. W., Temim, T., Fox, O. D., Larison, C., Camacho-Neves, Y., Brenner Newman, M. J., Pierel, J. D. R., Foley, R. J., Andrews, J. E., Badenes, C., Barna, B., \textbf{Bostroem, K. A.}, Deckers, M., Flörs, A., Garnavich, P., Graham, M. L., Graur, O., Hosseinzadeh, G., Howell, D. A., Hughes, J. P., Johansson, J., Kendrew, S., Kerzendorf, W. E., Maeda, K., Maguire, K., McCully, C., O'Brien, J. T., Rest, A., Sand, D. J., Shahbandeh, M., Strolger, L.-G., Szalai, T., Ashall, C., Baron, E., Burns, C. R., DerKacy, J. M., Evans, T. M., Fisher, A., Galbany, L., Hoeflich, P., Hsiao, E., de Jaeger, T., Karamehmetoglu, E., Krisciunas, K., Kumar, S., Lu, J., Maund, J., Mazzali, P. A., Medler, K., Morrell, N., Phillips, M. M., Shappee, B. J., Stritzinger, M., Suntzeff, N., Telesco, C., Tucker, M., Wang, L., 2023, ApJL, 944 
    \color{blue}\href{https://ui.adsabs.harvard.edu/abs/2023ApJ...944L...3K}{(ADS link)}\color{black}}\\
    
    \item{\textsc{Photometric and spectroscopic analysis of the Type II SN 2020jfo with a short plateau}\\ 
    Ailawadhi, B., Dastidar, R., Misra, K., Roy, R., Hiramatsu, D., Howell, D. A., Brink, T. G., Zheng, W., Galbany, L., Shahbandeh, M., Arcavi, I., Ashall, C., \textbf{Bostroem, K. A.}, Burke, J., Chapman, T., Dimple, Filippenko, A. V., Gangopadhyay, A., Ghosh, A., Hoffman, A. M., Hosseinzadeh, G., Jennings, C., Jha, V. K., Kumar, A., Karamehmetoglu, E., McCully, C., McGinness, E., Müller-Bravo, T. E., Murakami, Y. S., Pandey, S. B., Pellegrino, C., Piscarreta, L., Rho, J., Stritzinger, M., Sunseri, J., Van Dyk, S. D., Yadav, L., 2023, MNRAS, 519 
    \color{blue}\href{https://ui.adsabs.harvard.edu/abs/2023MNRAS.519..248A}{(ADS link)}\color{black}}\\

\item{\textsc{JWST Imaging of the Cartwheel Galaxy Reveals Dust Associated with SN 2021afdx}\\ 
Hosseinzadeh, G., Sand, D. J., Jencson, J. E., Andrews, J. E., Shivaei, I., \textbf{Bostroem, K. A.}, Valenti, S., Szalai, T., Burke, J., Howell, D. A., McCully, C., Newsome, M., Gonzalez, E. P., Pellegrino, C., Terreran, G., 2023, ApJL, 942 
\color{blue}\href{https://ui.adsabs.harvard.edu/abs/2023ApJ...942L..18H}{(ADS link)}\color{black}}\\
    
    \item{\textsc{A Multiwavelength View of the Rapidly Evolving SN 2018ivc: An Analog of SN IIb 1993J but Powered Primarily by Circumstellar Interaction}\\ 
    Maeda, K., Chandra, P., Moriya, T. J., Reguitti, A., Ryder, S., Matsuoka, T., Michiyama, T., Pignata, G., Hiramatsu, D., \textbf{Bostroem, K. A.}, Kundu, E., Kuncarayakti, H., Bersten, M. C., Pooley, D., Lee, S.-H., Patnaude, D., Rodríguez, Ó., Folatelli, G., 2023, ApJ, 942 
    \color{blue}\href{https://ui.adsabs.harvard.edu/abs/2023ApJ...942...17M}{(ADS link)}\color{black}}\\

    \item{\textsc{Optical studies of a bright Type Iax supernova SN 2020rea}\\ 
    Singh, M., Misra, K., Sahu, D. K., Ailawadhi, B., Dutta, A., Howell, D. A., Anupama, G. C., \textbf{Bostroem, K. A.}, Burke, J., Dastidar, R., Gangopadhyay, A., Hiramatsu, D., Im, H., McCully, C., Pellegrino, C., Srivastav, S., Teja, R. S., 2022, MNRAS.tmp, 
    \color{blue}\href{https://ui.adsabs.harvard.edu/abs/2022MNRAS.tmp.2872S}{(ADS link)}\color{black}}\\
    
    \item{\textsc{The Properties of Fast Yellow Pulsating Supergiants: FYPS Point the Way to Missing Red Supergiants}\\ 
Dorn-Wallenstein, T. Z., Levesque, E. M., Davenport, J. R. A., Neugent, K. F., Morris, B. M., \textbf{Bostroem, K. A.}, 2022, ApJ, 940 
\color{blue}\href{https://ui.adsabs.harvard.edu/abs/2022ApJ...940...27D}{(ADS link)}\color{black}}\\

    \item{\textsc{High-Cadence TESS and Ground-based Data of SN 2019esa, the Less Energetic Sibling of SN 2006gy}\\ 
    Andrews, J. E., Pearson, J., Lundquist, M. J., Sand, D. J., Jencson, J. E., \textbf{Bostroem, K. A}., Hosseinzadeh, G., Valenti, S., Smith, N., Amaro, R. C., Dong, Y., Janzen, D., Meza, N., Wyatt, S., Burke, J., Hiramatsu, D., Howell, D. A., McCully, C., Pellegrino, C., 2022, ApJ, 938 
    \color{blue}\href{https://ui.adsabs.harvard.edu/abs/2022ApJ...938...19A}{(ADS link)}\color{black}}\\
    
    \item{\textsc{The Diverse Properties of Type Icn Supernovae Point to Multiple Progenitor Channels}\\ 
    Pellegrino, C., Howell, D. A., Terreran, G., Arcavi, I., \textbf{Bostroem, K. A.}, Brown, P. J., Burke, J., Dong, Y., Gilkis, A., Hiramatsu, D., Hosseinzadeh, G., McCully, C., Modjaz, M., Newsome, M., Gonzalez, E. P., Pritchard, T. A., Sand, D. J., Valenti, S., Williamson, M., 2022, ApJ, 938 
    \color{blue}\href{https://ui.adsabs.harvard.edu/abs/2022ApJ...938...73P}{(ADS link)}\color{black}}\\
    
    \item{\textsc{Weak Mass Loss from the Red Supergiant Progenitor of the Type II SN 2021yja}\\ 
    Hosseinzadeh, G., Kilpatrick, C. D., Dong, Y., Sand, D. J., Andrews, J. E., \textbf{Bostroem, K. A.}, Janzen, D., Jencson, J. E., Lundquist, M., Meza Retamal, N. E., Pearson, J., Valenti, S., Wyatt, S., Burke, J., Hiramatsu, D., Howell, D. A., McCully, C., Newsome, M., Gonzalez, E. P., Pellegrino, C., Terreran, G., Auchettl, K., Davis, K. W., Foley, R. J., Miao, H.-Y., Pan, Y.-C., Rest, A., Siebert, M. R., Taggart, K., Tucker, B. E., Cyrus Leung, F. L., Swift, J. J., Yang, G., Anderson, J. P., Ashall, C., Benetti, S., Brown, P. J., Cartier, R., Chen, T.-W., Della Valle, M., Galbany, L., Gomez, S., Gromadzki, M., Haislip, J., Hsiao, E. Y., Inserra, C., Jha, S. W., Killestein, T. L., Kouprianov, V., Kozyreva, A., Müller-Bravo, T. E., Nicholl, M., Paraskeva, E., Reichart, D. E., Ryder, S., Shahbandeh, M., Shappee, B., Smith, N., Young, D. R., 2022, ApJ, 935 
    \color{blue}\href{https://ui.adsabs.harvard.edu/abs/2022ApJ...935...31H}{(ADS link)}\color{black}}\\
    
    \item{\textsc{SN 2016dsg: A Thermonuclear Explosion Involving a Thick Helium Shell}\\ 
    Dong, Y., Valenti, S., Polin, A., Boyle, A., Flörs, A., Vogl, C., Kerzendorf, W. E., Sand, D. J., Jha, S. W., Wyrzykowski, Ł., \textbf{Bostroem, K. A.}, Pearson, J., McCully, C., Andrews, J. E., Benetti, S., Blondin, S., Galbany, L., Gromadzki, M., Hosseinzadeh, G., Howell, D. A., Inserra, C., Jencson, J. E., Lundquist, M., Lyman, J. D., Magee, M., Maguire, K., Meza, N., Srivastav, S., Taubenberger, S., Terwel, J. H., Wyatt, S., Young, D. R., 2022, ApJ, 934 
    \color{blue}\href{https://ui.adsabs.harvard.edu/abs/2022ApJ...934..102D}{(ADS link)}\color{black}}\\
    
    \item{\textsc{Constraining the Progenitor System of the Type Ia Supernova 2021aefx}\\ 
    Hosseinzadeh, G., Sand, D. J., Lundqvist, P., Andrews, J. E., \textbf{Bostroem, K. A.}, Dong, Y., Janzen, D., Jencson, J. E., Lundquist, M., Meza Retamal, N. E., Pearson, J., Valenti, S., Wyatt, S., Burke, J., Howell, D. A., McCully, C., Newsome, M., Gonzalez, E. P., Pellegrino, C., Terreran, G., Kwok, L. A., Jha, S. W., Strader, J., Kundu, E., Ryder, S. D., Haislip, J., Kouprianov, V., Reichart, D. E., 2022, ApJL, 933 
    \color{blue}\href{https://ui.adsabs.harvard.edu/abs/2022ApJ...933L..45H}{(ADS link)}\color{black}}\\
    
    \item{\textsc{SN 2020acat: an energetic fast rising Type IIb supernova}\\ 
    Medler, K., Mazzali, P. A., Teffs, J., Ashall, C., Anderson, J. P., Arcavi, I., Benetti, S., \textbf{Bostroem, K. A.}, Burke, J., Cai, Y.-Z., Charalampopoulos, P., Elias-Rosa, N., Ergon, M., Galbany, L., Gromadzki, M., Hiramatsu, D., Howell, D. A., Inserra, C., Lundqvist, P., McCully, C., M\"{u}ller-Bravo, T., Newsome, M., Nicholl, M., Padilla Gonzalez, E., Paraskeva, E., Pastorello, A., Pellegrino, C., Pessi, P. J., Reguitti, A., Reynolds, T. M., Roy, R., Terreran, G., Tomasella, L., Young, D. R., 2022, MNRAS, 513 
    \color{blue}\href{https://ui.adsabs.harvard.edu/abs/2022MNRAS.513.5540M}{(ADS link)}\color{black}}\\
    


    

    
    \item{\textsc{The Lick AGN Monitoring Project 2016: Dynamical Modeling of Velocity-resolved H$\beta$ Lags in Luminous Seyfert Galaxies}\\ 
    Villafa\~{n}a, L., Williams, P. R., Treu, T., Brewer, B. J., Barth, A. J., U, V., Bennert, V. N., Alexander Vogler, H., Guo, H., Bentz, M. C., Canalizo, G., Filippenko, A. V., Gates, E., Hamann, F., Joner, M. D., Malkan, M. A., Woo, J.-H., Abolfathi, B., Abramson, L. E., Armen, S. F., Bae, H.-J., Bohn, T., Boizelle, B. D., \textbf{Bostroem, K. A.}, Brandel, A., Brink, T. G., Channa, S., Cooper, M. C., Cosens, M., Donohue, E., Fillingham, S. P., Gonz\'{a}lez-Buitrago, D., Halevi, G., Halle, A., Hood, C. E., Horne, K., Chuck Horst, J., de Kouchkovsky, M., Kuhn, B., Kumar, S., Leonard, D. C., Loveland, D., Manzano-King, C., McHardy, I., Michel, R., Olaes, M. K. B., Park, D., Park, S., Pei, L., Ross, T. W., Runco, J. N., S\'{a}nchez, J., Scott, B., Sexton, R. O., Shin, J., Shivvers, I., Spencer, C. L., Stahl, B. E., Stegman, S., Stomberg, I., Valenti, S., Walsh, J. L., Yuk, H., Zheng, W., 2022, ApJ, 930 
    \color{blue}\href{https://ui.adsabs.harvard.edu/abs/2022ApJ...930...52V}{(ADS link)}\color{black}}\\
    
    \item{\textsc{Nebular-phase spectra of Type Ia supernovae from the Las Cumbres Observatory Global Supernova Project}\\ 
    Graham, M. L., Kennedy, T. D., Kumar, S., Amaro, R. C., Sand, D. J., Jha, S. W., Galbany, L., Vinko, J., Wheeler, J. C., Hsiao, E. Y., \textbf{Bostroem, K. A.}, Burke, J., Hiramatsu, D., Hosseinzadeh, G., McCully, C., Howell, D. A., Diamond, T., Hoeflich, P., Wang, X., Li, W., 2022, MNRAS, 511 
    \color{blue}\href{https://ui.adsabs.harvard.edu/abs/2022MNRAS.511.3682G}{(ADS link)}\color{black}}\\
    
    \item{\textsc{The Candidate Progenitor Companion Star of the Type Ib/c SN 2013ge}\\ 
    Fox, O. D., Van Dyk, S. D., Williams, B. F., Drout, M., Zapartas, E., Smith, N., Milisavljevic, D., Andrews, J. E., \textbf{Bostroem, K. A.}, Filippenko, A. V., Gomez, S., Kelly, P. L., de Mink, S. E., Pierel, J., Rest, A., Ryder, S., Sravan, N., Strolger, L., Wang, Q., Weil, K. E., 2022, ApJL, 929 
    \color{blue}\href{https://ui.adsabs.harvard.edu/abs/2022ApJ...929L..15F}{(ADS link)}\color{black}}\\
    
    \item{\textsc{Infant-phase reddening by surface Fe-peak elements in a normal type Ia supernova}\\ 
    Ni, Y. Q., Moon, D.-S., Drout, M. R., Polin, A., Sand, D. J., Gonz\'{a}lez-Gait\'{a}n, S., Kim, S. C., Lee, Y., Park, H. S., Howell, D. A., Nugent, P. E., Piro, A. L., Brown, P. J., Galbany, L., Burke, J., Hiramatsu, D., Hosseinzadeh, G., Valenti, S., Afsariardchi, N., Andrews, J. E., Antoniadis, J., Arcavi, I., Beaton, R. L., \textbf{Bostroem, K. A.}, Carlberg, R. G., Cenko, S. B., Cha, S.-M., Dong, Y., Gal-Yam, A., Haislip, J., Holoien, T. W.-S., Johnson, S. D., Kouprianov, V., Lee, Y., Matzner, C. D., Morrell, N., McCully, C., Pignata, G., Reichart, D. E., Rich, J., Ryder, S. D., Smith, N., Wyatt, S., Yang, S., 2022, NatAs, 6 
    \color{blue}\href{https://ui.adsabs.harvard.edu/abs/2022NatAs...6..568N}{(ADS link)}\color{black}}\\
    
    \item{\textsc{SN 2020acat: A purr-fect example of a fast rising Type IIb Supernova}\\ 
    Medler, K., Mazzali, P. A., Teffs, J., Ashall, C., Anderson, J. P., Arcavi, I., Benetti, S., \textbf{Bostroem, K. A.}, Burke, J., Cai, Y.-Z., Charalampopoulos, P., Elias-Rosa, N., Ergon, M., Galbany, L., Gromadzki, M., Hiramatsu, D., Howell, D. A., Inserra, C., Lundqvist, P., McCully, C., Müller-Bravo, T., Newsome, M., Nicholl, M., Padilla Gonzalez, E., Paraskeva, E., Pastorello, A., Pellegrino, C., Pessi, P. J., Requitti, A., Reynolds, T. M., Roy, R., Terreran, G., Tomasella, L., Young, D. R., 2022, arXiv, 
    \color{blue}\href{https://ui.adsabs.harvard.edu/abs/2022arXiv220106991M}{(ADS link)}\color{black}}\\
    
    \item{\textsc{The Gravity Collective: A Search for the Electromagnetic Counterpart to the Neutron Star-Black Hole Merger GW190814}\\ 
    Kilpatrick, C. D., Coulter, D. A., Arcavi, I., Brink, T. G., Dimitriadis, G., Filippenko, A. V., Foley, R. J., Howell, D. A., Jones, D. O., Kasen, D., Makler, M., Piro, A. L., Rojas-Bravo, C., Sand, D. J., Swift, J. J., Tucker, D., Zheng, W., Allam, S. S., Annis, J. T., Antilen, J., Bachmann, T. G., Bloom, J. S., Bom, C. R., \textbf{Bostroem, K. A.}, Brout, D., Burke, J., Butler, R. E., Butner, M., Campillay, A., Clever, K. E., Conselice, C. J., Cooke, J., Dage, K. C., de Carvalho, R. R., de Jaeger, T., Desai, S., Garcia, A., Garcia-Bellido, J., Gill, M. S. S., Girish, N., Hallakoun, N., Herner, K., Hiramatsu, D., Holz, D. E., Huber, G., Kawash, A. M., McCully, C., Medallon, S. A., Metzger, B. D., Modak, S., Morgan, R., Mu\~{n}oz, R. R., Mu\~{n}oz-Elgueta, N., Murakami, Y. S., Felipe Olivares, E., Palmese, A., Patra, K. C., Pereira, M. E. S., Pessi, T. L., Pineda-Garcia, J., Quirola-V\'{a}squez, J., Ramirez-Ruiz, E., Rembold, S. B., Rest, A., Rodríguez, Ó., Santana-Silva, L., Sherman, N. F., Siebert, M. R., Smith, C., Smith, J. A., Soares-Santos, M., Stacey, H., Stahl, B. E., Strader, J., Strasburger, E., Sunseri, J., Tinyanont, S., Tucker, B. E., Ulloa, N., Valenti, S., Vasylyev, S. S., Wiesner, M. P., Zhang, K. D., 2021, ApJ, 923 
    \color{blue}\href{https://ui.adsabs.harvard.edu/abs/2021ApJ...923..258K}{(ADS link)}\color{black}}\\
    
    \item{\textsc{Circumstellar Medium Constraints on the Environment of Two Nearby Type Ia Supernovae: SN 2017cbv and SN 2020nlb}\\ 
    Sand, D. J., Sarbadhicary, S. K., Pellegrino, C., Misra, K., Dastidar, R., Brown, P. J., Itagaki, K., Valenti, S., Swift, J. J., Andrews, J. E., \textbf{Bostroem, K. A.}, Burke, J., Chomiuk, L., Dong, Y., Galbany, L., Graham, M. L., Hiramatsu, D., Howell, D. A., Hsiao, E. Y., Janzen, D., Jencson, J. E., Lundquist, M. J., McCully, C., Reichart, D., Smith, N., Wang, L., Wyatt, S., 2021, ApJ, 922 
    \color{blue}\href{https://ui.adsabs.harvard.edu/abs/2021ApJ...922...21S}{(ADS link)}\color{black}}\\
    
    \item{\textsc{A Bright Ultraviolet Excess in the Transitional 02es-like Type Ia Supernova 2019yvq}\\ 
    Burke, J., Howell, D. A., Sarbadhicary, S. K., Sand, D. J., Amaro, R. C., Hiramatsu, D., McCully, C., Pellegrino, C., Andrews, J. E., Brown, P. J., Itagaki, K., Shahbandeh, M., \textbf{Bostroem, K. A.}, Chomiuk, L., Hsiao, E. Y., Smith, N., Valenti, S., 2021, ApJ, 919 
    \color{blue}\href{https://ui.adsabs.harvard.edu/abs/2021ApJ...919..142B}{(ADS link)}\color{black}}\\
    
    \item{\textsc{AT 2019qyl in NGC 300: Internal Collisions in the Early Outflow from a Very Fast Nova in a Symbiotic Binary}\\ 
    Jencson, J. E., Andrews, J. E., Bond, H. E., Karambelkar, V., Sand, D. J., van Dyk, S. D., Blagorodnova, N., Boyer, M. L., Kasliwal, M. M., Lau, R. M., Mohamed, S., Williams, R., Whitelock, P. A., Amaro, R. C., \textbf{Bostroem, K. A.}, Dong, Y., Lundquist, M. J., Valenti, S., Wyatt, S. D., Burke, J., De, K., Jha, S. W., Johansson, J., Rojas-Bravo, C., Coulter, D. A., Foley, R. J., Gehrz, R. D., Haislip, J., Hiramatsu, D., Howell, D. A., Kilpatrick, C. D., Masci, F. J., McCully, C., Ngeow, C.-C., Pan, Y.-C., Pellegrino, C., Piro, A. L., Kouprianov, V., Reichart, D. E., Rest, A., Rest, S., Smith, N., 2021, ApJ, 920 
    \color{blue}\href{https://ui.adsabs.harvard.edu/abs/2021ApJ...920..127J}{(ADS link)}\color{black}}\\
    
    \item{\textsc{The Blue Supergiant Progenitor of the Supernova Imposter AT 2019krl}\\ 
    Andrews, J. E., Jencson, J. E., Van Dyk, S. D., Smith, N., Neustadt, J. M. M., Sand, D. J., Kreckel, K., Kochanek, C. S., Valenti, S., Strader, J., Bersten, M. C., Blanc, G. A., \textbf{Bostroem, K. A.}, Brink, T. G., Emsellem, E., Filippenko, A. V., Folatelli, G., Kasliwal, M. M., Masci, F. J., McElroy, R., Milisavljevic, D., Santoro, F., Szalai, T., 2021, ApJ, 917 
    \color{blue}\href{https://ui.adsabs.harvard.edu/abs/2021ApJ...917...63A}{(ADS link)}\color{black}}\\
    
    \item{\textsc{The Exotic Type Ic Broad-lined Supernova SN 2018gep: Blurring the Line between Supernovae and Fast Optical Transients}\\ 
    Pritchard, T. A., Bensch, K., Modjaz, M., Williamson, M., Thöne, C. C., Vink\'{o}, J., Bianco, F. B., \textbf{Bostroem, K. A.}, Burke, J., Garc\'{i}a-Benito, R., Galbany, L., Hiramatsu, D., Howell, D. A., Izzo, L., Kann, D. A., McCully, C., Pellegrino, C., de Ugarte Postigo, A., Valenti, S., Wang, X., Wheeler, J. C., Xiang, D., Sárneczky, K., Bódi, A., Cseh, B., Tarczay-Neh\'{e}z, D., Kriskovics, L., Ordasi, A., Pál, A., Szakáts, R., Vida, K., 2021, ApJ, 915 
    \color{blue}\href{https://ui.adsabs.harvard.edu/abs/2021ApJ...915..121P}{(ADS link)}\color{black}}\\
    
    \item{\textsc{Enormous explosion energy of Type IIP SN 2017gmr with bipolar ${}^{56}$Ni ejecta}\\ 
    Utrobin, V. P., Chugai, N. N., Andrews, J. E., Smith, N., Jencson, J., Howell, D. A., Burke, J., Hiramatsu, D., McCully, C., \textbf{Bostroem, K. A.}, 2021, MNRAS, 505 
    \color{blue}\href{https://ui.adsabs.harvard.edu/abs/2021MNRAS.505..116U}{(ADS link)}\color{black}}\\
    
    \item{\textsc{Strong Near-infrared Carbon Absorption in the Transitional Type Ia SN 2015bp}\\ 
    Wyatt, S. D., Sand, D. J., Hsiao, E. Y., Burns, C. R., Valenti, S., \textbf{Bostroem, K. A.}, Lundquist, M., Galbany, L., Lu, J., Ashall, C., Diamond, T. R., Filippenko, A. V., Graham, M. L., Hoeflich, P., Kirshner, R. P., Krisciunas, K., Marion, G. H., Morrell, N., Persson, S. E., Phillips, M. M., Stritzinger, M. D., Suntzeff, N. B., Taddia, F., 2021, ApJ, 914 
    \color{blue}\href{https://ui.adsabs.harvard.edu/abs/2021ApJ...914...57W}{(ADS link)}\color{black}}\\
    
    \item{\textsc{The electron-capture origin of supernova 2018zd}\\ 
    Hiramatsu, D., Howell, D. A., Van Dyk, S. D., Goldberg, J. A., Maeda, K., Moriya, T. J., Tominaga, N., Nomoto, K., Hosseinzadeh, G., Arcavi, I., McCully, C., Burke, J., \textbf{Bostroem, K. A.}, Valenti, S., Dong, Y., Brown, P. J., Andrews, J. E., Bilinski, C., Williams, G. G., Smith, P. S., Smith, N., Sand, D. J., Anand, G. S., Xu, C., Filippenko, A. V., Bersten, M. C., Folatelli, G., Kelly, P. L., Noguchi, T., Itagaki, K., 2021, NatAs, 5 
    \color{blue}\href{https://ui.adsabs.harvard.edu/abs/2021NatAs...5..903H}{(ADS link)}\color{black}}\\
    
    \item{\textsc{The Early Discovery of SN 2017ahn: Signatures of Persistent Interaction in a Fast-declining Type II Supernova}\\ 
    Tartaglia, L., Sand, D. J., Groh, J. H., Valenti, S., Wyatt, S. D., \textbf{Bostroem, K. A.}, Brown, P. J., Yang, S., Burke, J., Chen, T.-W., Davis, S., F\"{o}rster, F., Galbany, L., Haislip, J., Hiramatsu, D., Hosseinzadeh, G., Howell, D. A., Hsiao, E. Y., Jha, S. W., Kouprianov, V., Kuncarayakti, H., Lyman, J. D., McCully, C., Phillips, M. M., Rau, A., Reichart, D. E., Shahbandeh, M., Strader, J., 2021, ApJ, 907 
    \color{blue}\href{https://ui.adsabs.harvard.edu/abs/2021ApJ...907...52T}{(ADS link)}\color{black}}\\
    
    \item{\textsc{Supernova 2018cuf: A Type IIP Supernova with a Slow Fall from Plateau}\\ 
    Dong, Y., Valenti, S., \textbf{Bostroem, K. A.}, Sand, D. J., Andrews, J. E., Galbany, L., Jha, S. W., Eweis, Y., Kwok, L., Hsiao, E. Y., Davis, S., Brown, P. J., Kuncarayakti, H., Maeda, K., Rho, J., Amaro, R. C., Anderson, J. P., Arcavi, I., Burke, J., Dastidar, R., Folatelli, G., Haislip, J., Hiramatsu, D., Hosseinzadeh, G., Howell, D. A., Jencson, J., Kouprianov, V., Lundquist, M., Lyman, J. D., McCully, C., Misra, K., Reichart, D. E., Sánchez, S. F., Smith, N., Wang, X., Wang, L., Wyatt, S., 2021, ApJ, 906 
    \color{blue}\href{https://ui.adsabs.harvard.edu/abs/2021ApJ...906...56D}{(ADS link)}\color{black}}\\
    
    \item{\textsc{SN 2018gjx reveals that some SNe Ibn are SNe IIb exploding in dense circumstellar material}\\ 
    Prentice, S. J., Maguire, K., Boian, I., Groh, J., Anderson, J., Barbarino, C., \textbf{Bostroem, K. A.}, Burke, J., Clark, P., Dong, Y., Fraser, M., Galbany, L., Gromadzki, M., Guti\'{e}rrez, C. P., Howell, D. A., Hiramatsu, D., Inserra, C., James, P. A., Kankare, E., Kuncarayakti, H., Mazzali, P. A., McCully, C., Müller-Bravo, T. E., Nichol, M., Pellegrino, C., Smartt, S. J., Sollerman, J., Tartaglia, L., Valenti, S., Young, D. R., 2020, MNRAS, 499 
    \color{blue}\href{https://ui.adsabs.harvard.edu/abs/2020MNRAS.499.1450P}{(ADS link)}\color{black}}\\
    
    \item{\textsc{SN 2017ivv: two years of evolution of a transitional Type II supernova}\\ 
    Gutiérrez, C. P., Pastorello, A., Jerkstrand, A., Galbany, L., Sullivan, M., Anderson, J. P., Taubenberger, S., Kuncarayakti, H., Gonz\'{a}lez-Gait\'{a}n, S., Wiseman, P., Inserra, C., Fraser, M., Maguire, K., Smartt, S., M\"{u}ller-Bravo, T. E., Arcavi, I., Benetti, S., Bersier, D., Bose, S., \textbf{Bostroem, K. A.}, Burke, J., Chen, P., Chen, T.-W., Della Valle, M., Dong, S., Gal-Yam, A., Gromadzki, M., Hiramatsu, D., Holoien, T. W.-S., Hosseinzadeh, G., Howell, D. A., Kankare, E., Kochanek, C. S., McCully, C., Nicholl, M., Pignata, G., Prieto, J. L., Shappee, B., Taggart, K., Tomasella, L., Valenti, S., Young, D. R., 2020, MNRAS, 499 
    \color{blue}\href{https://ui.adsabs.harvard.edu/abs/2020MNRAS.499..974G}{(ADS link)}\color{black}}\\
    
    \item{\textsc{The slow demise of the long-lived SN 2005ip}\\ 
    Fox, O. D., Fransson, C., Smith, N., Andrews, J., \textbf{Azalee Bostroem, K.}, Brink, T. G., Bradley Cenko, S., Clayton, G. C., Filippenko, A. V., Fong, W.-. fai ., Gallagher, J. S., Kelly, P. L., Kilpatrick, C. D., Mauerhan, J. C., Miller, A. M., Montiel, E., Stritzinger, M. D., Szalai, T., Van Dyk, S. D., 2020, MNRAS, 498 
    \color{blue}\href{https://ui.adsabs.harvard.edu/abs/2020MNRAS.498..517F}{(ADS link)}\color{black}}\\
 
    
    \item{\textsc{SN 2017gmr: An Energetic Type II-P Supernova with Asymmetries}\\ 
    Andrews, J. E., Sand, D. J., Valenti, S., Smith, N., Dastidar, R., Sahu, D. K., Misra, K., Singh, A., Hiramatsu, D., Brown, P. J., Hosseinzadeh, G., Wyatt, S., Vinko, J., Anupama, G. C., Arcavi, I., Ashall, C., Benetti, S., Berton, M., \textbf{Bostroem, K. A.}, Bulla, M., Burke, J., Chen, S., Chomiuk, L., Cikota, A., Congiu, E., Cseh, B., Davis, S., Elias-Rosa, N., Faran, T., Fraser, M., Galbany, L., Gall, C., Gal-Yam, A., Gangopadhyay, A., Gromadzki, M., Haislip, J., Howell, D. A., Hsiao, E. Y., Inserra, C., Kankare, E., Kuncarayakti, H., Kouprianov, V., Kumar, B., Li, X., Lin, H., Maguire, K., Mazzali, P., McCully, C., Milne, P., Mo, J., Morrell, N., Nicholl, M., Ochner, P., Olivares, F., Pastorello, A., Patat, F., Phillips, M., Pignata, G., Prentice, S., Reguitti, A., Reichart, D. E., Rodríguez, Ó., Rui, L., Sanwal, P., Sárneczky, K., Shahbandeh, M., Singh, M., Smartt, S., Strader, J., Stritzinger, M. D., Szakáts, R., Tartaglia, L., Wang, H., Wang, L., Wang, X., Wheeler, J. C., Xiang, D., Yaron, O., Young, D. R., Zhang, J., 2019, ApJ, 885 
    \color{blue}\href{https://ui.adsabs.harvard.edu/abs/2019ApJ...885...43A}{(ADS link)}\color{black}}\\
   
    
    \item{\textsc{The Type II-P Supernova 2017eaw: From Explosion to the Nebular Phase}\\ 
    Szalai, T., Vinkó, J., Könyves-T\'{o}th, R., Nagy, A. P., \textbf{Bostroem, K. A.}, S\'{a}rneczky, K., Brown, P. J., Pejcha, O., Bódi, A., Cseh, B., Csörnyei, G., Dencs, Z., Hanyecz, O., Ign\'{a}cz, B., Kalup, C., Kriskovics, L., Ordasi, A., Pál, A., Seli, B., S\'{o}dor, \'{A}., Szak\'{a}ts, R., Vida, K., Zsidi, G., Konkoly Team, Arcavi, I., Ashall, C., Burke, J., Galbany, L., Hiramatsu, D., Hosseinzadeh, G., Hsiao, E. Y., Howell, D. A., McCully, C., Moran, S., Rho, J., Sand, D. J., Shahbandeh, M., Valenti, S., Wang, X., Wheeler, J. C., Supernova Project, G., 2019, ApJ, 876 
    \color{blue}\href{https://ui.adsabs.harvard.edu/abs/2019ApJ...876...19S}{(ADS link)}\color{black}}\\
    
    \item{\textsc{pwv\_kpno: A Python Package for Modeling the Atmospheric Transmission Function due to Precipitable Water Vapor}\\ 
    Perrefort, D., Wood-Vasey, W. M., \textbf{Bostroem, K. A.}, Gilmore, K., Joyce, R., Matheson, T., Corson, C., 2019, PASP, 131 
    \color{blue}\href{https://ui.adsabs.harvard.edu/abs/2019PASP..131b5002P}{(ADS link)}\color{black}}\\
    
    \item{\textsc{The Astropy Project: Building an Open-science Project and Status of the v2.0 Core Package}\\ 
    Astropy Collaboration, Price-Whelan, A. M., Sip\H{o}cz, B. M., Günther, H. M., Lim, P. L., Crawford, S. M., Conseil, S., Shupe, D. L., Craig, M. W., Dencheva, N., Ginsburg, A., VanderPlas, J. T., Bradley, L. D., P\'{e}rez-Su\'{a}rez, D., de Val-Borro, M., Aldcroft, T. L., Cruz, K. L., Robitaille, T. P., Tollerud, E. J., Ardelean, C., Babej, T., Bach, Y. P., Bachetti, M., Bakanov, A. V., Bamford, S. P., Barentsen, G., Barmby, P., Baumbach, A., Berry, K. L., Biscani, F., Boquien, M., \textbf{Bostroem, K. A.}, [+108 authors], Astropy Contributors, 2018, AJ, 156 
    \color{blue}\href{https://ui.adsabs.harvard.edu/abs/2018AJ....156..123A}{(ADS link)}\color{black}}\\
    
    \item{\textsc{Ultraviolet Detection of the Binary Companion to the Type IIb SN 2001ig}\\ 
    Ryder, S. D., Van Dyk, S. D., Fox, O. D., Zapartas, E., de Mink, S. E., Smith, N., Brunsden, E., \textbf{Bostroem, K. A.}, Filippenko, A. V., Shivvers, I., Zheng, W., 2018, ApJ, 856 
    \color{blue}\href{https://ui.adsabs.harvard.edu/abs/2018ApJ...856...83R}{(ADS link)}\color{black}}\\
    
    \item{\textsc{Early Blue Excess from the Type Ia Supernova 2017cbv and Implications for Its Progenitor}\\ 
    Hosseinzadeh, G., Sand, D. J., Valenti, S., Brown, P., Howell, D. A., McCully, C., Kasen, D., Arcavi, I., \textbf{Bostroem, K. A.}, Tartaglia, L., Hsiao, E. Y., Davis, S., Shahbandeh, M., Stritzinger, M. D., 2017, ApJL, 845 
    \color{blue}\href{https://ui.adsabs.harvard.edu/abs/2017ApJ...845L..11H}{(ADS link)}\color{black}}\\
    
    \item{\textsc{Predicting the Presence of Companions for Stripped-envelope Supernovae: The Case of the Broad-lined Type Ic SN 2002ap}\\ 
    Zapartas, E., de Mink, S. E., Van Dyk, S. D., Fox, O. D., Smith, N., \textbf{Bostroem, K. A.}, de Koter, A., Filippenko, A. V., Izzard, R. G., Kelly, P. L., Neijssel, C. J., Renzo, M., Ryder, S., 2017, ApJ, 842 
    \color{blue}\href{https://ui.adsabs.harvard.edu/abs/2017ApJ...842..125Z}{(ADS link)}\color{black}}\\
    
    \item{\textsc{The R136 star cluster dissected with Hubble Space Telescope/STIS. I. Far-ultraviolet spectroscopic census and the origin of He II $\lambda$1640 in young star clusters}\\ 
    Crowther, P. A., Caballero-Nieves, S. M., \textbf{Bostroem, K. A.}, Ma\'{i}z Apell\'{a}niz, J., Schneider, F. R. N., Walborn, N. R., Angus, C. R., Brott, I., Bonanos, A., de Koter, A., de Mink, S. E., Evans, C. J., Gr\"{a}fener, G., Herrero, A., Howarth, I. D., Langer, N., Lennon, D. J., Puls, J., Sana, H., Vink, J. S., 2016, MNRAS, 458 
    \color{blue}\href{https://ui.adsabs.harvard.edu/abs/2016MNRAS.458..624C}{(ADS link)}\color{black}}\\
    
    \item{\textsc{What powers the 3000-day light curve of SN 2006gy?}\\ 
    Fox, O. D., Smith, N., Ammons, S. M., Andrews, J., \textbf{Bostroem, K. A.}, Cenko, S. B., Clayton, G. C., Dwek, E., Filippenko, A. V., Gallagher, J. S., Kelly, P. L., Mauerhan, J. C., Miller, A. A., Van Dyk, S. D., 2015, MNRAS, 454 
    \color{blue}\href{https://ui.adsabs.harvard.edu/abs/2015MNRAS.454.4366F}{(ADS link)}\color{black}}\\
    
    \item{\textsc{Type Ia Supernova Rate Measurements to Redshift 2.5 from CANDELS: Searching for Prompt Explosions in the Early Universe}\\ 
    Rodney, S. A., Riess, A. G., Strolger, L.-G., Dahlen, T., Graur, O., Casertano, S., Dickinson, M. E., Ferguson, H. C., Garnavich, P., Hayden, B., Jha, S. W., Jones, D. O., Kirshner, R. P., Koekemoer, A. M., McCully, C., Mobasher, B., Patel, B., Weiner, B. J., Cenko, S. B., Clubb, K. I., Cooper, M., Filippenko, A. V., Frederiksen, T. F., Hjorth, J., Leibundgut, B., Matheson, T., Nayyeri, H., Penner, K., Trump, J., Silverman, J. M., U, V., \textbf{Azalee Bostroem, K.}, Challis, P., Rajan, A., Wolff, S., Faber, S. M., Grogin, N. A., Kocevski, D., 2014, AJ, 148 
    \color{blue}\href{https://ui.adsabs.harvard.edu/abs/2014AJ....148...13R}{(ADS link)}\color{black}}\\
    
    \item{\textsc{Uncovering the Putative B-star Binary Companion of the SN 1993J Progenitor}\\ 
    Fox, O. D., \textbf{Azalee Bostroem, K.}, Van Dyk, S. D., Filippenko, A. V., Fransson, C., Matheson, T., Cenko, S. B., Chandra, P., Dwarkadas, V., Li, W., Parker, A. H., Smith, N., 2014, ApJ, 790 
    \color{blue}\href{https://ui.adsabs.harvard.edu/abs/2014ApJ...790...17F}{(ADS link)}\color{black}}\\
    
    \item{\textsc{Astropy: A community Python package for astronomy}\\ 
    Astropy Collaboration, Robitaille, T. P., Tollerud, E. J., Greenfield, P., Droettboom, M., Bray, E., Aldcroft, T., Davis, M., Ginsburg, A., Price-Whelan, A. M., Kerzendorf, W. E., Conley, A., Crighton, N., Barbary, K., Muna, D., Ferguson, H., Grollier, F., Parikh, M. M., Nair, P. H., Unther, H. M., Deil, C., Woillez, J., Conseil, S., Kramer, R., Turner, J. E. H., Singer, L., Fox, R., Weaver, B. A., Zabalza, V., Edwards, Z. I., \textbf{Azalee Bostroem, K.}, Burke, D. J., Casey, A. R., Crawford, S. M., Dencheva, N., Ely, J., Jenness, T., Labrie, K., Lim, P. L., Pierfederici, F., Pontzen, A., Ptak, A., Refsdal, B., Servillat, M., Streicher, O., 2013, A\&A, 558 
    \color{blue}\href{https://ui.adsabs.harvard.edu/abs/2013A&A...558A..33A}{(ADS link)}\color{black}}\\
\end{revnumerate}































%-------------------------------------------------------------
%----------TECHNICAL REPORTS-------------------
%-------------------------------------------------------------
\vspace{-0.1in}   
\sectiontitle{Technical\\Reports\\ \color{blue}\href{https://ui.adsabs.harvard.edu/public-libraries/QCbOxuv6T6GmSVRgbdxYiw}{(ADS link)}\color{black}}
\begin{revnumerate}[24]
\item{\textsc{Summary of the COS Cycle 22 Calibration Program}\\
Sonnentrucker; P.; Becker, G.; {\bf Bostroem, K. A.}; Debes, J.H.; Ely, J.; Fox, A.; Lockwood, S.; Oliveira, C.; Penton, S.; Proffitt, C.; Roman-Duval, J.; Sahnow, D.; Sana, H.; Taylor, J.; Welty, A. D.; Wheeler, T., 2016, Tech. rep 
\color{blue}\href{https://ui.adsabs.harvard.edu/abs/2016cos..rept....3S/abstract}{(ADS link)}\color{black}}\\ %ISR
%\\
\item{\textsc{Summary of the COS Cycle 21 Calibration Program}\\
Sana, H.; Roman-Duval, J.; Ely, J.; {\bf Bostroem, K. A.}; Lockwood, S.; Oliveira, C.; Penton, S.; Proffitt, C.; Sahnow, D.; Sonnentrucker, P.; Welty, A. D.; Wheeler, T., 2015, Tech. rep 
\color{blue}\href{https://ui.adsabs.harvard.edu/abs/2015cos..rept....6S/abstract}{(ADS link)}\color{black}}\\ %ISR

\item{\textsc{Changes to the COS Extraction Algorithm for Lifetime Position 3}\\
Proffitt, C. R.; {\bf Bostroem, K. A.}; Ely, J.; Foster, D.; Hernandez, S.; Hodge, P.; Jedrzejewski, R. I.; Lockwood, S. A.; Massa, D.; Peeples, M. S.; Oliveira, C. M.; Penton, S. V.; Plesha, R.; Roman-Duval, J.; Sana, H.; Sahnow, D. J.; Sonnentrucker, P.; Taylor, J. M.,  2015, Tech. rep 
\color{blue}\href{https://ui.adsabs.harvard.edu/abs/2015cos..rept....3P/abstract}{(ADS link)}\color{black}}\\ %ISR

\item{\textsc{Summary of the COS Cycle 20 Calibration Program}\\
Roman-Duval, J.; Aloisi, A.; {\bf Bostroem, K. A.}; Ely, J.; Holland, S.; Lockwood, S.; Oliveira, C.; Penton, S.; Proffitt, C.; Sahnow, D.; Sonnentrucker, P.; Welty, A. D.; Wheeler, T., 2015, Tech. rep 
\color{blue}\href{https://ui.adsabs.harvard.edu/abs/2015cos..rept....2R/abstract}{(ADS link)}\color{black}}\\ %ISR

\item{\textsc{The Time-Dependent Sensitivity of the MAMA and CCD Long-Slit Gratings}\\
Holland, S. T.; Aloisi, A.; {\bf Bostroem, K. A.}; Oliveria, C.; Proffitt, C.,  2014, Tech. rep 
\color{blue}\href{https://ui.adsabs.harvard.edu/abs/2014stis.rept....2H/abstract}{(ADS link)}\color{black}}\\ %ISR

\item{\textsc{Summary of the STIS Cycle 19 Calibration Program}\\
Roman-Duval, J.; Ely, J.; Aloisi, A.; Oliveira, C.; Proffitt, C.; Hernandez, S.; Mason, E.; Sonnetrucker, P.; Wolfe, M.; Long, C.; DiFelice, A.; \textbf{Bostroem, K. A.}; Holland, S.; Lockwood, S.; Cox, C.; Wheeler, T., 2014, Tech. rep 
\color{blue}\href{https://ui.adsabs.harvard.edu/abs/2014stis.rept....1R/abstract}{(ADS link)}\color{black}}\\ %ISR

\item{\textsc{Summary of the Cycle 19 COS Calibration Program}\\
Roman-Duval, J.; Ely, J.; Oliveira, C.; Proffitt, C.; Aloisi, A.; \textbf{Bostroem, K. A.}; Cox, C.; Lockwood, S.; Mason, E.; Massa, D.; Osten, R.; Penton, S.; Sahnow, D.; Sonnetrucker, P.; Wheeler, T., 2014, Tech. rep 
\color{blue}\href{https://ui.adsabs.harvard.edu/abs/2014cos..rept....1R/abstract}{(ADS link)}\color{black}}\\ %ISR

\item{\textsc{Updated Absolute Flux Calibration of the COS FUV Modes}\\
Massa, D.; Ely, J.; Osten, R.; Penton, S.; Aloisi, A.; \textbf{Bostroem, K. A.}; Roman-Duval, J.; Proffitt, C. 2014, Tech. rep. 
\color{blue}\href{https://ui.adsabs.harvard.edu/abs/2014cos..rept....9M/abstract}{(ADS link)}\color{black}}\\ %ISR

\item{\textsc{Summary of STIS Cycle 18 Calibration Program}\\
Kriss, G. A.; Wolfe, M. A.; Aloisi, A.; \textbf{Bostroem, K. A.}; Cox, C.; Dixon, V.; Ely, J.; Long, C.; Mason, E.; Massa, D.; Osten, R.; Proffitt, C.; Roman-Duval, J.; Sonnentrucker, P.; Wheeler, T.; Zheng, W., 2013, Tech. rep. 
\color{blue}\href{https://ui.adsabs.harvard.edu/abs/2013stis.rept....3K/abstract}{(ADS link)}\color{black}}\\ %ISR

\item{\textsc{Summary of Results from the First Move to a New COS FUV Lifetime Position}\\
Osten, R. A.; Aloisi, A.; \textbf{Bostroem, K. A.};  Debes, J.; Ely, J.; Hodge, P. E.; Kriss, G.; Massa, D.; Oliveira, C.; Osten, R.; Osterman, S. N.; Penton, S. V.; Proffitt, C.; Roman-Duval, J.; Sonnentrucker, P., 2013, Tech. rep. 
\color{blue}\href{https://ui.adsabs.harvard.edu/abs/2013cos..rept...16O/abstract}{(ADS link)}\color{black}}\\ %ISR

\item{\textsc{Characterization, modeling, and management of the COS FUV detector lifetime}\\
Sahnow, D. J.; Aloisi, A.; \textbf{Bostroem, K. A.}; Debes, J.; Ely, J.; Hodge, P. E.; Kriss, G.; Massa, D.; Oliveira, C.; Osten, R.; Osterman, S. N.; Penton, S. V.; Proffitt, C.; Roman-Duval, J.; Sonnentrucker, P., 2013, Proc. of the SPIE, 8859 
\color{blue}\href{https://ui.adsabs.harvard.edu/abs/2013SPIE.8859E..0SS/abstract}{(ADS link)}\color{black}}\\

\item{\textsc{Summary of the COS Cycle 18 Calibration Program}\\
Kriss, G. A.; Wolfe, M.; Aloisi, A.; \textbf{Bostroem, K. A.}; Cox, C.; Ely, J.; Long, C.; Massa, D.; Oliveria, C.; Osten, R.; Proffitt, C.; Sahnow, D.; Wheeler, T.; Zheng, W., 2013, Tech. rep. 
\color{blue}\href{https://ui.adsabs.harvard.edu/abs/2013cos..rept....4K/abstract}{(ADS link)}\color{black}}\\ %ISR

\item{\textsc{Second COS FUV Lifetime Position Results from the Focus Sweep Enabling Program}\\
Oliveira, C.; \textbf{Bostroem, K. A.}; Osterman, S., FENA3 (12796), 2013, Tech. rep. 
\color{blue}\href{https://ui.adsabs.harvard.edu/abs/2013cos..rept....1O/abstract}{(ADS link)}\color{black}}\\ %ISR

\item{\textsc{Summary of the COS Cycle 17 Calibration Program}\\
Osten, R. A.; Wolfe, M.; Ake, T.; Aloisi, A.; \textbf{Bostroem, K. A.}; Dixon, W. V. D.; Ghavamian, P.; Goudfrooij, P.; Ely, J.; Massa, D.; Niemi, S.; Oliveira, C.; Osterman, S.; Pascucci, I.; Penton, S.; Proffitt, C.; Sahnow, D.; Wheeler, T.; York, B.; Zheng, W., 2012, Tech. rep. 
\color{blue}\href{https://ui.adsabs.harvard.edu/abs/2012cos..rept....2O/abstract}{(ADS link)}\color{black}}\\ %ISR

\item{\textsc{Post-SM4 Sensitivity Calibration of the STIS Echelle Modes}\\
{\bf Bostroem, K. A.}; Aloisi, A.; Bohlin, R.; Hodge, P.; Proffitt, C., 2012, Tech. rep 
\color{blue}\href{https://ui.adsabs.harvard.edu/abs/2012stis.rept....1B/abstract}{(ADS link)}\color{black}}\\%Post-SM4 Sensitivity Calibration of the STIS Echelle Modes

\item{\textsc{The COS FUV channel: on-orbit performance trends and early characterization of a new detector lifetime position}\\
Sahnow, D. J.; Aloisi, A.; \textbf{Bostroem, K. A.}; Debes, J.; Duval, J.; Ely, J.; Hodge, P. E.; Kriss, G.; Lindsay, K.; Massa, D.; Oliveira, C.; Osten, R.; Osterman, S. N.; Penton, S. V.; Proffitt, C.; Sonnentrucker, P.; York, B., 2012, Proc. of the SPIE, 8443 
\color{blue}\href{https://ui.adsabs.harvard.edu/abs/2012SPIE.8443E..4CS/abstract}{(ADS link)}\color{black}}\\

\item{\textsc{Summary of the STIS Cycle 17 Calibration Program}\\
Wolfe, M. A.; Osten, R. A.; Hernandez, S.; Aloisi, A.; Bohlin, R.; \textbf{Bostroem, K. A.}; Diaz, R.; Dixon, V.; Ely, J.; Hodge, P.; Lennon, D.; Mason, E.; Niemi, S.; Pascuucci, I.; Proffitt, C.; Wheeler, T.; Zheng, W., 2012, Tech. rep. 
\color{blue}\href{https://ui.adsabs.harvard.edu/abs/2012stis.rept....3W/abstract}{(ADS link)}\color{black}}\\%ISR

\item{\textsc{Gain sag in the FUV detector of the Cosmic Origins Spectrograph}\\
Sahnow, D. J.; Oliveira, C.; Aloisi, A.; Hodge, P. E.; Massa, D.; Osten, R.; Proffitt, C.; \textbf{Bostroem, K. A.}; McPhate, J. B.; B\'{e}land, S.; Osterman, S. N.; Penton, S. V., 2011,  Proc. of the SPIE, 8145 
\color{blue}\href{https://ui.adsabs.harvard.edu/abs/2011SPIE.8145E..0QS/abstract}{(ADS link)}\color{black}}\\ %ISR

\item{\textsc{Updated Results from the COS Spectroscopic Sensitivity Monitoring Program}\\
Osten, R. A.; Massa, D.; \textbf{Bostroem, K. A.}; Aloisi, A.; Proffitt, C., 2011, Tech. rep. 
\color{blue}\href{https://ui.adsabs.harvard.edu/abs/2011cos..rept....2O/abstract}{(ADS link)}\color{black}}\\ %ISR

\item{\textsc{STIS Data Handbook v. 6.0}\\
{\bf Bostroem, K. A.} \& Proffitt, C. 2011
\color{blue}\href{https://ui.adsabs.harvard.edu/abs/2011stis.book.....B/abstract}{(ADS link)}\color{black}}\\

\item{\textsc{Post - SM4 Flux Calibration of the STIS Echelle Modes}\\
\textbf{Bostroem, K. A.}; Aloisi, A.; Bohlin, R. C.; Proffitt, C. R.; Osten, R. A.; Lennon, D., 2010, HST Calibration Workshop Proceedings
\color{blue}\href{https://ui.adsabs.harvard.edu/abs/2010hstc.workE..51B/abstract}{(ADS link)}\color{black}}\\%Post - SM4 Flux Calibration of the STIS Echelle Modes

\item{\textsc{Trend of Dark Rates of the COS and STIS NUV MAMA Detectors}\\
Zheng, W.; Proffitt, C. R.; Sahnow, D.; Ake, T. B.; Keyes, C.; Goudfrooij, P.; Hodge, P.; Oliveira, C.; \textbf{Bostroem, K. A.}; Long, C.; Aloisi, A., 2010, HST Calibration Workshop Proceedings 
\color{blue}\href{https://ui.adsabs.harvard.edu/abs/2010hstc.workE..47Z/abstract}{(ADS link)}\color{black}}\\

\item{\textsc{Performance of the Space Telescope Imaging Spectrograph after SM4}\\
Proffitt, C. R.; Aloisi, A.; Bohlin, C.; \textbf{Bostroem, K. A.}; Cox, C. R.; Diaz, R. I.; Dixon, W. V.; Goudfrooij, P.; Hodge, P.; Kaiser, M. E.; Lallo, M. D.; Lennon, D.; Niemi, S.; Osten, R. A.; Pascucci, I.; Smith, E.; Wolfe, M. A.; York, B.; Zheng, W.; Gull, T. R.; Lindler, D. J.; Woodgate, B. E., 2010, HST Calibration Workshop Proceedings
\color{blue}\href{https://ui.adsabs.harvard.edu/abs/2010hstc.workE...6P/abstract}{(ADS link)}\color{black}}\\
%\\
\item{\textsc{The On-Orbit Performance of the Cosmic Origins Spectrograph}\\
Aloisi, A.; Ake, T.; \textbf{Bostroem, K. A.}; Bohlin, R.; Cox, C.; Diaz, R.; Dixon, V.; Ghavamian, P.; Goudfrooij, P.; Hartig, G.; Hodge, P.; Keyes, C.; Kriss, G.; Lallo, M.; Lennon, D.; Massa, D.; Niemi, S.; Oliveira, C.; Osten, R.; Proffitt, C. R.; Sahnow, D.; Smith, E.; Wheeler, T.; Wolfe, M.; York, B.; Zheng, W.; Green, J.; Froning, C.; Beland, S.; Burgh, E.; France, K.; Osterman, S.; Penton, S.; McPhate, J.; Delker, T., 2010, HST Calibration Workshop Proceedings 
\color{blue}\href{https://ui.adsabs.harvard.edu/abs/2010hstc.workE...3A/abstract}{(ADS link)}\color{black}}\\
\end{revnumerate}
%16 technical reports with {\bf Bostroem} as co-author; see \color{blue}\url{http://bit.ly/2iUeoax }\color{black}\hspace*{0pt} for details

%-------------------------------------------------------------
%-----------------------POSTERS-----------------------
%-------------------------------------------------------------
\vspace{-0.1in}   
\sectiontitle{Conference \\Posters}
\textsc{What Can Supernovae Tell us About Massive Stars}\\
{\bf Bostroem, K. A.}; Valenti, S.; Sand, D.; Morozova, V.; Jerkstrand, A.; Horesh, A.; et al, STScI Spring Symposium, 2019\\
\\
\textsc{Do Type IIP/IIL Supernovae Experience Interaction?}\\ 
{\bf Bostroem, K. A.}; Valenti, S.; Horesh, A.; Morozova, V.; Kuin, P.; Wyatt, S.; Jerkstrand, A.; Sand, D., American Astronomical Society, AAS Meeting \#233, 2018\\
\\
\textsc{Spectral Types and Wind Velocities for Massive Stars in R136}\\
\textbf{Bostroem, K. A.}; Ma�z Apell�niz, J.; Caballero-Nieves, S. M.; Walborn, N. R.; Crowther, P. A., American Astronomical Society, AAS Meeting \#223, 2014\\
\\
\textsc{New HST/STIS Spectroscopy of Massive Members of R136 in 30 Doradus}\\ 
{\bf Bostroem, K. A.}; Walborn, N.; Crowther, P.; Caballero-Nieves, S.; Lennon, D.; Ma\'{i}z Apell\'{a}niz, J., et al., Massive Stars for $\rm{\alpha}$ to $\rm{\Omega}$, 2013\\
\\
\textsc{Long-Slit Spectroscopy for 30 Doradus}\\ 
{\bf Bostroem, K. A.}; Walborn, N.; Crowther, P.; ; Lennon, D.; Ma\'{i}z Apell\'{a}niz, J.,  American Astronomical Society, AAS Meeting \#221, 2013\\
\\
\textsc{An Update on the Performance of the Space Telescope Imaging Spectrograph}\\
\textbf{Bostroem, K. Azalee}; Aloisi, A.; Bohlin, R. C.; Cox, C.; Diaz, R.; Dixon, W.; Duval, J.; Ely, J.; Mason, E.; Osten, R.; Proffitt, C.; Sonnentrucker, P.; Wolfe, M. A.; York, B.; Zheng, W., American Astronomical Society, AAS Meeting \#219, 2012\\
%-------------------------------------------------------------
%----------SERVICE & OUTREACH-----------------
%-------------------------------------------------------------
\sectiontitle{Service \\\& Outreach}
\textsc{Code Review Leader}, Davis, CA\hfill 2015-Present\\
Organize, and lead weekly meetings attended by graduate students, post-docs, research staff, and faculty to improve and share coding knowledge and best practices.\\
\\
\newpage
\textsc{Diversity and Inclusion in Physics Group Member and Co-Leader}, Davis, CA\hfill 2014-Present\\
Bring discussions, programs, and activities to the department to improve the departmental culture surrounding diversity, equity, and inclusion.\\
%\\
%\textsc{Astronomy on Tap}, Davis, CA\hfill 2019\\
%
\\
\textsc{Python in Astronomy SOC member} \hfill 2017-2018\\
Organized and coordinated the 2018 Python in Astronomy conference.\\
\\
\textsc{Python Lesson Maintainer}, Software Carpentry Lessons\hfill 2014-2016\\
Provide feedback on and incorporate suggested improvements to the Python lesson using git and GitHub.\\
\\
\textsc{Scipy Conference Proceedings Editor}\hfill 2015\\
Edited conference proceedings via GitHub for conference on scientific computing with Python.\\
\\
\textsc{HST Spectroscopic Legacy Working Group}, STScI\hfill2013-2014\\
Worked to define tools to improve spectroscopic archival products and interface for the HST spectrographs.\\
\\
\textsc{STScI Summer Student Selection Committee}, Baltimore, MD\hfill2012\\
Evaluated applications for the Space Astronomy Summer Program at STScI.\\
\\
\textsc{HST Calibration Workshop Organizing Committee}\hfill 2012\\
Organized and Coordinated the HST Calibration Workshop.\\
\\
\textsc{Project Astro Astronomer}, San Diego, CA\hfill 2007-2008\\
Brought hands on astronomy projects to a third grade class.\\
%-------------------------------------------------------------
%-----------TEACHING EXPERIENCE--------------
%-------------------------------------------------------------
\vspace{-0.1in}   
\sectiontitle{Teaching\\Experience}
\textsc{Lead Instructor, Software Carpentry Workshops}\hfill 2012-present\\
Lead two day workshops at institutions and conferences internationally to enable scientists to work more efficiently and reproducibly by improving their coding skills. Workshops include Python, Git, Unix, test driven development, object oriented and functional programming. 
%\begin{itemize}
%\item []Tel Aviv Univeristy, 2019, Tel Aviv, Israel %21st Century Software and Data Analysis Tools for Physics and Astronomy
%\item []American Astronomical Society 233rd Meeting, 2019, Seattle, WA 
%\item []American Astronomical Society 231st Meeting, 2018, National Harbor,  MD 
%\item []American Astronomical Society 229th Meeting, 2017, Grapevine, TX 
%\item []American Astronomical Society 227th Meeting, 2016, Kissammee, FL 
%\item []Women in Science and Engineering, 2015, Davis, CA 
%\item []American Astronomical Society 225th Meeting, 2015, Seattle, WA 
%\item []Stanford University, 2014, Palo Alto, CA 
%\item []Women in Science and Engineering, 2014, Davis, CA 
%\item []European Space Astronomy Centre, 2014, Cebreros, Avila, Spain 
%\item []Space Telescope Science Institute, 2013, Baltimore, MD
%\item []University of California, Davis, 2013, Davis, CA
%\item []Lawrence Berkeley National Laboratory, 2013, Berkeley, CA
%\item []University of Maryland, 2013, Baltimore, MD
%\item []University of Edinburgh, 2012, Edinburgh, Scotland, UK
%\item []George Mason University, 2012, Fairfax, VA
%\end{itemize}
\\
\textsc{Teaching Assistant, University of California} (PHYS 7A, 7B) \hfill 2014-2018\\
Guided students through lab work to understand thermodynamics, mechanics, electrical circuits, and fluid dynamics with lecture, small group work, and whole class discussions.\\
%TODO: Add course evaluations
\\
\textsc{Lead Teaching Assistant, San Diego State University}\hfill Fall 2008\\
Supervised astronomy lab instructors and prepared them to teach each week. \\
\\
\textsc{Teaching Assistant, San Diego State University}\hfill 2006-2009\\
Enabled a better understanding of general astronomy through hand on applications of topics covered in lecture. Prepared labs, developed lesson plans and curriculum.\\
\\
\textsc{Mathematics Teacher, Ross Valley Summer School}\hfill Summer 2006\\
Designed the curriculum and lesson plans for and taught first through sixth grade mathematics.\\
\\
\textsc{Mathematics Student Teacher, Poughkeepsie High School}\hfill Fall 2006\\
Created lesson plans and taught a full course load (five periods) of high school mathematics.\\
\\
\textsc{Science Teacher and Curriculum Designer, Crossroads Summer Camp}\hfill Summer 2003, 2004 \\
Created a curriculum and lesson plans for and taught sixth, seventh, and eighth grade science courses.
%
%-------------------------------------------------------------
%-----------------REFERENCES-----------------------
%-------------------------------------------------------------
%\sectiontitle{References}
%Stefano Valenti \\
%Alessandra Aloisi \\
%Ori Fox \\
\end{llist}
\end{document}






























