% LaTeX resume using res.cls
\let\nofiles\relax  %let latex create .aux
\documentclass{res}
\usepackage{bibentry}
\usepackage{url}
\usepackage{color}
\usepackage{aas_macros}
\usepackage{hanging}
\usepackage{natbib}
\bibliographystyle{apj}
\newcommand\hangbibentry[1]{%
    \smallskip\par\hangpara{1em}{1}\bibentry{#1}\smallskip\par %{indent}{afterline}
}

%\usepackage{helvetica} % uses helvetica postscript font (download helvetica.sty)
%\usepackage{newcent}   % uses new century schoolbook postscript font  
\setlength{\topmargin}{-0.6in}  % Start text higher on the page 
\setlength{\textheight}{9.8in}  % increase textheight to fit more on a page
\setlength{\headsep}{0.2in}     % space between header and text
\setlength{\headheight}{12pt}   % make room for header
\usepackage{fancyhdr}  % use fancyhdr package to get 2-line header
\renewcommand{\headrulewidth}{0pt} % suppress line drawn by default by fancyhdr
%\lhead{\hspace*{-\sectionwidth}Justin Ely} % force lhead all the way left
%\rhead{Page \thepage}  % put page number at right
\cfoot{}  % the footer is empty
\pagestyle{fancy} % set pagestyle for the document

\begin{document} 
\nobibliography{refs.bib}

\section{{\LARGE \bf{K. Azalee Bostroem}}}\\
{\rule{\linewidth}{0.5mm}}
Space Telescope Science Institute \hfill {\color{blue}\url{http://azaleebostroem.wordpress.com}}\\
P631  \hfill {\color{blue}\url{https://github.com/abostroem}} \\
3700 San Martin Dr. \\
Baltimore, MD 21210  \\
bostroem@stsci.edu (410) 338 - 4459

\begin{resume}

\section{EDUCATION}
\vspace{-.2in} 
\begin{center}
\line(1,0){425}
\end{center}
\vspace{-.3in} 
\vspace{0.1in} 
 
    M. S. in Astronomy\\
    San Diego State University, San Diego, CA, May 2009 \\
    GPA: 3.82/4.0 \\
    \emph{Graduate courses in astronomy with research thesis} \\
    
    
    California State Secondary Teaching Certification in Mathematics \\
 
    
    B.A. in Mathematics \\
    Minor in Hispanic Studies \\
    New York State Secondary Teaching Certification in Mathematics \\
    Vassar College, Poughkeepsie, NY, May 2006\\
    GPA: 3.68/4.0   
 
\section{HONORS}
\vspace{-.2in} 
\begin{center}
\line(1,0){425}
\end{center}
\vspace{-.3in} 
\vspace{0.1in} 

NASA Group Achievement Award, HST SM4 Servicing Implementation Team, 2010 \\
Cliff E. Smith and Ruth Kinnell Graduate Fellowship, 2007 - 2008 \\
 
\section{EMPLOYMENT}
\vspace{-.2in} 
\begin{center}
\line(1,0){425}
\end{center}
\vspace{-.3in} 
\vspace{0.1in} 
    {\bf Senior Research and Instrument Analyst} COS/STIS Team, Space Telescope Science Institute, Baltimore, MD
    2012 - Present

    {\bf Reseach and Instrument Analyst II,} COS/STIS Team, Space Telescope Science Institute, Baltimore, MD 
    2009 - 2012 
 
    {\bf Astronomy Lab Instructor and Teaching Assistant} Dept. of Astronomy, San Diego State University, San Diego, CA  Fall 2006 - Spring 2009 
 
    {\bf Lead Teaching Assistant,} Dept. of Astronomy, San Diego State University, San Diego, CA, Fall 2008
    

\section{RESEARCH} 
\vspace{-.2in}
\begin{center}
\line(1,0){425}
\end{center}
\vspace{-.3in}
\vspace{0.1in}
   {\bf  Supernova 1993J}
          \begin{itemize}
   	\item[] Supervisor: Alex Filippenko

        \item[] Manually calibrated HST/COS FUV and NUV spectra from HST/GO program 12531 of Supernova 1993J to evaluate supernova cooling models, 
        supernova progenitor models, and the possible companion star of Supernova 1993J. 
        \end{itemize} 
 
   {\bf  Characterization of Massive Stars in R136 } 
        \begin{itemize}
        \item[] Supervisor: Paul A. Crowther
        \item[] Calibrated HST/STIS FUV and optical spectra from HST/GO programs 12465 and 13052 to derive properties such as mass, metallicity, stellar wind velocity, binarity, and spectral class for massive stars in the star cluster R136. The understanding of these parameters for a resolved star cluster informs models of unresolved stellar cluster in other galaxies. In this project I developed tools that improved the calibration of the data and facilitated the extraction of the spectra.
        \end{itemize} 
        
	{\bf CANDELS Supernova Project}
		\begin{itemize}
		\item[] Supervisor: Adam Riess
		\item[] Search for supernova in the CLASH/CANDELS images to generate a sample of high-redshift Type Ia supernova to better constrain cosmological parameters and Type Ia supernova progenitor models
		\end{itemize}

   {\bf  A New calibration of the Tully-Fisher Relation and Estimate of Hubble's Constant} 
           \begin{itemize}
   	\item[] Supervisor: Douglas Leonard
        \item[] Master's thesis project to recalibrate the zero point of the Tully-Fisher relation using a homogeneous
        sample of Cepheid variable stars and use the relation to calculate Hubble's Constant. In this project, I derived I-band Tully-Fisher relation using the SFI++ catalog. Using this relation, those of Masters et al. (2008), and a uniform photometric analysis of the cepheid distance scale I calculated Hubble's Constant for the I, J, K, and H bands.
        \end{itemize} 
        
   {\bf Photometry of Open Star Clusters }
      		\begin{itemize}
   		\item[] Supervisor: Eric Sandquist
		\item[] Studied detached binaries near the turn off point of the main sequence of open star clusters to derive an accurate age for each cluster. I observed and reduced optical data taken at the Mount Laguna Observatory. I performed aperture photometry and calibrated fluxes to an absolute scale to create color magnitude diagrams.
   		\end{itemize}
   		
   {\bf Pre-Service Teacher: Universe Adventure}
   		\begin{itemize}
   		\item[] Supervisor: George Smoot
		\item[] The Universe Adventure Website ({\color{blue}\url{http://universeadventure.org}}) is an educational website to be explored by individual users and with supplemental material for classroom use. It provides easy to understand explanations, graphics, videos, and exercises to help users understand cosmological concepts. In the Pre-Service Teacher program I designed webpages, teaching materials, and classroom activities for the website.

		\end{itemize}

\section{HST INSTRUMENT SUPPORT} 
\vspace{-.2in}
\begin{center}
\line(1,0){425}
\end{center}
\vspace{-.3in}
\vspace{0.1in}

	{\bf HST/COS and HST/STIS Calibration Pipeline Lead}
		\begin{itemize}
		\item[]  As calibration pipeline lead, I organized and lead bi-weekly meetings, supervised the testing of the HST/COS and HST/STIS calibration pipelines, and coordinated the HST/COS and HST/STIS calibration pipeline work with the HST/COS and HST/STIS teams and the archive and pipeline developers. I also prioritized HST/COS and HST/STIS pipeline development items and oversaw the implementation of improvements to the HST/COS and HST/STIS calibration pipeline.
		\end{itemize}


   {\bf  Instrument Monitoring and Calibration}
        \begin{itemize}
        \item[]  As a member of the HST/STIS team I created and tested HST/STIS CCD Dark and Bias reference files. I also characterized the flux calibration and blaze shift correction for the HST/STIS echelle modes following Servicing Mission 4 and developed tools to monitor the HST/STIS time-dependent sensitivity. I continued this work on the HST/COS team developing tools to monitor the HST/COS FUV and NUV time-dependent sensitivity and create and test new reference files as needed. Additionally, I have created and tested throughput reference files for HST/COS and HST/STIS to be used in the Exposure Time Calculator.
		\end{itemize}
		
	{\bf Summer Student Supervision}
		\begin{itemize}
		\item[]  Supervised the characterization of the HST/COS and HST/STIS NUV detector vignetting by Kenneth Hart.
		\item[]  Supervised the characterization of the HST/COS throughput as a function of wavelength by Inna Bojinova.
		\item[]  Supervised the characterization of the HST/COS throughput as a function of time by Martha Saladino.
		\end{itemize}

   {\bf  User Support}
   		\begin{itemize}
		As user support deputy for the HST/COS and HST/STIS teams, I maintained the internal and external web pages, created and delivered Space Telescope Analysis Newsletters, and tracked user support issues within the team. I also support the Spectrographs' Help Desk, answering questions from the community about HST/COS, HST/STIS, HST/GHRS, and HST/FOS
		\end{itemize}
	
	


\section{COMPUTER SKILLS}
\vspace{-.2in}
\begin{center}
\line(1,0){425}
\end{center}
\vspace{-.3in}
\vspace{0.1in}
Advanced: Python, IDL, GIT, IRAF/PyRAF, BASH

Basic: APT, HTML, SQL, LaTeX, fortran
 

\section{ADDITIONAL TRAINING}
\vspace{-.2in} 
\begin{center}
\line(1,0){425}
\end{center}
\vspace{-.24in} 
\vspace{0.1in}

Python Software Carptentry, \\
Instr: Greg Wilson, Spring 2012, STScI, MD

Statistics in Astronomy, \\
Instr: Penn State Mathematics Dept., Fall 2011, STScI, MD

Advanced Python Mastery, \\
Instr: David Beazley, Spring 2011, STScI, MD

\section{HST CALIBRATION PROPOSALS}
\vspace{-.2in} 
\begin{center}
\line(1,0){425}
\end{center}
\vspace{-.25in} 
\vspace{0.1in}

13548, Cycle 21, MAMA Spectroscopic Sensitivity and Focus Monitor Cycle 21 and COS Observations of Geocoronal Lyman-Alpha Emission \\
H. Sana, R. A. Osten, C. R. Proffitt, {\bf K. A. Bostroem}

13544, Cycle 21, STIS/CCD Spectroscopic Sensitivity Monitor for Cycle 21 \\
H. Sana, R. A. Osten, {\bf K. A. Bostroem}, C. R. Proffitt

13527, Cycle 21, NUV Spectroscopic Sensitivity Monitoring \\
{\bf K. A. Bostroem}, J. Taylor, C. R. Proffitt

13520, Cycle 21, COS FUV Spectroscopic Monitoring \\
{\bf K. A. Bostroem}, J. H. Debes, C. R. Proffitt

13447, Cycle 21, The Massive Monsters Living Deep in the Tarantula Nebula: How Massive are they Really? \\
S. E. de Mink, S. C. Nieves, {\bf K. A. Bostroem}, P. A. Crowther, et al.

13145, Cycle 20, MAMA Spectroscopic Sensitivity and Focus Monitor Cycle 20 and COS Observations of Geocoronal Lyman-Alpha Emission \\
S. T. Holland, R. A. Osten, C. R. Proffitt, {\bf K. A. Bostroem}

13141, Cycle 20, STIS/CCD Spectroscopic Sensitivity Monitor for Cycle 20 \\
S. T. Holland, R. A. Osten, {\bf K. A. Bostroem}, C. R. Proffitt

13125, Cycle 20, NUV Spectroscopic Sensitivity Monitoring \\
{\bf K. A. Bostroem}, R. A. Osten, C. R. Proffitt

13119, Cycle 20, COS FUV Spectroscopic Sensitivity Monitoring \\
{\bf K. A. Bostroem}, R. A. Osten, C. R. Proffitt

12796, Cycle 19, Second COS FUV Lifetime Position: Focus Sweep Enabling Program (FENA3) \\
C. Oliveira, {\bf K. A. Bostroem}

12775, Cycle 19, MAMA Spectroscopic Sensitivity and Focus Monitor \\
{\bf K. A. Bostroem}, R. A. Osten, C. R. Proffitt. S. T. Holland

12772, Cycle 19, STIS/CCD Spectroscopic Sensitivity Monitor \\
{\bf K. A. Bostroem}, R. A. Osten, C. R. Proffitt

12715, Cycle 19, COS FUV Spectroscopic Sensitivity Monitoring \\
R. A. Osten, C. D. Keyes, D. J. Sahnow, A. Aloisi, {\bf K. A. Bostroem}

12414, Cycle 18, MAMA Spectroscopic Sensitivity and Focus Monitor Cycle 18 and COS Observations of Geocoronal Lyman-Alpha Emission \\
R. A. Osten, C. R. Proffitt, {\bf K. A. Bostroem}

12411, Cycle 18, STIS/CCD Spectroscopic Sensitivity Monitor for Cycle 18 \\
R. A. Osten, {\bf K. A. Bostroem}, C. R. Proffitt



\section{PROFESSIONAL AFFILIATIONS}
\vspace{-.2in} 
\begin{center}
\line(1,0){425}
\end{center}
\vspace{-.3in}  
\vspace{0.1in} 
    American Astronomical Society, 2008 - present

    Software Carpentry, 2012 - present
    
\section{PUBLICATIONS}
\vspace{-.2in} 
\begin{center}
\line(1,0){425}
\end{center}
\vspace{-.25in} 
\vspace{0.1in}


Fox, O. D., {\bf Bostroem, K. A.}, Van Dyke, S., et al., {\bf A Search for the B-Star Binary Companion
of the SN 1993J Progenitor}, \emph{in prep}

Rodney, S. A., Riess, A. G., Strolger, L. G., et al., {\bf Type Ia Supernova Rate Measurements to Redshift 2.5 from CANDELS: Searching for Prompt Explosions in the Early Universe}, \emph{in prep}

Osten, R. A. and {\bf Bostroem, K. A.}, {\bf Examining the COS Spectroscopic Sensitivity Monitors}, \emph{in prep}

Hart, K. and {\bf Bostroem, K. A.}, {\bf Characterization of the STIS NUV MAMA Vignetting by Optical Element}, \emph{in prep}

\hangbibentry{2013A&A...558A..33A}
\hangbibentry{2013cos..rept...16O}
\hangbibentry{2013cos..rept....4K}
\hangbibentry{2013cos..rept....1O}
\hangbibentry{2012cos..rept....2O}
\hangbibentry{2012stis.rept....1B}
\hangbibentry{2012stis.rept....3W}
\hangbibentry{2011cos..rept....2O}
\hangbibentry{2011stis.book.....B}



\section{PRESENTATIONS}
\vspace{-.2in} 
\begin{center}
\line(1,0){425}
\end{center}
\vspace{-.3in} 
\vspace{0.1in} 
{\bf Bostroem, K. A.}, {\bf Post-SM4 Sensitivity Calibration of the STIS Echelle Modes}, STScI TIPS/JIM Monthly Meeting, March 2012

\section{CONFERENCE ABSTRACTS/PROCEEDINGS}
\vspace{-.2in} 
\begin{center}
\line(1,0){425}
\end{center}
\vspace{-.24in} 
\vspace{0.1in}

\hangbibentry{2013msao.confE..56B}
\hangbibentry{2013AAS...22231607H}
\hangbibentry{2013AAS...22231606T}
\hangbibentry{2013AAS...22134404P}
\hangbibentry{2013AAS...22134403O}
\hangbibentry{2013AAS...22125004B}
\hangbibentry{2012SPIE.8443E..4CS}
\hangbibentry{2012AAS...22013605E}
\hangbibentry{2012AAS...22013601D}
\hangbibentry{2012AAS...21924116E}
\hangbibentry{2012AAS...21924115B}
\hangbibentry{2011SPIE.8145E.257S}
\hangbibentry{2011AAS...21833102W}
\hangbibentry{2011AAS...21725402P}
\hangbibentry{2010hstc.workE..51B}
\hangbibentry{2010hstc.workE..47Z}
\hangbibentry{2010hstc.workE...6P}
\hangbibentry{2010hstc.workE...3A}
\hangbibentry{2010AAS...21641306W}

\section{TEACHING EXPERIENCE}
\vspace{-.2in} 
\begin{center}
\line(1,0){425}
\end{center}
\vspace{-.24in} 
\vspace{0.1in}
   {\bf  Workshop Co-Leader, Software Carpentry Bootcamps } \\
         Two day workshop introducing scientists to the programming skills to improve their research including Python, GIT, SVN, Shell, test-driven developement, object oriented and functional programming. \\
        \begin{itemize}
        \item[] STScI, Sept 2013, Baltimore, MD
        \item[] UC Davis, May 2013, Davis, CA
        \item[] Lawrence Berkeley National Laboratory, May 2013, Berkeley, CA
        \item[] University of Maryland, April 2013, Baltimore, MD
        \item[] University of Edinburgh, December 2012, Edinburgh, United Kingdom
        \item[] George Mason University, October 2012, Fairfax, VA
        \end{itemize} 
    {\bf  Lead Teaching Assistant,} Fall 2008  \\
         Student supervisor of all Introduction to Astronomy Lab instructors at San Diego State University, 
         San Diego, CA. Duties included editing the Lab Manual and training new instructors. \\
		 
   {\bf  Introduction to Astronomy Lab Instructor,} Fall 2006 - Spring 2009  \\
         Taught the Introduction to Astronomy Lab at San Diego State University, San Diego, CA. 
         Designed curriculum and created lesson plans. \\
         
   {\bf Project Astro Astronomer, } Fall 2007 - Spring 2008 \\
   		Brought hands on astronomy projects to a 3rd grade class in San Diego, CA. \\

   {\bf Private Tutor, } Fall 2006 - Spring 2009 \\
   		Tutors middle and high school students in mathematics, physics, chemistry, and Spanish in Escondido, CA. \\
   		
   {\bf Mathematics Teacher, } Summer 2006 \\
   		Designed the curriculum for, created lesson plans for, and taught 1st - 6th grade 
   		mathematics at the Ross Valley Summer School, Ross Valley, CA. \\
   	
   {\bf Student Teacher, } Fall 2006 \\
   		Taught 5 periods of High School mathematics at Poughkeepsie High School, Poughkeepsie, NY \\
   	
   {\bf Science Teacher, } Summer 2003, 2004 \\
   		Designed the curriculum for, created lesson plans for, and taught 6th, 7th, and 8th grade 
   		science courses at Crossroads Summer Camp, San Rafael, CA.


\end{resume}
\end{document}






























