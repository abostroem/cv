% LaTeX resume using res.cls
\let\nofiles\relax  %let latex create .aux
\documentclass{res}
\usepackage{bibentry}
\usepackage{url}
\usepackage{color}
\usepackage{aas_macros}
\usepackage{hanging}
\usepackage{natbib}
\bibliographystyle{apj}
\newcommand\hangbibentry[1]{%
    \smallskip\par\hangpara{1em}{1}\bibentry{#1}\smallskip\par %{indent}{afterline}
}

%\usepackage{helvetica} % uses helvetica postscript font (download helvetica.sty)
%\usepackage{newcent}   % uses new century schoolbook postscript font  
\setlength{\topmargin}{-0.6in}  % Start text higher on the page 
\setlength{\textheight}{9.8in}  % increase textheight to fit more on a page
\setlength{\headsep}{0.2in}     % space between header and text
\setlength{\headheight}{12pt}   % make room for header
\usepackage{fancyhdr}  % use fancyhdr package to get 2-line header
\renewcommand{\headrulewidth}{0pt} % suppress line drawn by default by fancyhdr
%\lhead{\hspace*{-\sectionwidth}Justin Ely} % force lhead all the way left
%\rhead{Page \thepage}  % put page number at right
\cfoot{}  % the footer is empty
\pagestyle{fancy} % set pagestyle for the document

\begin{document} 
\nobibliography{refs.bib}

\section{{\LARGE \bf{K. Azalee Bostroem}}}
{\rule{\linewidth}{0.5mm}} \\
Department of Physics \hfill {\color{blue}\url{http://azaleebostroem.wordpress.com}}\\
University of California, Davis \hfill {\color{blue}\url{https://github.com/abostroem} }\\
1 Shields Ave. \\
Davis, CA, 95616 \\
kabostroem@ucdavis.edu
\begin{resume}
\section{EDUCATION}
\vspace{-.2in} 
\begin{center}
\line(1,0){425}
\end{center}
\vspace{-.3in} 
\vspace{0.1in} 
 
   {\bf Ph.D. in Physics} \\
   University of California at Davis, Davis, CA, In Progress \\
   GPA: 3.9/4.0 \\
   \emph{Graduate courses in Physics and Astrophysics with research thesis} \\
    {\bf M.S. in Astronomy} \\
    San Diego State University, San Diego, CA, May 2009 \\
    GPA: 3.82/4.0 \\
    \emph{Graduate courses in astronomy with research thesis} \\
    {\bf California State Secondary Teaching Certification in Mathematics} \\
    {\bf B.A. in Mathematics }\\
    Minor in Hispanic Studies \\
    New York State Secondary Teaching Certification in Mathematics \\
    Vassar College, Poughkeepsie, NY, May 2006\\
    GPA: 3.68/4.0   
 
\section{HONORS}
\vspace{-.2in} 
\begin{center}
\line(1,0){425}
\end{center}
\vspace{-.3in} 
\vspace{0.1in} 

NASA Hubble 25th Anniversary Commendation, HST Science Team, 2016 \\ %	CHECK THIS DATE
Ray and Constance Chandler Fellowship, 2014-2015 \\
NASA Group Achievement Award, HST SM4 Servicing Implementation Team, 2010 \\
Cliff E. Smith and Ruth Kinnell Graduate Fellowship, 2007 - 2008 \\
 
\section{EMPLOYMENT}
\vspace{-.2in} 
\begin{center}
\line(1,0){425}
\end{center}
\vspace{-.3in} 
\vspace{0.1in} 

    {\bf Graduate Student Researcher} , Department of Physics, University of California at Davis, Davis, CA, 2016 \\
    {\bf Supernova Spectroscopy Analyst}, Department of Physics and Astronomy, University of Pittsburgh, Pittsburgh, PA, Fall 2015 \\
    {\bf Research and Instrument Analyst} COS/STIS Team, Space Telescope Science Institute, Baltimore, MD, 2009 - 2014 \\   
    
\section{RESEARCH} 
\vspace{-.2in}
\begin{center}
\line(1,0){425}
\end{center}
\vspace{-.3in}
\vspace{0.1in}

{\bf ASASSN 2015oz}
    \begin{itemize}
    \item[] \emph{Supervisor: Stefano Valenti}
    \item[] Reduce spectroscopic observations of supernova AS-ASSN 2015oz taken over the course of a year as part of the PESSTO project using two different telescopes.
        \end{itemize}
{\bf Extinction in the LMC}
     \begin{itemize}
    \item[] \emph{Supervisor: Jesus Maiz Apellaniz}
    \item[] Extract and analyze long-slit HST/STIS spectra of 10 slit positions throughout 30 Doradus, with varying degrees of extinction to create a family of UV-optical extinction laws. 
    \end{itemize}  

{\bf Telluric Line Correction using PWV Measurements}
    \begin{itemize}
    \item[] \emph{Supervisor: Michael Wood-Vasey}
    \item[] Correct NIR spectra at Kitt Peak National Observatory for atmospheric effected based on precipitable water vapor levels measured from GPS stations. 
    \end{itemize}

   {\bf  Supernova 1993J}
          \begin{itemize}
   	\item[] \emph{Supervisor: Alex Filippenko}
        \item[] Manually calibrated HST/COS FUV and NUV spectra from HST/GO program 12531 of Supernova 1993J to evaluate supernova cooling models, supernova progenitor models, and the possible companion star of Supernova 1993J. 
        \end{itemize} 
 
   {\bf  Characterization of Massive Stars in R136 } 
        \begin{itemize}
        \item[] \emph{Supervisor: Paul A. Crowther}
        \item[] Calibrated HST/STIS FUV and optical spectra from HST/GO programs 12465 and 13052 to derive properties such as mass, metallicity, stellar wind velocity, binarity, and spectral class for massive stars in the star cluster R136. 
        \end{itemize} 
        
	{\bf CANDELS Supernova Project}
		\begin{itemize}
		\item[] \emph{Supervisor: Adam Riess}
		\item[] Search for supernova in the CLASH/CANDELS images to generate a sample of high-redshift Type Ia supernova to better constrain cosmological parameters and Type Ia supernova progenitor models
		\end{itemize}

   {\bf  A New calibration of the Tully-Fisher Relation and Estimate of Hubble's Constant} 
           \begin{itemize}
   	\item[] \emph{Supervisor: Douglas Leonard}
        \item[] Master's thesis project to recalibrate the zero point of the Tully-Fisher relation using a homogeneous
        sample of Cepheid variable stars and use the relation to calculate Hubble's Constant. 
         \end{itemize} 
        
   {\bf Photometry of Open Star Clusters }
      		\begin{itemize}
   		\item[] \emph{Supervisor: Eric Sandquist}
		\item[] Studied detached binaries near the turn off point of the main sequence of open star clusters to derive an accurate age for each cluster. I observed and reduced optical data taken at the Mount Laguna Observatory. 
   		\end{itemize}
   		
   {\bf Pre-Service Teacher: Universe Adventure}
   		\begin{itemize}
   		\item[] \emph{Supervisor: George Smoot}
		\item[] The Universe Adventure Website ({\color{blue}\url{http://universeadventure.org}}) is an educational website to be explored by individual users and with supplemental material for classroom use. 
		\end{itemize}

\section{HST INSTRUMENT SUPPORT} 
\vspace{-.2in}
\begin{center}
\line(1,0){425}
\end{center}
\vspace{-.3in}
\vspace{0.1in}

	{\bf HST/COS and HST/STIS Calibration Pipeline Lead}
		\begin{itemize}
		\item[]  As calibration pipeline lead, I organized and lead bi-weekly meetings, supervised the testing of the HST/COS and HST/STIS calibration pipelines, and coordinated the HST/COS and HST/STIS calibration pipeline work with the HST/COS and HST/STIS teams and the archive and pipeline developers. I also prioritized HST/COS and HST/STIS pipeline development items and oversaw the implementation of improvements to the HST/COS and HST/STIS calibration pipeline.
		\end{itemize}


   {\bf  Instrument Monitoring and Calibration}
        \begin{itemize}
        \item[]  As a member of the HST/STIS team I created and tested HST/STIS CCD Dark and Bias reference files. I also characterized the flux calibration and blaze shift correction for the HST/STIS echelle modes following Servicing Mission 4 and developed tools to monitor the HST/STIS time-dependent sensitivity. I continued this work on the HST/COS team developing tools to monitor the HST/COS FUV and NUV time-dependent sensitivity and create and test new reference files as needed. Additionally, I have created and tested throughput reference files for HST/COS and HST/STIS to be used in the Exposure Time Calculator.
		\end{itemize}
		
	{\bf Summer Student Supervision}
		\begin{itemize}
		\item[]  Supervised the characterization of the HST/COS and HST/STIS NUV detector vignetting by Kenneth Hart.
		\item[]  Supervised the characterization of the HST/COS throughput as a function of wavelength by Inna Bojinova.
		\item[]  Supervised the characterization of the HST/COS throughput as a function of time by Martha Saladino.
		\end{itemize}

   {\bf  User Support}
   		\begin{itemize}
		\item[] As user support deputy for the HST/COS and HST/STIS teams, I maintained the internal and external web pages, created and delivered Space Telescope Analysis Newsletters, and tracked user support issues within the team. I also support the Spectrographs' Help Desk, answering questions from the community about HST/COS, HST/STIS, HST/GHRS, and HST/FOS
		\end{itemize}
	

\section{HST PROPOSALS}
14599, Cycle 24, UV Spectroscopic Signatures from Type Ia Supernovae Strongly Interacting with a Circumstellar Medium\\
O. Fox, A. Filippenko, J. Silverman, R. Foley, {\bf K. A. Bostroem}, M. Graham, P. Kelly, J. Mauerhan, S. B. Cenko

14075, Cycle 23, Long-Lost Companions: A Search for the Binary Secondaries of Three Nearby Supernovae \\
O. Fox, S. D. Van Dyk,  A. Filippenko, S. de Mink, N. Smith, S. Ryder, J. Silverman, P. Kelly, J. Mauerhan, E. Zapartas, W. Zheng, K. A. Bostroem, I. Shivvers, H. Yuk           

13648, Cycle 22, Uncovering the Putative B-Star Binary Companion of the Sn 1993J Progenitor\\
O. Fox, {\bf K. A. Bostroem}, S. Van Dyk, A. Filippenko, T. Matheson, P. Chandra, V. Dwarkadas, N. Smith, C. Fransson 

13649, Cycle 22, UV Spectroscopic Signatures from Type Ia Supernovae Strongly Interacting with a Circumstellar Medium \\
O. Fox, A. Filippenko, J. Silverman, R. Foley, {\bf K. A. Bostroem}, M. Graham, P. Kelly, J. Mauerhan, S. Cenko

15 HST Calibration Proposals including 6 as PI



\section{PROFESSIONAL AFFILIATIONS}
\vspace{-.2in} 
\begin{center}
\line(1,0){425}
\end{center}
\vspace{-.3in}  
\vspace{0.1in} 
    American Astronomical Society, 2008 - present

    Software Carpentry, 2012 - present
    
\section{REFEREED PUBLICATIONS}
\vspace{-.2in} 
\begin{center}
\line(1,0){425}
\end{center}
\vspace{-.25in} 
\vspace{0.1in}

\hangbibentry{2016MNRAS.458..624C}
\hangbibentry{2015MNRAS.454.4366F}
\hangbibentry{2014ApJ...790...17F}
\hangbibentry{2014AJ....148...13R}
\hangbibentry{2013A&A...558A..33A}

\section{TECHNICAL REPORTS}
\vspace{-.2in} 
\begin{center}
\line(1,0){425}
\end{center}
\vspace{-.25in} 
\vspace{0.1in}
\hangbibentry{2012stis.rept....1B}
\hangbibentry{2011stis.book.....B}
16 technical reports with Bostroem as co-author
5 presentations with Bostroem as first author and 21 presentations with Bostroem as co-author 

\section{TEACHING EXPERIENCE}
\vspace{-.2in} 
\begin{center}
\line(1,0){425}
\end{center}
\vspace{-.24in} 
\vspace{0.1in}
   {\bf Teaching Assistant} Fall 2014-present \\
        Taught physics discussion lab for life science students and physics discussion section for physics majors at UC Davis, Davis, CA. \\
   {\bf  Workshop Co-Leader, Software Carpentry Bootcamps } \\
         Two day workshop introducing scientists to the programming skills to improve their research including Python, GIT, SVN, Shell, test-driven developement, object oriented and functional programming. I have taught 15 workshops between 2012 and the present \\		 
   {\bf  Introduction to Astronomy Lab Instructor,} Fall 2006 - Spring 2009  \\
         Taught the Introduction to Astronomy Lab at San Diego State University, San Diego, CA. 
         Designed curriculum and created lesson plans. \\
   {\bf Project Astro Astronomer, } Fall 2007 - Spring 2008 \\
   		Brought hands on astronomy projects to a 3rd grade class in San Diego, CA. \\
   {\bf Mathematics Teacher, } Summer 2006 \\
   		Designed the curriculum for, created lesson plans for, and taught 1st - 6th grade 
   		mathematics at the Ross Valley Summer School, Ross Valley, CA. \\	
   {\bf Student Teacher, } Fall 2006 \\
   		Taught 5 periods of High School mathematics at Poughkeepsie High School, Poughkeepsie, NY \\
   {\bf Science Teacher, } Summer 2003, 2004 \\
   		Designed the curriculum for, created lesson plans for, and taught 6th, 7th, and 8th grade 
   		science courses at Crossroads Summer Camp, San Rafael, CA.


\end{resume}
\end{document}






























