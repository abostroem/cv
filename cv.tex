\documentclass[10pt]{cv}
\font\cap=cmcsc10
\usepackage{amsmath}

\topmargin 0pt
\headheight 0pt
\headsep 0pt
\textheight 674pt
\pagestyle{empty}
\parindent 0.08in
\parskip \baselineskip
\topmargin 0in
\footskip 1in
%\oddsidemargin 0in
\oddsidemargin -0.25in
%\evensidemargin 0in
\evensidemargin -0.25in

\textwidth 6.9in

\setlength{\topmargin}{-0.25in}
\setlength{\headheight}{0in}
\setlength{\headsep}{0in}
\setlength{\topskip}{0.75in}
\setlength{\textheight}{9in}
\pagestyle{myheadings}
\setlength{\parskip}{0in}

%set the number in the second set of curly braces to the number you want displayed on
%the first page
%setcounter{page}{2}
  

\usepackage{bibentry}
\usepackage{url}
\usepackage{color}
\usepackage{aas_macros}
\usepackage{hanging}
\usepackage{natbib}
\bibliographystyle{apj}
\newcommand\hangbibentry[1]{%
    \smallskip\par\hangpara{1em}{1}\bibentry{#1}\smallskip\par %{indent}{afterline}
}

%\usepackage{helvetica} % uses helvetica postscript font (download helvetica.sty)
%\usepackage{newcent}   % uses new century schoolbook postscript font  
\setlength{\topmargin}{-0.6in}  % Start text higher on the page 
\setlength{\textheight}{9.5in}  % increase textheight to fit more on a page
\setlength{\headsep}{0.2in}     % space between header and text
\setlength{\headheight}{12pt}   % make room for header
\usepackage{fancyhdr}  % use fancyhdr package to get 2-line header
\renewcommand{\headrulewidth}{0pt} % suppress line drawn by default by fancyhdr
%\lhead{\hspace*{-\sectionwidth}Justin Ely} % force lhead all the way left
%\rhead{Page \thepage}  % put page number at right
\cfoot{}  % the footer is empty
%\pagestyle{fancy} % set pagestyle for the document

\begin{document} 
\nobibliography{refs.bib}

\section{{\LARGE \bf{K. Azalee Bostroem}}}
{\rule{\linewidth}{0.5mm}} \\
\begin{minipage}{0.55\textwidth}
Department of Physics \\
University of California, Davis \\
Davis, CA
\end{minipage}
\begin{minipage}{0.45\textwidth}
\begin{flushright}
\vspace{0.25cm}
\color{blue}\url{kabostroem@ucdavis.edu}\\
\color{blue}\url{https://github.com/abostroem} \\
\color{blue}\url{https://abostroem.wixsite.com/home}
\end{flushright}
\end{minipage} \\
{\rule{\linewidth}{0.5mm}} 
\vspace{-.5em}
%\begin{resume}
\begin{llist}
\vspace{0.1in} 
\sectiontitle{Education}
%\vspace{0.1in} 
\textsc{University of California, Davis}\hfill Sept 2014-Present\\
Davis, CA, USA \\
Ph.D. in Physics \\
Thesis Advisor: Stefano Valenti \\
\\
\textsc{San Diego State University}\hfill May 2009\\
San Diego, CA, USA \\
M.S. in Astronomy \\
Thesis Advisor: Douglas Leonard \\
\\
\textsc{Vassar College} \hfill May 2006 \\
Poughkeepsie, NY, USA \\
B.A. in Mathematics \\
California State and New York State Secondary Teaching Certifications in Mathematics \\
\vspace{-0.1in}   
\sectiontitle{Honors}
%\vspace{0.1in}
NASA Hubble 25th Anniversary Commendation, HST Science Team \hfill 2016 \\ %	CHECK THIS DATE
Ray and Constance Chandler Fellowship \hfill 2014-2015 \\
NASA Group Achievement Award, HST SM4 Servicing Implementation Team \hfill2010 \\
Cliff E. Smith and Ruth Kinnell Graduate Fellowship \hfill 2007 - 2008\\
\vspace{-0.1in}   
\sectiontitle{Proposals\\\& Grants}
%\vspace{-0.3in}
\$40,000, AIP Venture Partnership Fund \hfill 2017-2019\\
\textsc{Developing a Data Carpentry Curriculum for Astronomers and Physicists}\\
P-I: K. A. Bostroem \& Rodolfo Montez\\
\\
\$800, FAMOUS Travel Grant \hfill2018\\ 
\textsc{.Astronomy Conference}\\
\\
25.5 hours, Gemini Observatory \hfill 2018 - 2019\\
\textsc{Progenitors of SNe type II from nebular spectra}\\
P-I: K. Azalee Bostroem; \\
\\
14 half nights, Keck Observatory \hfill 2017 - 2019\\
\textsc{The Global Supernova Project}\\
P-I: Stefano Valenti; \\
\\
??? {\it Las Cumbres Observatory} \hfill 2018 - 2019???\\
\textsc{The Global Supernova Project}\\
P-I: Stefano Valenti; \\
\\
\$30,000, Swift General Observer Program \hfill 2016-2018 \\
\textsc{Early Spectroscopy of Supernovae with Swift and Floyds}\\
P-I: Stefano Valenti \\
\\
25 orbits, Hubble Space Telescope General Observer Program \hfill 2015, 2017\\ %Cycle 22, Cycle 24
\textsc{UV Spectroscopic Signatures from Type Ia Supernovae Strongly Interacting with a Circumstellar Medium}\\
P-I: Ori Fox \\
\\
Nine orbits, Hubble Space Telescope General Observer Program \hfill 2016\\ %
\textsc{Long-Lost Companions: A Search for the Binary Secondaries of Three Nearby Supernovae}\\
P-I: Ori Fox \\
\\
15 orbits, Hubble Space Telescope General Observer Program \hfill \hfill 2015-2016\\ %Cycle 23
\textsc{The optical-UV extinction law in 30 Doradus}\\
P-I: Jesus Maiz-Apellaniz \\
Admin P-I: K. Azalee Bostroem \\
\\
Three orbits, Hubble Space Telescope General Observer Program \hfill 2015\\ %Cycle 22; this gave me summer funding - how so I specify that???
\textsc{Uncovering the Putative B-Star Binary Companion of the SN 1993J Progenitor}\\
P-I: Ori Fox \\
\\
Six orbits, Hubble Space Telescope Calibration Program \hfill 2013-2014\\ %%13527, Cycle 21
\textsc{COS NUV Spectroscopic Sensitivity Monitoring}\\
P-I: K. Azalee Bostroem \\ % J. Taylor, C. R. Proffitt
\\
23 orbits, Hubble Space Telescope Calibration Program \hfill 2013-2014\\ %13520, Cycle 21
\textsc{COS FUV Spectroscopic Sensitivity Monitoring}\\
P-I: K. Azalee Bostroem \\ %J. H. Debes, C. R. Proffitt
\\
12 orbits, Hubble Space Telescope Calibration Program \hfill  2013-2014\\ %13548, Cycle 21
\textsc{MAMA Spectroscopic Sensitivity and Focus Monitor Cycle 21}\\ 
P-I: H. Sana \\ %, R. A. Osten, C. R. Proffitt, {\bf K. A. Bostroem}
\\
Five orbits, Hubble Space Telescope Calibration Program \hfill  2013-2014\\ %13544, Cycle 21
\textsc{STIS/CCD Spectroscopic Sensitivity Monitor for Cycle 21} \\
P-I: H. Sana\\ %, R. A. Osten, {\bf K. A. Bostroem}, C. R. Proffitt
\\
One orbit, Hubble Space Telescope General Observer Program \hfill  2013-2014\\ %13447, Cycle 21
\textsc{The massive monsters living deep in the Tarantula nebula: How massive are they really?} \\
P-I: S. E. de Mink\\ %, S. C. Nieves, {\bf K. A. Bostroem}, P. A. Crowther, et al.
\\
Six orbits, Hubble Space Telescope Calibration Program \hfill 2012-2013\\ %13125, Cycle 20
\textsc{COS NUV Spectroscopic Sensitivity Monitoring}\\
P-I: K. Azalee Bostroem \\ %R. A. Osten, C. R. Proffitt
\\
33 orbits, Hubble Space Telescope Calibration Program \hfill 2012-2013\\ %13119, Cycle 20
\textsc{COS FUV Spectroscopic Sensitivity Monitoring}\\
P-I: K. Azalee Bostroem \\ %R. A. Osten, C. R. Proffitt
\\
12 orbits, Hubble Space Telescope Calibration Program \hfill  2012-2013\\ %13145, Cycle 20 
\textsc{MAMA Spectroscopic Sensitivity and Focus Monitor Cycle 20}\\
P-I: S. T. Holland\\ %, R. A. Osten, C. R. Proffitt, {\bf K. A. Bostroem}
\\
Five orbits, Hubble Space Telescope Calibration Program \hfill  2012-2013\\ %13141, Cycle 20
\textsc{STIS/CCD Spectroscopic Sensitivity Monitor for Cycle 20} \\
P-I: S. T. Holland\\ %, R. A. Osten, {\bf K. A. Bostroem}, C. R. Proffitt
\\
12 orbits, Hubble Space Telescope Calibration Program \hfill 2011-2012\\ %12775, Cycle 19
\textsc{MAMA Spectroscopic Sensitivity and Focus Monitor Cycle 19}\\
P-I: K. Azalee Bostroem \\ % R. A. Osten, C. R. Proffitt. S. T. Holland
\\
Five orbits, Hubble Space Telescope Calibration Program \hfill 2011-2012\\ %12772, Cycle 19
\textsc{STIS/CCD Spectroscopic Sensitivity Monitor for Cycle 19}\\
P-I: K. Azalee Bostroem \\ %. A. Osten, C. R. Proffitt
\\
Eight orbits, Hubble Space Telescope Calibration Program \hfill 2011-2012\\ %12796, Cycle 19, 
\textsc{Second COS FUV Lifetime Position: Focus Sweep Enabling Program (FENA3)} \\
P-I: C. Oliveira \\%, {\bf K. A. Bostroem}
\\
44 orbits, Hubble Space Telescope Calibration Program \hfill 2011-2012\\ %12715, Cycle 19
\textsc{COS FUV Spectroscopic Sensitivity Monitoring} \\
P-I: R. A. Osten\\%, C. D. Keyes, D. J. Sahnow, A. Aloisi, {\bf K. A. Bostroem}
\\
12 orbits, Hubble Space Telescope Calibration Program \hfill 2010-2011\\ %12414, Cycle 18
\textsc{MAMA Spectroscopic Sensitivity and Focus Monitor Cycle 18} \\
P-I: R. A. Osten \\%, C. R. Proffitt, {\bf K. A. Bostroem}
\\
Five orbits, Hubble Space Telescope Calibration Program \hfill 2010-2011\\ %12411, Cycle 18, 
\textsc{STIS/CCD Spectroscopic Sensitivity Monitor for Cycle 18} \\
P-I: R. A. Osten\\%, {\bf K. A. Bostroem}, C. R. Proffitt
%Nine HST Calibration Proposals including with Bostroem as Co-I\\ % How do I include this? 15 HST Calibration Proposals including {\bf 6 with Bostroem as PI}
%
\vspace{-0.1in}   
\sectiontitle{Research\\ Experience} 
%\vspace{0.1in}
Supervisor: Stefano Valenti \hfill 2015-Present\\
Using supernovae to characterize the properties of massive stars with 
observations from the Keck Observatory, the Gemini Observatory, the 
Las Cumbres Observatory, the ePESSTO collaboration, the Global 
Supernova Project, and the DLT40 Survey.\\
\\
Supervisor: Jesus Maiz-Apellaniz \hfill 2015-2018\\
Analyzing long-slit HST/STIS spectra of the 30 Doradus cluster in 
the Large Magellanic Cloud to create a family of UV-optical extinction laws. \\
\\
Supervisor: Michael Wood-Vasey \hfill Fall 2015\\
Correcting telluric lines in spectra from the Kitt Peak Observatory using
precipitable water vapor measurements from nearby GPS stations.\\
\\
Supervisor: Alex Filippenko \& Ori Fox \hfill 2013-2016\\
Searching for the companion star to supernova 1993J using UV and optical 
spectroscopy and imaging from the Hubble Space Telescope.\\
\\
Supervisor: Paul Crowther, Daniel Lennon, \& Nolan Walborn \hfill 2011-2014\\
Characterized the massive stars in the central cluster of 30 Doradus using 
UV, optical, and NIR spectroscopy from the Hubble Space Telescope.\\
\\
Supervisor: Adam Riess \& Steven Rodeny\hfill 2011-2013\\
Searched for high-redshift type Ia supernovae in the CLASH/CANDELS images.\\
\\
Supervisor: Douglas Leonard \hfill 2006-2009 \\
Calibrated the NIR Tully-FIsher Relation with a uniform sample of Cepheid 
variable stars and used it to find Hubble's Constant.\\
\\
Supervisor: Eric Sandquist \hfill Spring 2009\\
Observe and reduce observations from the Mount Laguna Observatory. \\ 
\\
Supervisor: George Smooth \hfill Summer 2005\\
Created content for the Universe Adventure website as part of the 
Pre-Service Teacher Program.\\
\vspace{-0.1in}  
\sectiontitle{Skills}
Working knowledge of UNIX, IDL, Python, PyRAF, SQL, Git/GitHub\\
Experience using cluster computing resources at UC Davis\\ 
%
\vspace{-0.1in}   
\sectiontitle{Observing\\Experience}
Optical Spectroscopy and Imaging ({\it Lick Observatory/KAST, Nickel, ShARCS})\\
Optical Spectroscopy ({\it Keck/LRIS})\\
Optical Imaging ({\it Mount Laguna Observatory/40-inch})
\vspace{-0.1in}  
\sectiontitle{Invited\\Talks}
%\vspace{0.1in}
\textsc{US Participation in the Global Supernova Project} \hfill January 2019\\
Time-Domain Follow-up Observations with Las Cumbres Observatory, American Astronomical Society 233rd Meeting, Seattle, WA \\
\\
\textsc{Code Review: Building a Community to Talk about Coding}\hfill May 2017\\
Python in Astronomy Keynote, Lorentz Center, Leiden, NL\\
\\
\textsc{Beyond Direct Detection: Understanding the Progenitors of Type II Supernovae}\hfill August 2017\\
Northern California Graduate Physics Admissions Bootcamp at UC Davis, Davis, CA\\
\\
\textsc{The Hubble Space Telescope and Me}\hfill August 2016, 2017\\
California State Summer School for Mathematics and Science at UC Davis, Davis, CA\\
\\
\textsc{SN 1993J and the Search for the Type IIb Supernova Progenitor System}\hfill 2014\\
Stellar and Extragalactic Astronomy Lunch at the Goddard Space Flight Center, Greenbelt, MD\\
\vspace{-0.1in}  
\sectiontitle{Mentored Students} 
%\vspace{0.1in}
Isabele Ye \hfill 2018-2019\\
Monitor the data acquisition of Type II SNe observed with the Las Cumbres Observatory 
as part of the Global Supernova Project to ensure the complete light curve is observed.\\
\\
Gayle Zhang \hfill 2017-2018\\
Derived the radius around saturated stars within which supernova detections in 
the DLT40 survey should be considered false positives.\\
\\
\clearpage
Martha Saladino \hfill Summer 2012\\
Refined the flux calibration of the HST/COS spectrography by characterizing  
throughput as a function of time\\
\\
Kenneth Hart \hfill 2011-2012\\
Improved the calibration of the HST/COS and HST/STIS NUV detectors by 
characterizing the vignetted region on each\\
\\
Inna Bojinova \hfill Summer 2011\\
Improved the flux calibration of the HST/COS spectrography by characterizing  
throughput as a function of wavelength\\
%
\vspace{-0.1in}  
\sectiontitle{Refereed\\Publications}
%\vspace{0.1in}
{\bf Bostroem, K. A.}, et al, 2019, in prep\\
\\
%2017eaw Paper
{\bf Bostroem, K. A.}, et al., 2018, submitted\\
\\
The Astropy Collaboration, [29 people], {\bf Bostroem, K. A.} et al., 2018, AJ, 156, 123A\\
\\
Hosseinzadeh, G., [7 people], {\bf Bostroem, K. A.}, et al., 2017, ApJL, 845, L11\\ %August 2017
\\
Zapartas, E., [4 people], {\bf Bostroem, K. A.}, et al., 2017, ApJ, 842, 125\\ %June 2017
\\
Crowther, P. A., Caballero-Nieves, S. M., {\bf Bostroem, K. A.}, et al., 2016, MNRAS, 458, 624\\
\\
Fox, O. D., Smith, [3 people], {\bf Bostroem, K. A.}, et al., 2015, MNRAS, 454, 4366\\
\\
Fox, O. D., {\bf Azalee Bostroem, K.}, Van Dyk, S. D., et al., 2014, ApJ, 790, 17\\
\\
Rodney, S. A., [30 people], {\bf Bostroem, K. A.}, et al., 2014, AJ, 148, 13\\
\\
Astropy Collaboration, [29 people], {\bf Bostroem, K. A.}, et al., 2013, A\&A, 558, A33 \\
%
\vspace{-0.1in}   
\sectiontitle{Posters\\\& Technical\\Reports}
{\bf Bostroem, K. A.}, et al., Do Type IIP/IIL Supernovae Experience Interaction?, 2018\\
\\
{\bf Bostroem, K. A.}, et al., Spectral Types and Wind Velocities for Massive Stars in R136, 2014\\
\\
{\bf Bostroem, K. A.}, et al., New HST/STIS Spectroscopy of Massive Members of R136 in 30 Doradus, 2013\\
\\
{\bf Bostroem, K. A.}, et al., Long-Slit Spectroscopy for 30 Doradus, 2013\\
\\
{\bf Bostroem, K. A.}, Aloisi, A., Bohlin, R., Hodge, P., \& Proffitt, C. 2012, Tech. rep \\%Post-SM4 Sensitivity Calibration of the STIS Echelle Modes
\\
\textbf{Bostroem, K. A.}, et al., 2010, HST Calibration Workshop Proceedings%Post - SM4 Flux Calibration of the STIS Echelle Modes
\\
{\bf Bostroem, K. A.}, \& Proffitt, C. 2011, STIS Data Handbook v. 6.0\\
\\
Sonnentrucker, P., Becker, G., {\bf Bostroem, K. A.}, et al., Summary of the COS Cycle 22 Calibration Program, 2016, Tech. rep \\ %ISR
\\
Sana, H., [3 people], {\bf Bostroem, K. A.}, et al., Summary of the COS Cycle 21 Calibration Program, 2015, Tech. rep \\ %ISR
\\
Proffitt, C. R., {\bf Bostroem, K. A.}, et al., Changes to the COS Extraction Algorithm for Lifetime Position 3, 2015, Tech. rep \\ %ISR
\\
Roman-Duval, J., Aloisi, A., {\bf Bostroem, K. A.}, et al., Summary of the COS Cycle 20 Calibration Program, 2015, Tech. rep \\ %ISR
\\
Holland, S. T., Aloisi, A., {\bf Bostroem, K. A.}, et al., The Time-Dependent Sensitivity of the MAMA and CCD Long-Slit Gratings, 2014, Tech. rep \\ %ISR
\\
Roman-Duval, J.,[10 people] {\bf Bostroem, K. A.}, et al., Summary of the STIS Cycle 19 Calibration Program, 2014, Tech. rep \\ %ISR
\\
Roman-Duval, J. [4 people], {\bf Bostroem, K. A.}, et al., Summary of the Cycle 19 COS Calibration Program, 2014, Tech. rep \\ %ISR
\\
Massa, D., [4 people], \textbf{Bostroem, K. A.}, et al., Updated Absolute Flux Calibration of the COS FUV Modes, 2014, Tech. rep. \\ %ISR
\\
Kriss, G. A., [2 people], \textbf{Bostroem, K. A.}, et al., Summary of STIS Cycle 18 Calibration Program, 2013, Tech. rep. \\ %ISR
\\
Osten, R. A., Aloisi, A., \textbf{Bostroem, K. A.}, et al., Summary of Results from the First Move to a New COS FUV Lifetime Position, 2013, Tech. rep. \\ %ISR
\\
Sahnow, D. J., Aloisi, A., \textbf{Bostroem, K. A.}, et al., Characterization, modeling, and management of the COS FUV detector lifetime, 2013, Proc. of the SPIE, 8859 \\
\\
Kriss, G. A., [2 people], \textbf{Bostroem, K. A.}, et al., Summary of the COS Cycle 18 Calibration Program, 2013, Tech. rep. \\ %ISR
\\
Oliveira, C., \textbf{Bostroem, K. A.}, Osterman, S., Second COS FUV Lifetime Position Results from the Focus Sweep Enabling Program, FENA3 (12796), 2013, Tech. rep. \\ %ISR
\\
Osten, R. A., [3 people], \textbf{Bostroem, K. A.}, et al. Summary of the COS Cycle 17 Calibration Program, 2012, Tech. rep. \\ %ISR
\\
Sahnow, D. J., Aloisi, A., \textbf{Bostroem, K. A.}, et al., The COS FUV channel: on-orbit performance trends and early characterization of a new detector lifetime position, 2012, Proc. of the SPIE, 8443 \\
\\
Wolfe, M. A., [4 people], \textbf{Bostroem, K. A.}, et al., Summary of the STIS Cycle 17 Calibration Program, 2012, Tech. rep. \\%ISR
\\
Sahnow, D. J., [6 people], \textbf{Bostroem, K. A.}, et al. Gain sag in the FUV detector of the Cosmic Origins Spectrograph, 2011,  Proc. of the SPIE, 8145 \\ %ISR
\\
Osten, R. A., Massa, D., \textbf{Bostroem, K. A.}, et al., Updated Results from the COS Spectroscopic Sensitivity Monitoring Program, 2011, Tech. rep. \\ %ISR
\\
Zheng, W., [7 people], \textbf{Bostroem, K. A.}, et al., Trend of Dark Rates of the COS and STIS NUV MAMA Detectors, 2010, HST Calibration Workshop Proceedings \\
\\
Aloisi, A., Ake, T., \textbf{Bostroem, K. A.}, et al., The On-Orbit Performance of the Cosmic Origins Spectrograph, 2010, HST Calibration Workshop Proceedings \\
%16 technical reports with {\bf Bostroem} as co-author; see \color{blue}\url{http://bit.ly/2iUeoax }\color{black}\hspace*{0pt} for details
\vspace{-0.1in}   
\sectiontitle{Conference\\Talks}
Bostroem, K. A., et al., Signs of Circumstellar Interaction in Type IIL Supernovae, American Astronomical Society 233rd Meeting,  2019\\
\\
Bostroem, K. A., Post-SM4 Sensitivity Calibration of the STIS Echelle Modes, STScI TIPS/JIM Monthly Meeting, March 2012 \\
\vspace{-0.1in}   
\sectiontitle{Software}
Bradley, L., [7 people], {\bf Bostroem,  K. A.}, et al., Photutils: Photometry tools, 2016, ASCL \\
%
\vspace{-0.1in}   
\sectiontitle{Instrument\\Support} 

\textsc{HST/COS and HST/STIS Calibration Pipeline Lead}\\
As calibration pipeline lead, I organized and lead bi-weekly meetings, supervised the testing of the HST/COS and HST/STIS calibration pipelines, and coordinated the HST/COS and HST/STIS calibration pipeline work with the HST/COS and HST/STIS teams and the archive and pipeline developers. I also prioritized HST/COS and HST/STIS pipeline development items and oversaw the implementation of improvements to the HST/COS and HST/STIS calibration pipeline.\\
\\
\textsc{Instrument Monitoring and Calibration}\\
As a member of the HST/STIS team I created and tested HST/STIS CCD Dark and Bias reference files. I also characterized the flux calibration and blaze shift correction for the HST/STIS echelle modes following Servicing Mission 4 and developed tools to monitor the HST/STIS time-dependent sensitivity. I continued this work on the HST/COS team developing tools to monitor the HST/COS FUV and NUV time-dependent sensitivity and create and test new reference files as needed. Additionally, I have created and tested throughput reference files for HST/COS and HST/STIS to be used in the Exposure Time Calculator.\\
\\
\textsc{User Support}\\
As user support deputy for the HST/COS and HST/STIS teams, I maintained the internal and external web pages, created and delivered Space Telescope Analysis Newsletters, and tracked user support issues within the team. I also support the Spectrographs? Help Desk, answering questions from the community about HST/COS, HST/STIS, HST/GHRS, and HST/FOS.\\
%HST/COS and HST/STIS Calibration Pipeline Lead, Community Support Deputy, and instrument monitoring and calibration
%
\vspace{-0.1in}   
\sectiontitle{Employment}
    \textsc{Graduate Student Researcher}\hfill2016-present \\
    University of California, Davis, CA\\
    \\
    \textsc{Supernova Spectroscopy Analyst}\hfill Fall 2015 \\
    University of Pittsburgh, Pittsburgh, PA\\
    \\
    \textsc{Research and Instrument Analyst}\hfill 2009 - 2014 \\
     Space Telescope Science Institute, Baltimore, MD\\
%
\vspace{-0.1in}   
\sectiontitle{Teaching\\Experience}
\textsc{Lead Instructor, Software Carpentry Workshops}\hfill 2012-present\\
Lead two day workshops at institutions and conferences internationally to enable scientists to work more efficiently and reproducibly by improving their coding skills. Workshops include Python, Git, Unix, test driven development, object oriented and functional programming.
\begin{itemize}
\item []American Astronomical Society 233rd Meeting, 2019, Seattle, WA 
\item []American Astronomical Society 231st Meeting, 2018, National Harbor,  MD 
\item []American Astronomical Society 229th Meeting, 2017, Grapevine, TX 
\item []American Astronomical Society 227th Meeting, 2016, Kissammee, FL 
\item []Women in Science and Engineering, 2015, Davis, CA 
\item []American Astronomical Society 225th Meeting, 2015, Seattle, WA 
\item []Stanford University, 2014, Palo Alto, CA 
\item []Women in Science and Engineering, 2014, Davis, CA 
\item []European Space Astronomy Centre, 2014, Cebreros, Avila, Spain 
\item []Space Telescope Science Institute, 2013, Baltimore, MD
\item []University of California, Davis, 2013, Davis, CA
\item []Lawrence Berkeley National Laboratory, 2013, Berkeley, CA
\item []University of Maryland, 2013, Baltimore, MD
\item []University of Edinburgh, 2012, Edinburgh, Scotland, UK
\item []George Mason University, 2012, Fairfax, VA
\end{itemize}
\textsc{Teaching Assistant, University of California} (PHYS 7A, 7B) \hfill 2014-2018\\
Guided students through lab work to understand thermodynamics, mechanics, electrical circuits, and fluid dynamics with lecture, small group work, and whole class discussions.\\
%TODO: Add course evaluations
\\
\textsc{Lead Teaching Assistant, San Diego State University}\hfill Fall 2008\\
Supervise astronomy lab instructors and prepare them to teach each week. \\
\\
\textsc{Teaching Assistant, San Diego State University}\hfill 2006-2009\\
Enabled a better understanding of general astronomy through hand on applications of topics covered in lecture. In addition to prepared labs, developed lesson plans and curriculum.\\
\\
\textsc{Mathematics Teacher, Ross Valley Summer School}\hfill Summer 2006\\
Designed the curriculum for, created lesson plans for, and taught first through sixth grade mathematics.\\
\\
\textsc{Mathematics Student Teacher, Poughkeepsie High School}\hfill Fall 2006\\
Created lesson plans and taught a full course load (five periods) of high school mathematics.\\
\\
\textsc{Science Teacher and Curriculum Designer, Crossroads Summer Camp}\hfill Summer 2003, 2004 \\
Created a curriculum and lesson plans for and taught sixth, seventh, and eighth grade science courses.\\
\sectiontitle{Service \& Outreach}
\textsc{Code Review Leader, UC Davis}\hfill 2015-Present\\
Founded, organize, and lead weekly meetings attended by graduate students, post-docs, research staff, and faculty to improve and share coding knowledge.\\
\\
\textsc{Diversity and Inclusion in Physics Group Member and Co-Leader}\hfill 2014-Present\\
Bring discussion, programs, and activities to the department to improve the departmental culture surrounding diversity, equity, and inclusion.\\
\\
\textsc{Python in Astronomy SOC member} \hfill 2017-2018\\
Organized and coordinated the Python in Astronomy 2018 conference.\\
\\
\textsc{Python Lesson Maintainer, Software Carpentry Lessons}\hfill 2014-2016\\
Provide feedback on and incorporate suggested improvements to the Python lesson using git and GitHub.\\
\\
\textsc{Scipy Conference Proceedings Editor}\hfill 2015\\
\\
\textsc{Project Astro Astronomer, San Diego, CA}\hfill 2007-2008\\
Brought hands on astronomy projects to a third grade class\\
\\
\end{llist}
\end{document}






























