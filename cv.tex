\documentclass[10pt]{cv}
\font\cap=cmcsc10
\usepackage{amsmath}

\topmargin 0pt
\headheight 0pt
\headsep 0pt
\textheight 674pt
\pagestyle{empty}
\parindent 0.08in
\parskip \baselineskip
\topmargin 0in
\footskip 1in
%\oddsidemargin 0in
\oddsidemargin -0.25in
%\evensidemargin 0in
\evensidemargin -0.25in

\textwidth 6.9in

\setlength{\topmargin}{-0.25in}
\setlength{\headheight}{0in}
\setlength{\headsep}{0in}
\setlength{\topskip}{0.75in}
\setlength{\textheight}{9in}
\pagestyle{myheadings}
\setlength{\parskip}{0in}

%set the number in the second set of curly braces to the number you want displayed on
%the first page
%setcounter{page}{2}
  

\usepackage{bibentry}
\usepackage{url}
\usepackage{color}
\usepackage{aas_macros}
\usepackage{hanging}
\usepackage{natbib}
\bibliographystyle{apj}
\newcommand\hangbibentry[1]{%
    \smallskip\par\hangpara{1em}{1}\bibentry{#1}\smallskip\par %{indent}{afterline}
}

%\usepackage{helvetica} % uses helvetica postscript font (download helvetica.sty)
%\usepackage{newcent}   % uses new century schoolbook postscript font  
\setlength{\topmargin}{-0.6in}  % Start text higher on the page 
\setlength{\textheight}{9.5in}  % increase textheight to fit more on a page
\setlength{\headsep}{0.2in}     % space between header and text
\setlength{\headheight}{12pt}   % make room for header
\usepackage{fancyhdr}  % use fancyhdr package to get 2-line header
\renewcommand{\headrulewidth}{0pt} % suppress line drawn by default by fancyhdr
%\lhead{\hspace*{-\sectionwidth}Justin Ely} % force lhead all the way left
%\rhead{Page \thepage}  % put page number at right
\cfoot{}  % the footer is empty
%\pagestyle{fancy} % set pagestyle for the document

\begin{document} 
\nobibliography{refs.bib}

\section{{\LARGE \bf{K. Azalee Bostroem}}}
{\rule{\linewidth}{0.5mm}} \\
\begin{minipage}{0.55\textwidth}
Department of Physics \\
University of California, Davis \\
Davis, CA
\end{minipage}
\begin{minipage}{0.45\textwidth}
\begin{flushright}
\vspace{0.25cm}
\color{blue}\url{kabostroem@ucdavis.edu}\\
\color{blue}\url{https://github.com/abostroem} \\
\color{blue}\url{http://azaleebostroem.wordpress.com}
\end{flushright}
\end{minipage} \\
{\rule{\linewidth}{0.5mm}} 
\vspace{-.5em}
%\begin{resume}
\begin{llist}
\vspace{0.1in} 
\sectiontitle{Education}
%\vspace{0.1in} 
\textsc{University of California, Davis}\hfill Sept 2014-Present\\
Davis, CA, USA \\
Ph.D. in Physics \\
Thesis Advisor: Stefano Valenti \\
\\
\textsc{San Diego State University}\hfill May 2009\\
San Diego, CA, USA \\
M.S. in Astronomy \\
Thesis Advisor: Douglas Leonard \\
\\
\textsc{Vassar College} \hfill May 2006 \\
Poughkeepsie, NY, USA \\
B.A. in Mathematics \\
California State and New York State Secondary Teaching Certifications in Mathematics \\
\vspace{-0.1in}   
\sectiontitle{Honors}
%\vspace{0.1in}
NASA Hubble 25th Anniversary Commendation, HST Science Team \hfill 2016 \\ %	CHECK THIS DATE
Ray and Constance Chandler Fellowship \hfill 2014-2015 \\
NASA Group Achievement Award, HST SM4 Servicing Implementation Team \hfill2010 \\
Cliff E. Smith and Ruth Kinnell Graduate Fellowship \hfill 2007 - 2008\\
\vspace{-0.1in}   
\sectiontitle{Proposals\\\& Grants}
%\vspace{-0.3in}
\$40,000, AIP Venture Partnership Fund \hfill 2017-2019\\
\textsc{Developing a Data Carpentry Curriculum for Astronomers and Physicists}
P-I: K. A. Bostroem \& Rodolfo Montez\\
\\
\$800, FAMOUS Travel Grant \hfill2018\\ 
\textsc{.Astronomy Conference}\\
\\
25.5 hours, Gemini Observatory \hfill 2018 - 2019\\
\textsc{Progenitors of SNe type II from nebular spectra}\\
P-I: K. Azalee Bostroem; \\
\\
14 half nights, Keck Observatory \hfill 2017 - 2019\\
\textsc{The Global Supernova Project}\\
P-I: Stefano Valenti; \\
\\
%??? {\it Las Cumbres Observatory} \hfill 2018 - 2019???\\
%\textsc{The Global Supernova Project}\\
%P-I: Stefano Valenti; \\
%\\
\$30,000, Swift General Observer Program \hfill 2016-2018 \\
\textsc{Early Spectroscopy of Supernovae with Swift and Floyds}\\
P-I: Stefano Valenti \\
\\
25 orbits, Hubble Space Telescope General Observer Program \hfill 2015, 2017\\ %Cycle 22, Cycle 24
\textsc{UV Spectroscopic Signatures from Type Ia Supernovae Strongly Interacting with a Circumstellar Medium}\\
P-I: Ori Fox \\
\\
Nine orbits, Hubble Space Telescope General Observer Program \hfill 2016\\ %
\textsc{Long-Lost Companions: A Search for the Binary Secondaries of Three Nearby Supernovae}\\
P-I: Ori Fox \\
\\
15 orbits, Hubble Space Telescope General Observer Program \hfill \hfill 2015-2016\\ %Cycle 23
\textsc{The optical-UV extinction law in 30 Doradus}\\
P-I: Jesus Maiz-Apellaniz \\
Admin P-I: K. Azalee Bostroem \\
\\
Three orbits, Hubble Space Telescope General Observer Program \hfill 2015\\ %Cycle 22; this gave me summer funding - how so I specify that???
\textsc{Uncovering the Putative B-Star Binary Companion of the SN 1993J Progenitor}\\
P-I: Ori Fox \\
\\
Six orbits, Hubble Space Telescope Calibration Program \hfill 2013-2014\\ %Cycle 21
\textsc{COS NUV Spectroscopic Sensitivity Monitoring}\\
P-I: K. Azalee Bostroem \\
\\
23 orbits, Hubble Space Telescope Calibration Program \hfill 2013-2014\\ %Cycle 21
\textsc{COS FUV Spectroscopic Sensitivity Monitoring}\\
P-I: K. Azalee Bostroem \\
\\
Six orbits, Hubble Space Telescope Calibration Program \hfill 2012-2013\\ %Cycle 20
\textsc{COS NUV Spectroscopic Sensitivity Monitoring}\\
P-I: K. Azalee Bostroem \\
\\
33 orbits, Hubble Space Telescope Calibration Program \hfill 2012-2013\\ %Cycle 20
\textsc{COS FUV Spectroscopic Sensitivity Monitoring}\\
P-I: K. Azalee Bostroem \\
\\
12 orbits, Hubble Space Telescope Calibration Program \hfill 2011-2012\\ %Cycle 19
\textsc{MAMA Spectroscopic Sensitivity and Focus Monitor Cycle 19}\\
P-I: K. Azalee Bostroem \\
\\
Five orbits, Hubble Space Telescope Calibration Program \hfill 2011-2012\\ %Cycle 19
\textsc{STIS/CCD Spectroscopic Sensitivity Monitor for Cycle 19}\\
P-I: K. Azalee Bostroem \\
\\
Nine HST Calibration Proposals including with Bostroem as Co-I\\ % How do I include this? 15 HST Calibration Proposals including {\bf 6 with Bostroem as PI}
%
\vspace{-0.1in}   
\sectiontitle{Research\\ Experience} 
%\vspace{0.1in}
Supervisor: Stefano Valenti \hfill 2015-Present\\
Using supernovae to characterize the properties of massive stars with 
observations from the Keck Observatory, the Gemini Observatory, the 
Las Cumbres Observatory, the ePESSTO collaboration, the Global 
Supernova Project, and the DLT40 Survey.\\
\\
Supervisor: Jesus Maiz-Apellaniz \hfill 2015-2018\\
Analyzing long-slit HST/STIS spectra of the 30 Doradus cluster in 
the Large Magellanic Cloud to create a family of UV-optical extinction laws. \\
\\
Supervisor: Michael Wood-Vasey \hfill Fall 2015\\
Correcting telluric lines in spectra from the Kitt Peak Observatory using
precipitable water vapor measurements from nearby GPS stations.\\
\\
Supervisor: Alex Filippenko \& Ori Fox \hfill 2013-2016\\
Searching for the companion star to supernova 1993J using UV and optical 
spectroscopy and imaging from the Hubble Space Telescope.\\
\\
Supervisor: Paul Crowther, Daniel Lennon, \& Nolan Walborn \hfill 2011-2014\\
Characterized the massive stars in the central cluster of 30 Doradus using 
UV, optical, and NIR spectroscopy from the Hubble Space Telescope.\\
\\
Supervisor: Adam Riess \& Steven Rodeny\hfill 2011-2013\\
Searched for high-redshift type Ia supernovae in the CLASH/CANDELS images.\\
\\
Supervisor: Douglas Leonard \hfill 2006-2009 \\
Calibrated the NIR Tully-FIsher Relation with a uniform sample of Cepheid 
variable stars and used it to find Hubble's Constant.\\
\\
Supervisor: Eric Sandquist \hfill Spring 2009\\
Observe and reduce observations from the Mount Laguna Observatory. \\ 
\\
Supervisor: George Smooth \hfill Summer 2005\\
Created content for the Universe Adventure website as part of the 
Pre-Service Teacher Program.\\
\vspace{-0.1in}  
\sectiontitle{Skills}
Working knowledge of UNIX, IDL, Python, PyRAF, SQL, Git/GitHub\\
Experience using cluster computing resources at UC Davis\\ 
%
\vspace{-0.1in}   
\sectiontitle{Observing\\Experience}
Optical Spectroscopy and Imaging ({\it Lick Observatory/KAST, Nickel, ShARCS})\\
Optical Spectroscopy ({\it Keck/LRIS})\\
Optical Imaging ({\it Mount Laguna Observatory/40-inch})
\vspace{-0.1in}  
\sectiontitle{Invited\\Talks}
%\vspace{0.1in}
\textsc{Code Review: Building a Community to Talk about Coding}\hfill May 2017\
Python in Astronomy Keynote, Lorentz Center, Leiden, NL\\
\\
\textsc{Beyond Direct Detection: Understanding the Progenitors of Type II Supernovae}\hfill August 2017\\
Northern California Graduate Physics Admissions Bootcamp at UC Davis, Davis, CA\\
\\
\textsc{The Hubble Space Telescope and Me}\hfill August 2016, 2017\\
California State Summer School for Mathematics and Science at UC Davis, Davis, CA\\
\\
\textsc{SN 1993J and the Search for the Type IIb Supernova Progenitor System}\hfill 2014\\
Stellar and Extragalactic Astronomy Lunch at the Goddard Space Flight Center, Greenbelt, MD\\
\vspace{-0.1in}  
\sectiontitle{Mentored Students} 
%\vspace{0.1in}
Gayle Zhang \hfill 2017-2018\\
Derived the radius around saturated stars within which supernova detections in 
the DLT40 survey should be considered false positives.\\
\\
Kenneth Hart \hfill Summer 2012\\
Improved the calibration of the HST/COS and HST/STIS NUV detectors by 
characterizing the vignetted region on each\\
\\
\clearpage
Martha Saladino \hfill Summer 2012\\
Refined the flux calibration of the HST/COS spectrography by characterizing  
throughput as a function of time\\
\\
Inna Bojinova \hfill Summer 2013\\
Improved the flux calibration of the HST/COS spectrography by characterizing  
throughput as a function of wavelength\\
%
\vspace{-0.1in}  
\sectiontitle{Refereed\\Publications}
%\vspace{0.1in}
{\bf Bostroem, K. A.}, et al, 2019, in prep\\
{\bf Bostroem, K. A.}, et al., 2018, submitted\\
The Astropy Collaboration, [29 people], {\bf Bostroem, K. A.} et al., 2018, AJ, 156, 123A\\
Hosseinzadeh, G., [7 people], {\bf Bostroem, K. A.}, et al., 2017, ApJL, 845, L11\\
Zapartas, E., [4 people], {\bf Bostroem, K. A.}, et al., 2017, ApJ, 842, 125\\
Crowther, P. A., Caballero-Nieves, S. M., {\bf Bostroem, K. A.}, et al., 2016, MNRAS, 458, 624\\
Fox, O. D., Smith, [3 people], {\bf Bostroem, K. A.}, et al., 2015, MNRAS, 454, 4366\\
Fox, O. D., {\bf Azalee Bostroem, K.}, Van Dyk, S. D., et al., 2014, ApJ, 790, 17\\
Rodney, S. A., [30 people], {\bf Bostroem, K. A.}, et al., 2014, AJ, 148, 13\\
Astropy Collaboration, [29 people], {\bf Bostroem, K. A.}, et al., 2013, A\&A, 558, A33 \\
%
\vspace{-0.1in}   
\sectiontitle{Posters\\\& Technical\\Reports}
{\bf Bostroem, K. A.}, et al., Do Type IIP/IIL Supernovae Experience Interaction?, 2018\\
{\bf Bostroem, K. A.}, et al., Spectral Types and Wind Velocities for Massive Stars in R136, 2014\\
{\bf Bostroem, K. A.}, et al., New HST/STIS Spectroscopy of Massive Members of R136 in 30 Doradus, 2013\\
{\bf Bostroem, K. A.}, et al., Long-Slit Spectroscopy for 30 Doradus, 2013\\
{\bf Bostroem, K. A.}, Aloisi, A., Bohlin, R., Hodge, P., \& Proffitt, C. 2012, Tech. rep \\
{\bf Bostroem, K. A.}, \& Proffitt, C. 2011, STIS Data Handbook v. 6.0\\
16 technical reports with {\bf Bostroem} as co-author; see \color{blue}\url{http://bit.ly/2iUeoax }\color{black}\hspace*{0pt} for details
%
\vspace{-0.1in}   
\sectiontitle{Instrument\\Support} 
HST/COS and HST/STIS Calibration Pipeline Lead, Community Support Deputy, and instrument monitoring and calibration
%
\vspace{-0.1in}   
\sectiontitle{Employment}
    \textsc{Graduate Student Researcher}\hfill2016-present \\
    University of California, Davis, CA\\
    \\
    \textsc{Supernova Spectroscopy Analyst}\hfill Fall 2015 \\
    University of Pittsburgh, Pittsburgh, PA\\
    \\
    \textsc{Research and Instrument Analyst}\hfill 2009 - 2014 \\
     Space Telescope Science Institute, Baltimore, MD\\
%
\vspace{-0.1in}   
\sectiontitle{Teaching\\Experience}
\textsc{Lead Instructor, Software Carpentry Workshops}\hfill 2012-present\\
Lead two day workshops at institutions and conferences internationally to enable scientists to work more efficiently and reproducibly by improving their coding skills. Workshops include Python, Git, Unix, test driven development, object oriented and functional programming.\\
\\
\textsc{Teaching Assistant, University of California} (PHYS 7A, 7B) \hfill 2014-2018\\
Guided students through lab work to understand thermodynamics, mechanics, electrical circuits, and fluid dynamics with lecture, small group work, and whole class discussions.\\
%TODO: Add course evaluations
\\
\textsc{Lead Astronomy Lab Instructor, San Diego State University}\hfill Fall 2009\\
Supervise astronomy lab instructors and prepare them to teach each week. \\
\\
\textsc{Astronomy Lab Instructor, San Diego State University}\hfill 2006-2009\\
Enabled a better understanding of general astronomy through hand on applications of topics covered in lecture. In addition to prepared labs, developed lesson plans and curriculum.\\
\\
\textsc{Mathematics Teacher, Ross Valley Summer School}\hfill Summer 2006\\
Designed the curriculum for, created lesson plans for, and taught first through sixth grade mathematics.\\
\\
\textsc{Mathematics Student Teacher, Poughkeepsie High School}\hfill Fall 2006\\
Created lesson plans and taught a full course load (five periods) of high school mathematics.\\
\\
\textsc{Science Teacher and Curriculum Designer, Crossroads Summer Camp}\hfill Summer 2003, 2004 \\
Created a curriculum and lesson plans for and taught sixth, seventh, and eighth grade science courses.\\
\sectiontitle{Service \& Outreach}
\textsc{Code Review Leader, UC Davis}\hfill 2015-Present
Founded, organize, and lead weekly meetings attended by graduate students, post-docs, research staff, and faculty to improve and share coding knowledge.\\
\\
\textsc{Diversity and Inclusion in Physics Group Member and Co-Leader}\hfill 2014-Present\\
Bring discussion, programs, and activities to the department to improve the departmental culture surrounding diversity, equity, and inclusion.\\
\\
\textsc{Python in Astronomy SOC member} \hfill 2017-2018\\
Organized and coordinated the Python in Astronomy 2018 conference.\\
\\
\textsc{Python Lesson Maintainer, Software Carpentry Lessons}\hfill 2014-2016\\
Provide feedback on and incorporate suggested improvements to the Python lesson using git and GitHub.\\
\\
\textsc{Scipy Conference Proceedings Editor}\hfill 2015\\
\\
\textsc{Project Astro Astronomer, San Diego, CA}\hfill 2007-2008\\
Brought hands on astronomy projects to a third grade class\\
\\
\end{llist}
\end{document}






























