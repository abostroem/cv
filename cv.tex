\documentclass[10pt]{cv}
\font\cap=cmcsc10
\usepackage{amsmath}

\topmargin 0pt
\headheight 0pt
\headsep 0pt
\textheight 674pt
\pagestyle{empty}
\parindent 0.08in
\parskip \baselineskip
\topmargin 0in
\footskip 1in
%\oddsidemargin 0in
\oddsidemargin -0.25in
%\evensidemargin 0in
\evensidemargin -0.25in

\textwidth 6.9in

\setlength{\topmargin}{-0.25in}
\setlength{\headheight}{0in}
\setlength{\headsep}{0in}
\setlength{\topskip}{0.75in}
\setlength{\textheight}{9in}
\pagestyle{myheadings}
\setlength{\parskip}{0in}

%set the number in the second set of curly braces to the number you want displayed on
%the first page
%setcounter{page}{2}
  
\usepackage{hyperref}
\usepackage{bibentry}
\usepackage{url}
\usepackage{color}
\usepackage{aas_macros}
\usepackage{hanging}
\usepackage{natbib}
\bibliographystyle{apj}
\newcommand\hangbibentry[1]{%
    \smallskip\par\hangpara{1em}{1}\bibentry{#1}\smallskip\par %{indent}{afterline}
}

%\usepackage{helvetica} % uses helvetica postscript font (download helvetica.sty)
%\usepackage{newcent}   % uses new century schoolbook postscript font  
\setlength{\topmargin}{-0.6in}  % Start text higher on the page 
\setlength{\textheight}{9.5in}  % increase textheight to fit more on a page
\setlength{\headsep}{0.2in}     % space between header and text
\setlength{\headheight}{12pt}   % make room for header
\usepackage{fancyhdr}  % use fancyhdr package to get 2-line header
\renewcommand{\headrulewidth}{0pt} % suppress line drawn by default by fancyhdr
%\lhead{\hspace*{-\sectionwidth}Justin Ely} % force lhead all the way left
%\rhead{Page \thepage}  % put page number at right
\cfoot{}  % the footer is empty
%\pagestyle{fancy} % set pagestyle for the document

\begin{document} 
\section{{\LARGE \bf{K. Azalee Bostroem}}}
{\rule{\linewidth}{0.5mm}} \\
\begin{minipage}{0.55\textwidth}
Department of Physics \\
University of California, Davis \\
Davis, CA
\end{minipage}
\begin{minipage}{0.45\textwidth}
\begin{flushright}
\vspace{0.25cm}
\color{blue}\url{kabostroem@ucdavis.edu}\\
\color{blue}\url{https://github.com/abostroem} \\
\color{blue}\url{https://abostroem.wixsite.com/home}
\end{flushright}
\end{minipage} \\
{\rule{\linewidth}{0.5mm}} 
\vspace{-.5em}
%\begin{resume}
\begin{llist}
%-------------------------------------------------------------
%-----------------EMPLOYMENT-----------------------
%-------------------------------------------------------------
\vspace{-0.1in}   
\sectiontitle{Employment}
    \textsc{Graduate Student Researcher}\hfill2015-Present \\
    University of California, Davis, CA\\
\\
\textsc{Supernova Spectroscopy Analyst}\hfill Fall 2015\\ 
University of Pittsburgh, Pittsburgh, PA\\
\\
\textsc{Research and Instrument Analyst}\hfill 2009 - 2014 \\
Space Telescope Science Institute, Baltimore, MD\\
%    \textsc{Graduate Student Researcher}\hfill2016-present \\
%    University of California, Davis, CA\\
%    \\

%-------------------------------------------------------------
%--------------------EDUCATION------------------------
%-------------------------------------------------------------
\vspace{-0.1in}   
\sectiontitle{Education}
\textsc{University of California, Davis}\hfill 2014 - Present\\
Davis, CA, USA \\
Ph.D. in Physics \\
Thesis Advisor: Stefano Valenti \\
\\
\textsc{San Diego State University}\hfill 2006 - 2009\\
San Diego, CA, USA \\
M.S. in Astronomy \\
Thesis Advisor: Douglas Leonard \\
\\
\textsc{Vassar College} \hfill 2002 - 2006 \\
Poughkeepsie, NY, USA \\
B.A. in Mathematics \\
California State and New York State Secondary Teaching Certifications in Mathematics \\
%-------------------------------------------------------------
%--------------------HONORS----------------------------
%-------------------------------------------------------------
\vspace{-0.1in}   
\sectiontitle{Honors}
NASA Hubble 25th Anniversary Commendation, HST Science Team \hfill 2016 \\ %	CHECK THIS DATE
Ray and Constance Chandler Fellowship \hfill 2014 - 2015 \\ %$5000?
NASA Group Achievement Award, HST SM4 Servicing Implementation Team \hfill2010 \\
Cliff E. Smith and Ruth Kinnell Graduate Fellowship \hfill 2007 - 2008\\ %$2000
%-------------------------------------------------------------
%----------RESEARCH EXPERIENCE--------------
%-------------------------------------------------------------
\vspace{-0.1in}   
\sectiontitle{Research\\ Interests}
Core collapse supernovae and their progenitor systems\\
Transient astronomy\\
Big data management in astronomy\\
Development and support of astronomical pipelines and archives\\ 
\vspace{-0.1in}   
\sectiontitle{Research\\ Experience} 
%\vspace{0.1in}
Supervisor: Stefano Valenti \hfill 2015-Present\\
Characterizing the properties of massive stars with observations of supernovae from the Swift Observatory, Very Large Array, Keck Observatory, the Gemini Observatory, the Las Cumbres Observatory, the ePESSTO collaboration, the Global Supernova Project, and the DLT40 Survey.\\
\\
Supervisor: Jesus Maiz-Apellaniz \hfill 2015-2018\\
Analyzed long-slit HST/STIS spectra of the 30 Doradus cluster in 
the Large Magellanic Cloud to create a family of UV-optical extinction laws. \\
\\
Supervisor: Michael Wood-Vasey \hfill Fall 2015\\
Corrected telluric lines in spectra from the Kitt Peak Observatory using
precipitable water vapor measurements from nearby GPS stations.\\
\\
Supervisors: Alex Filippenko \& Ori Fox \hfill 2013-2016\\
Searching for the companion star to supernova 1993J using UV and optical 
spectroscopy and imaging from the Hubble Space Telescope.\\
\\
Supervisors: Paul Crowther, Daniel Lennon, \& Nolan Walborn \hfill 2011-2014\\
Characterized the massive stars in the central cluster of 30 Doradus using 
UV, optical, and NIR spectroscopy from the Hubble Space Telescope.\\
\\
Supervisors: Adam Riess \& Steven Rodeny\hfill 2011-2013\\
Searched for high-redshift type Ia supernovae in the CLASH/CANDELS images.\\
\\
Supervisor: Douglas Leonard \hfill 2006-2009 \\
Calibrated the NIR Tully-Fisher Relation with a uniform sample of Cepheid variable stars and used it to find Hubble's Constant.\\
\\
Supervisor: Eric Sandquist \hfill Spring 2009\\
Observed and reduced observations of young open clusters from the Mount Laguna Observatory. \\ 
\\
Supervisor: George Smoot \hfill Summer 2005\\
Created content for the Universe Adventure website as part of the 
Pre-Service Teacher Program.\\
%-------------------------------------------------------------
%--------------------Proposals & Grants---------------
%-------------------------------------------------------------
\vspace{-0.1in}   
\sectiontitle{Proposals\\\& Grants}
\$39,200, Swift General Observer Program \hfill 2019\\
Co-I: \textsc{High Cadence UV Light Curves of Extremely Young Supernova}\\
P-I: David J. Sand\\
\\
90 nights, Las Cumbres Observatory \hfill 2017 - 2019\\
Co-I: \textsc{The Global Supernova Project}\\
P-I: Stefano Valenti \\
\\
\$40,000, AIP Venture Partnership Fund \hfill 2017-2019\\
\textsc{Developing a Data Carpentry Curriculum for Astronomers and Physicists}\\
{\bf P-I: K. A. Bostroem} \& Rodolfo Montez\\
\\
\$800, FAMOUS Travel Grant \hfill2018\\ 
\textsc{.Astronomy Conference}\\
\\
25.5 hours, Gemini Observatory \hfill 2018 - 2019\\
\textsc{Progenitors of SNe type II from nebular spectra}\\
{\bf P-I: K. Azalee Bostroem} \\
\\
14 half nights, Keck Observatory \hfill 2017 - 2019\\
Co-I: \textsc{The Global Supernova Project}\\
P-I: Stefano Valenti \\
%\\
%\$30,000, Swift General Observer Program \hfill 2016-2018 \\
%Co-I\textsc{Early Spectroscopy of Supernovae with Swift and Floyds}\\
%P-I: Stefano Valenti \\
\\
25 orbits, Hubble Space Telescope General Observer Program \hfill 2015, 2017\\ %Cycle 22, Cycle 24
Co-I: \textsc{UV Spectroscopic Signatures from Type Ia Supernovae Strongly Interacting with a Circumstellar Medium}\\
P-I: Ori Fox \\
\\

Nine orbits, Hubble Space Telescope General Observer Program \hfill 2016\\ %
Co-I: \textsc{Long-Lost Companions: A Search for the Binary Secondaries of Three Nearby Supernovae}\\
P-I: Ori Fox \\
\\
15 orbits, Hubble Space Telescope General Observer Program \hfill 2015-2016\\ %Cycle 23
\textsc{The optical-UV extinction law in 30 Doradus}\\
P-I: Jesus Maiz-Apellaniz \\
{\bf Admin P-I: K. Azalee Bostroem} \\
\\
Summer funding, Three orbits, Hubble Space Telescope General Observer Program \hfill 2015\\ %Cycle 22; 
Co-I: \textsc{Uncovering the Putative B-Star Binary Companion of the SN 1993J Progenitor}\\
P-I: Ori Fox \\
\\
Six orbits, Hubble Space Telescope Calibration Program \hfill 2013-2014\\ %%13527, Cycle 21
\textsc{COS NUV Spectroscopic Sensitivity Monitoring}\\
{\bf P-I: K. Azalee Bostroem} \\ % J. Taylor, C. R. Proffitt
\\
23 orbits, Hubble Space Telescope Calibration Program \hfill 2013-2014\\ %13520, Cycle 21
\textsc{COS FUV Spectroscopic Sensitivity Monitoring}\\
{\bf P-I: K. Azalee Bostroem} \\ %J. H. Debes, C. R. Proffitt
\\
12 orbits, Hubble Space Telescope Calibration Program \hfill  2013-2014\\ %13548, Cycle 21
Co-I: \textsc{MAMA Spectroscopic Sensitivity and Focus Monitor Cycle 21}\\ 
P-I: H. Sana \\ %, R. A. Osten, C. R. Proffitt, {\bf K. A. Bostroem}
\\
Five orbits, Hubble Space Telescope Calibration Program \hfill  2013-2014\\ %13544, Cycle 21
Co-I: \textsc{STIS/CCD Spectroscopic Sensitivity Monitor for Cycle 21} \\
P-I: H. Sana\\ %, R. A. Osten, {\bf K. A. Bostroem}, C. R. Proffitt
\\
One orbit, Hubble Space Telescope General Observer Program \hfill  2013-2014\\ %13447, Cycle 21
Co-I: \textsc{The massive monsters living deep in the Tarantula nebula: How massive are they really?} \\
P-I: S. E. de Mink\\ %, S. C. Nieves, {\bf K. A. Bostroem}, P. A. Crowther, et al.
\\
Six orbits, Hubble Space Telescope Calibration Program \hfill 2012-2013\\ %13125, Cycle 20
\textsc{COS NUV Spectroscopic Sensitivity Monitoring}\\
{\bf P-I: K. Azalee Bostroem} \\ %R. A. Osten, C. R. Proffitt
\\
33 orbits, Hubble Space Telescope Calibration Program \hfill 2012-2013\\ %13119, Cycle 20
\textsc{COS FUV Spectroscopic Sensitivity Monitoring}\\
{\bf P-I: K. Azalee Bostroem} \\ %R. A. Osten, C. R. Proffitt
\\
12 orbits, Hubble Space Telescope Calibration Program \hfill  2012-2013\\ %13145, Cycle 20 
Co-I: \textsc{MAMA Spectroscopic Sensitivity and Focus Monitor Cycle 20}\\
P-I: S. T. Holland\\ %, R. A. Osten, C. R. Proffitt, {\bf K. A. Bostroem}
\\
Five orbits, Hubble Space Telescope Calibration Program \hfill  2012-2013\\ %13141, Cycle 20
Co-I: \textsc{STIS/CCD Spectroscopic Sensitivity Monitor for Cycle 20} \\
P-I: S. T. Holland\\ %, R. A. Osten, {\bf K. A. Bostroem}, C. R. Proffitt
\\
12 orbits, Hubble Space Telescope Calibration Program \hfill 2011-2012\\ %12775, Cycle 19
\textsc{MAMA Spectroscopic Sensitivity and Focus Monitor Cycle 19}\\
{\bf P-I: K. Azalee Bostroem} \\ % R. A. Osten, C. R. Proffitt. S. T. Holland
\\
Five orbits, Hubble Space Telescope Calibration Program \hfill 2011-2012\\ %12772, Cycle 19
\textsc{STIS/CCD Spectroscopic Sensitivity Monitor for Cycle 19}\\
{\bf P-I: K. Azalee Bostroem} \\ %. A. Osten, C. R. Proffitt
\\
Eight orbits, Hubble Space Telescope Calibration Program \hfill 2011-2012\\ %12796, Cycle 19, 
Co-I: \textsc{Second COS FUV Lifetime Position: Focus Sweep Enabling Program (FENA3)} \\
P-I: C. Oliveira \\%, {\bf K. A. Bostroem}
\\
44 orbits, Hubble Space Telescope Calibration Program \hfill 2011-2012\\ %12715, Cycle 19
Co-I: \textsc{COS FUV Spectroscopic Sensitivity Monitoring} \\
P-I: R. A. Osten\\%, C. D. Keyes, D. J. Sahnow, A. Aloisi, {\bf K. A. Bostroem}
\\
12 orbits, Hubble Space Telescope Calibration Program \hfill 2010-2011\\ %12414, Cycle 18
Co-I: \textsc{MAMA Spectroscopic Sensitivity and Focus Monitor Cycle 18} \\
P-I: R. A. Osten \\%, C. R. Proffitt, {\bf K. A. Bostroem}
\\
Five orbits, Hubble Space Telescope Calibration Program \hfill 2010-2011\\ %12411, Cycle 18, 
Co-I: \textsc{STIS/CCD Spectroscopic Sensitivity Monitor for Cycle 18} \\
P-I: R. A. Osten\\%, {\bf K. A. Bostroem}, C. R. Proffitt
%Nine HST Calibration Proposals including with Bostroem as Co-I\\ % How do I include this? 15 HST Calibration Proposals including {\bf 6 with Bostroem as PI}
%-------------------------------------------------------------
%----------------TALKS-----------------------
%-------------------------------------------------------------
\vspace{-0.1in}  
\sectiontitle{Talks}
\textsc{US Participation in the Global Supernova Project} (invited) \hfill January 2019\\
Time-Domain Follow-up Observations with Las Cumbres Observatory, American Astronomical Society 233rd Meeting, Seattle, WA \\
\\
\textsc{Signs of Circumstellar Interaction in Type IIL Supernovae} \hfill January 2019\\
American Astronomical Society 233rd Meeting,  Seattle, WA\\
\\
\textsc{Code Review: Building a Community to Talk about Coding} (invited)\hfill May 2017\\
Keynote; Python in Astronomy, Lorentz Center, Leiden, NL\\
\\
\textsc{Beyond Direct Detection: Understanding the Progenitors of Type II Supernovae} (invited)\hfill August 2017\\
Northern California Graduate Physics Admissions Bootcamp at UC Davis, Davis, CA\\
\\
\textsc{The Hubble Space Telescope and Me} (invited)\hfill August 2016, 2017\\
California State Summer School for Mathematics and Science at UC Davis, Davis, CA\\
\\
\null \hfill June 2014\\
\textsc{SN 1993J and the Search for the Type IIb Supernova Progenitor System} (invited)\\
Stellar and Extragalactic Astronomy Lunch at the Goddard Space Flight Center, Greenbelt, MD\\
\\
\textsc{Post-SM4 Sensitivity Calibration of the STIS Echelle Modes}\hfill March 2012 \\
STScI TIPS/JIM Monthly Meeting, Baltimore, MD\\
%-------------------------------------------------------------
%------------MENTORED STUDENTS---------------
%-------------------------------------------------------------
\vspace{-0.1in}  
\sectiontitle{Mentored \\Students} 
%\vspace{0.1in}
Isabele Ye (w/ Stefano Valenti) \hfill 2018-2019\\
Monitor the data acquisition of Type II SNe observed with the Las Cumbres Observatory as part of the Global Supernova Project to ensure the complete light curve is observed.\\
\\
Gayle Zhang (w/ Stefano Valenti) \hfill 2017-2018\\
Derived the radius around saturated stars within which supernova detections in the DLT40 survey should be considered false positives.\\
\\
Martha Saladino (w/ Justin Ely)\hfill Summer 2012\\
Refined the flux calibration of the HST/COS spectroscopy by characterizing throughput as a function of time.\\
\\
Kenneth Hart \hfill 2011-2012\\
Improved the calibration of the HST/COS and HST/STIS NUV detectors by characterizing the vignetted region on each.\\
\\
Inna Bojinova \hfill Summer 2011\\
Improved the flux calibration of the HST/COS spectroscopy by characterizing throughput as a function of wavelength.\\
%-------------------------------------------------------------
%--------------------SKILLS-------------------------------
%-------------------------------------------------------------
\vspace{-0.1in}  
\sectiontitle{Skills}
Working knowledge of UNIX, IDL, Python, PyRAF, SQL, Git/GitHub, HTML\\
Experience using cluster computing resources at UC Davis\\ 
%-------------------------------------------------------------
%--------------------SOFTWARE------------------------
%-------------------------------------------------------------
%\vspace{-0.1in}   
%\sectiontitle{Software \\Contributions}
%\textsc{Photutils: Photometry tools}, Bradley, L., [7 people], {\bf Bostroem,  K. A.}, et al., 2016, ASCL \\
%
%-------------------------------------------------------------
%-------OBSERVING EXPERIENCE----------------
%-------------------------------------------------------------
\vspace{-0.1in}   
\sectiontitle{Observing\\Experience}
6 nights, Optical Spectroscopy and Imaging ({\it Lick Observatory/KAST, Nickel, ShARCS})\\
14 half nights, Optical Spectroscopy ({\it Keck/LRIS})\\
3 nights, Optical Imaging ({\it Mount Laguna Observatory/40-inch})
%
%-------------------------------------------------------------
%----------INSTRUMENT SUPPORT----------------
%-------------------------------------------------------------
\vspace{-0.1in}   
\sectiontitle{Instrument\\Support} 
\textsc{HST/COS and HST/STIS Calibration Pipeline Lead}\\
Supervised the development and testing of the HST/COS and HST/STIS pipelines coordinating between the scientists, software developers, and archive team. \\
%\begin{itemize}
%\item Organized and lead bi-weekly meetings
%\item Supervised the testing of the HST/COS and HST/STIS calibration pipelines
%\item Coordinated the HST/COS and HST/STIS calibration pipeline work with the HST/COS and HST/STIS teams and the archive and pipeline developers
%\item Prioritized HST/COS and HST/STIS pipeline development items
%\item Oversaw the implementation of improvements to the HST/COS and HST/STIS calibration pipeline
%\end{itemize}
\\
\textsc{Instrument Monitoring and Calibration}\\
Improve the quality of HST/STIS and HST/COS observations through the development of monitoring tools and the development and testing of new calibration reference files.\\
%As a member of the HST/STIS team I created and tested HST/STIS CCD Dark and Bias reference files. I also characterized the flux calibration and blaze shift correction for the HST/STIS echelle modes following Servicing Mission 4 and developed tools to monitor the HST/STIS time-dependent sensitivity. I continued this work on the HST/COS team developing tools to monitor the HST/COS FUV and NUV time-dependent sensitivity and create and test new reference files as needed. Additionally, I have created and tested throughput reference files for HST/COS and HST/STIS to be used in the Exposure Time Calculator.\\
\\
\textsc{User Support Deputy}\\
Maintained the internal and external web pages, created and delivered Space Telescope Analysis Newsletters, answered help desk questions, and tracked user support issues\\
%As user support deputy for the HST/COS and HST/STIS teams, I maintained the internal and external web pages, created and delivered Space Telescope Analysis Newsletters, and tracked user support issues within the team. I also support the Spectrographs? Help Desk, answering questions from the community about HST/COS, HST/STIS, HST/GHRS, and HST/FOS.\\
%HST/COS and HST/STIS Calibration Pipeline Lead, Community Support Deputy, and instrument monitoring and calibration
%-------------------------------------------------------------
%----------REFEREED PUBLICATIONS------------
%-------------------------------------------------------------
\vspace{-0.1in}  
\sectiontitle{Refereed\\Publications}
%\vspace{0.1in}
%\textsc{Discovery and Rapid Follow-up of the type II SN 2018ivc in NGC 1068}, {\bf Bostroem, K. A.}, et al, 2019, in prep\\
%\\
\textsc{The Type II-P Supernova 2017eaw: from Explosion to the Nebular Phase}\\ 
Szalai, T.; Vink\`{o}, J.; K\"{o}nyves-T\`{o}th, R.; Nagy, A.; {\bf Bostroem, K. A.}; [34 people], 2019, ApJ, accepted
\color{blue}\href{https://ui.adsabs.harvard.edu/#abs/2019arXiv190309048S/abstract}{(ADS link)}\color{black}\\
\\
\textsc{Signatures of Circumstellar Interaction in the Type IIL Supernova ASASSN-15oz}\\ 
{\bf Bostroem, K. A.}; Valenti, S.; Horesh, A.; Morozova, V.; Kuin, N. P. M.; Wyatt, S.; Jerkstrand, A.; Sand, D. J.; Lundquist, M.; Smith, M.; Sullivan, M.; Hosseinzadeh, G.; Arcavi, I.;  [18 people], 2019, MNRAS, tmp.562B 
\color{blue}\href{https://ui.adsabs.harvard.edu/#abs/2019MNRAS.tmp..562B/abstract }{(ADS link)}\color{black}\\
\\
\textsc{pwv\_kpno: A Python Package for Modeling the Atmospheric Transmission Function due to Precipitable Water Vapor}\\ 
Perrefort, D.; Wood-Vasey, W. M.; {\bf Bostroem, K. A.}; Gilmore, K.; Joyce, R.; Matheson, T.; Corson, C., 2019, PASP, 131, 996 
\color{blue}\href{https://ui.adsabs.harvard.edu/#abs/2019PASP..131b5002P/abstract}{(ADS link)}\color{black}\\
\\
\clearpage
\textsc{The Astropy Project: Building an Open-science Project and Status of the v2.0 Core Package}\\ 
The Astropy Collaboration; [29 people]; {\bf Bostroem, K. A.}; [108 people], 2018, AJ, 156, 123A
\color{blue}\href{https://ui.adsabs.harvard.edu/#abs/2018AJ....156..123A/abstract}{(ADS link)}\color{black}\\
\\
\textsc{Ultraviolet Detection of the Binary Companion to the Type IIb SN 2001ig}\\ 
Ryder, S. D.; Van Dyk, S. D.; Fox, O. D.; Zapartas, E.; de Mink, S. E.; Smith, N.; Brunsden, E.; {\bf Bostroem, K. A.}; Filippenko, A. V.; Shivvers, I.; Zheng, W., 2018, ApJ, 856, 1
\color{blue}\href{https://ui.adsabs.harvard.edu/#abs/2018ApJ...856...83R/abstract}{(ADS link)}\color{black}\\
\\
\textsc{Early Blue Excess from the Type Ia Supernova 2017cbv and Implications for Its Progenitor}\\ 
Hosseinzadeh, G., Sand, D.J.; Valenti, S.; Brown, P.; Howell, D. A.; McCully, C.; Kasen, D.; Arcavi, I.; {\bf Bostroem, K. A.}, Tartaglia, L.; Hsiao, E. Y.; Davis, S.; Shahbandeh, M.; Stritzinger, M. D., 2017, ApJL, 845, L11
\color{blue}\href{https://ui.adsabs.harvard.edu/#abs/2017ApJ...845L..11H/abstract}{(ADS link)}\color{black}\\ %August 2017
\\
\textsc{Predicting the Presence of Companions for Stripped-envelope Supernovae: The Case of the Broad-lined Type Ic SN 2002ap}\\ 
Zapartas, E., de Mink, S. E.; Van Dyk, S. D.; Fox, O. D.; Smith, N.; {\bf Bostroem, K. A.}; de Koter, A.; Filippenko, A. V.; Izzard, R. G.; Kelly, P. L.; Neijssel, C. J.; Renzo, M.; Ryder, S, 2017, ApJ, 842, 125
\color{blue}\href{https://ui.adsabs.harvard.edu/#abs/2017ApJ...842..125Z/abstract}{(ADS link)}\color{black}\\ %June 2017
\\
\textsc{The R136 star cluster dissected with Hubble Space Telescope/STIS. I. Far-ultraviolet spectroscopic census and the origin of the He II $\rm{\lambda}$1640 in young star clusters}\\ 
Crowther, P. A.; Caballero-Nieves, S. M.; {\bf Bostroem, K. A.}; Ma\`{i}z Apell\`{a}niz, J.; Schneider, F. R. N.; Walborn, N. R.; Angus, C. R.; Brott, I.; Bonanos, A.; de Koter, A.; de Mink, S. E.; Evans, C. J.; Gr\"{a}fener, G.; Herrero, A.; Howarth, I. D.; Langer, N.; Lennon, D. J.; Puls, J.; Sana, H.; Vink, J. S., 2016, MNRAS, 458, 624
\color{blue}\href{https://ui.adsabs.harvard.edu/#abs/2016MNRAS.458..624C/abstract}{(ADS link)}\color{black}\\
\\
\textsc{What powers the 3000-day light curve of SN 2006gy?}\\ 
Fox, O. D.; Smith, N.; Ammons, S. M.; Andrews, J.; {\bf Bostroem, K. A.}; Cenko, S. B.; Clayton, G. C.; Dwek, E.; Filippenko, A. V.; Gallagher, J. S.; Kelly, P. L.; Mauerhan, J. C.; Miller, A. A.; Van Dyk, S. D., 2015, MNRAS, 454, 4366
\color{blue}\href{https://ui.adsabs.harvard.edu/#abs/2015MNRAS.454.4366F/abstract}{(ADS link)}\color{black}\\
\\
\textsc{Uncovering the Putative B-star Binary Companion of the SN 1993J Progenitor}\\
Fox, O. D., {\bf Bostroem, K. A.}, Van Dyk, S. D., Filippenko, A. V.; Fransson, C.; Matheson, T.; Cenko, S. B.; Chandra, P.; Dwarkadas, V.; Li, W.; Parker, A. H.; Smith, N., 2014, ApJ, 790, 17
\color{blue}\href{https://ui.adsabs.harvard.edu/#abs/2014ApJ...790...17F/abstract}{(ADS link)}\color{black}\\
\\
\textsc{Type Ia Supernova Rate Measurements to Redshift 2.5 from CANDELS: Searching for Prompt Explosions in the Early Universe}\\ 
Rodney, S. A.; [30 people]; {\bf Bostroem, K. A.}; [6 people], 2014, AJ, 148, 13
\color{blue}\href{https://ui.adsabs.harvard.edu/#abs/2014AJ....148...13R/abstract}{(ADS link)}\color{black}\\
\\
\textsc{Astropy: A community Python package for astronomy}\\ 
Astropy Collaboration; [29 people]; {\bf Bostroem, K. A.}; [14 people], 2013, A\&A, 558, A33
\color{blue}\href{https://ui.adsabs.harvard.edu/#abs/2013A\%26A...558A..33A/abstract}{(ADS link)}\color{black} \\
%
%-------------------------------------------------------------
%----------TECHNICAL REPORTS-------------------
%-------------------------------------------------------------
\vspace{-0.1in}   
\sectiontitle{Technical\\Reports}
\textsc{Summary of the COS Cycle 22 Calibration Program}\\  
Sonnentrucker; P., Becker, G.; {\bf Bostroem, K. A.}; Debes, J.H.; Ely, J.; Fox, A.; Lockwood, S.; Oliveira, C.; Penton, S.; Proffitt, C.; Roman-Duval, J.; Sahnow, D.; Sana, H.; Taylor, J.; Welty, A. D.; Wheeler, T., 2016, Tech. rep 
\color{blue}\href{https://ui.adsabs.harvard.edu/#abs/2016cos..rept....3S/abstract}{(ADS link)}\color{black}\\ %ISR
%\\
\clearpage
\textsc{Summary of the COS Cycle 21 Calibration Program}\\ 
Sana, H.; Roman-Duval, J.; Ely, J.; {\bf Bostroem, K. A.}; Lockwood, S.; Oliveira, C.; Penton, S.; Proffitt, C.; Sahnow, D.; Sonnentrucker, P.; Welty, A. D.; Wheeler, T., 2015, Tech. rep 
\color{blue}\href{https://ui.adsabs.harvard.edu/#abs/2015cos..rept....6S/abstract}{(ADS link)}\color{black}\\ %ISR
\\
\textsc{Changes to the COS Extraction Algorithm for Lifetime Position 3}\\ 
Proffitt, C. R.; {\bf Bostroem, K. A.}; Ely, J.; Foster, D.; Hernandez, S.; Hodge, P.; Jedrzejewski, R. I.; Lockwood, S. A.; Massa, D.; Peeples, M. S.; Oliveira, C. M.; Penton, S. V.; Plesha, R.; Roman-Duval, J.; Sana, H.; Sahnow, D. J.; Sonnentrucker, P.; Taylor, J. M.,  2015, Tech. rep 
\color{blue}\href{https://ui.adsabs.harvard.edu/#abs/2015cos..rept....3P/abstract}{(ADS link)}\color{black}\\ %ISR
\textsc{Summary of the COS Cycle 20 Calibration Program}\\ 
Roman-Duval, J.; Aloisi, A.; {\bf Bostroem, K. A.}; Ely, J.; Holland, S.; Lockwood, S.; Oliveira, C.; Penton, S.; Proffitt, C.; Sahnow, D.; Sonnentrucker, P.; Welty, A. D.; Wheeler, T., 2015, Tech. rep 
\color{blue}\href{https://ui.adsabs.harvard.edu/#abs/2015cos..rept....2R/abstract}{(ADS link)}\color{black}\\ %ISR
\\
\textsc{The Time-Dependent Sensitivity of the MAMA and CCD Long-Slit Gratings}\\ 
Holland, S. T.; Aloisi, A.; {\bf Bostroem, K. A.}; Oliveria, C.; Proffitt, C.,  2014, Tech. rep 
\color{blue}\href{https://ui.adsabs.harvard.edu/#abs/2014stis.rept....2H/abstract}{(ADS link)}\color{black}\\ %ISR
\\
\textsc{Summary of the STIS Cycle 19 Calibration Program}\\ 
Roman-Duval, J.; Ely, J.; Aloisi, A.; Oliveira, C.; Proffitt, C.; Hernandez, S.; Mason, E.; Sonnetrucker, P.; Wolfe, M.; Long, C.; DiFelice, A.; \textbf{Bostroem, K. A.}; Holland, S.; Lockwood, S.; Cox, C.; Wheeler, T., 2014, Tech. rep 
\color{blue}\href{https://ui.adsabs.harvard.edu/#abs/2014stis.rept....1R/abstract}{(ADS link)}\color{black}\\ %ISR
\\
\textsc{Summary of the Cycle 19 COS Calibration Program}\\ 
Roman-Duval, J.; Ely, J.; Oliveira, C.; Proffitt, C.; Aloisi, A.; \textbf{Bostroem, K. A.}; Cox, C.; Lockwood, S.; Mason, E.; Massa, D.; Osten, R.; Penton, S.; Sahnow, D.; Sonnetrucker, P.; Wheeler, T., 2014, Tech. rep 
\color{blue}\href{https://ui.adsabs.harvard.edu/#abs/2014cos..rept....1R/abstract}{(ADS link)}\color{black}\\ %ISR
\\
\textsc{Updated Absolute Flux Calibration of the COS FUV Modes}\\ 
Massa, D.; Ely, J.; Osten, R.; Penton, S.; Aloisi, A.; \textbf{Bostroem, K. A.}; Roman-Duval, J.; Proffitt, C. 2014, Tech. rep. 
\color{blue}\href{https://ui.adsabs.harvard.edu/#abs/2014cos..rept....9M/abstract}{(ADS link)}\color{black}\\ %ISR
\\
\textsc{Summary of STIS Cycle 18 Calibration Program}\\ 
Kriss, G. A.; Wolfe, M. A.; Aloisi, A.; \textbf{Bostroem, K. A.}; Cox, C.; Dixon, V.; Ely, J.; Long, C.; Mason, E.; Massa, D.; Osten, R.; Proffitt, C.; Roman-Duval, J.; Sonnentrucker, P.; Wheeler, T.; Zheng, W., 2013, Tech. rep. 
\color{blue}\href{https://ui.adsabs.harvard.edu/#abs/2013stis.rept....3K/abstract}{(ADS link)}\color{black}\\ %ISR
\\
\textsc{Summary of Results from the First Move to a New COS FUV Lifetime Position}\\ 
Osten, R. A.; Aloisi, A.; \textbf{Bostroem, K. A.};  Debes, J.; Ely, J.; Hodge, P. E.; Kriss, G.; Massa, D.; Oliveira, C.; Osten, R.; Osterman, S. N.; Penton, S. V.; Proffitt, C.; Roman-Duval, J.; Sonnentrucker, P., 2013, Tech. rep. 
\color{blue}\href{https://ui.adsabs.harvard.edu/#abs/2013cos..rept...16O/abstract}{(ADS link)}\color{black}\\ %ISR
\\
\textsc{Characterization, modeling, and management of the COS FUV detector lifetime}\\ 
Sahnow, D. J., Aloisi, A., \textbf{Bostroem, K. A.}; Debes, J.; Ely, J.; Hodge, P. E.; Kriss, G.; Massa, D.; Oliveira, C.; Osten, R.; Osterman, S. N.; Penton, S. V.; Proffitt, C.; Roman-Duval, J.; Sonnentrucker, P., 2013, Proc. of the SPIE, 8859 
\color{blue}\href{https://ui.adsabs.harvard.edu/#abs/2013SPIE.8859E..0SS/abstract}{(ADS link)}\color{black}\\
\\
\textsc{Summary of the COS Cycle 18 Calibration Program}\\ 
Kriss, G. A.; Wolfe, M.; Aloisi, A.; \textbf{Bostroem, K. A.}; Cox, C.; Ely, J.; Long, C.; Massa, D.; Oliveria, C.; Osten, R.; Proffitt, C.; Sahnow, D.; Wheeler, T.; Zheng, W., 2013, Tech. rep. 
\color{blue}\href{https://ui.adsabs.harvard.edu/#abs/2013cos..rept....4K/abstract}{(ADS link)}\color{black}\\ %ISR
\\
\textsc{Second COS FUV Lifetime Position Results from the Focus Sweep Enabling Program}\\ 
Oliveira, C.; \textbf{Bostroem, K. A.}; Osterman, S., FENA3 (12796), 2013, Tech. rep. 
\color{blue}\href{https://ui.adsabs.harvard.edu/#abs/2013cos..rept....1O/abstract}{(ADS link)}\color{black}\\ %ISR
\\
\textsc{Summary of the COS Cycle 17 Calibration Program}\\ 
Osten, R. A.; Wolfe, M.; Ake, T.; Aloisi, A.; \textbf{Bostroem, K. A.}; Dixon, W. V. D.; Ghavamian, P.; Goudfrooij, P.; Ely, J.; Massa, D.; Niemi, S.; Oliveira, C.; Osterman, S.; Pascucci, I.; Penton, S.; Proffitt, C.; Sahnow, D.; Wheeler, T.; York, B.; Zheng, W., 2012, Tech. rep. 
\color{blue}\href{https://ui.adsabs.harvard.edu/#abs/2012cos..rept....2O/abstract}{(ADS link)}\color{black}\\ %ISR
\\
\textsc{Post-SM4 Sensitivity Calibration of the STIS Echelle Modes}\\ 
{\bf Bostroem, K. A.}; Aloisi, A.; Bohlin, R.; Hodge, P.; Proffitt, C., 2012, Tech. rep 
\color{blue}\href{https://ui.adsabs.harvard.edu/#abs/2012stis.rept....1B/abstract}{(ADS link)}\color{black}\\%Post-SM4 Sensitivity Calibration of the STIS Echelle Modes
\\
\textsc{The COS FUV channel: on-orbit performance trends and early characterization of a new detector lifetime position}\\ 
Sahnow, D. J.; Aloisi, A.; \textbf{Bostroem, K. A.}; Debes, J.; Duval, J.; Ely, J.; Hodge, P. E.; Kriss, G.; Lindsay, K.; Massa, D.; Oliveira, C.; Osten, R.; Osterman, S. N.; Penton, S. V.; Proffitt, C.; Sonnentrucker, P.; York, B., 2012, Proc. of the SPIE, 8443 
\color{blue}\href{https://ui.adsabs.harvard.edu/#abs/2012SPIE.8443E..4CS/abstract}{(ADS link)}\color{black}\\
\\
\textsc{Summary of the STIS Cycle 17 Calibration Program}\\ 
Wolfe, M. A.; Osten, R. A.; Hernandez, S.; Aloisi, A.; Bohlin, R.; \textbf{Bostroem, K. A.}; Diaz, R.; Dixon, V.; Ely, J.; Hodge, P.; Lennon, D.; Mason, E.; Niemi, S.; Pascuucci, I.; Proffitt, C.; Wheeler, T.; Zheng, W., 2012, Tech. rep. 
\color{blue}\href{https://ui.adsabs.harvard.edu/#abs/2012stis.rept....3W/abstract}{(ADS link)}\color{black}\\%ISR
\\
\textsc{Gain sag in the FUV detector of the Cosmic Origins Spectrograph}\\ 
Sahnow, D. J.; Oliveira, C.; Aloisi, A.; Hodge, P. E.; Massa, D.; Osten, R.; Proffitt, C.; \textbf{Bostroem, K. A.}; McPhate, J. B.; B\'{e}land, S.; Osterman, S. N.; Penton, S. V., 2011,  Proc. of the SPIE, 8145 
\color{blue}\href{https://ui.adsabs.harvard.edu/#abs/2011SPIE.8145E..0QS/abstract}{(ADS link)}\color{black}\\ %ISR
\\
\textsc{Updated Results from the COS Spectroscopic Sensitivity Monitoring Program}\\ 
Osten, R. A.; Massa, D.; \textbf{Bostroem, K. A.}; Aloisi, A.; Proffitt, C., 2011, Tech. rep. 
\color{blue}\href{https://ui.adsabs.harvard.edu/#abs/2011cos..rept....2O/abstract}{(ADS link)}\color{black}\\ %ISR
\\
\textsc{STIS Data Handbook v. 6.0}\\ 
{\bf Bostroem, K. A.} \& Proffitt, C. 2011
\color{blue}\href{https://ui.adsabs.harvard.edu/#abs/2011stis.book.....B/abstract}{(ADS link)}\color{black}\\
\\
\textsc{Post - SM4 Flux Calibration of the STIS Echelle Modes}\\ 
\textbf{Bostroem, K. A.}; Aloisi, A.; Bohlin, R. C.; Proffitt, C. R.; Osten, R. A.; Lennon, D., 2010, HST Calibration Workshop Proceedings
\color{blue}\href{https://ui.adsabs.harvard.edu/#abs/2010hstc.workE..51B/abstract}{(ADS link)}\color{black}\\%Post - SM4 Flux Calibration of the STIS Echelle Modes
\\
\textsc{Trend of Dark Rates of the COS and STIS NUV MAMA Detectors}\\ 
Zheng, W.; Proffitt, C. R.; Sahnow, D.; Ake, T. B.; Keyes, C.; Goudfrooij, P.; Hodge, P.; Oliveira, C.; \textbf{Bostroem, K. A.}; Long, C.; Aloisi, A., 2010, HST Calibration Workshop Proceedings 
\color{blue}\href{https://ui.adsabs.harvard.edu/#abs/2010hstc.workE..47Z/abstract}{(ADS link)}\color{black}\\
\\
\textsc{Performance of the Space Telescope Imaging Spectrograph after SM4}\\
Proffitt, C. R.; Aloisi, A.; Bohlin, C.; \textbf{Bostroem, K. A.}; Cox, C. R.; Diaz, R. I.; Dixon, W. V.; Goudfrooij, P.; Hodge, P.; Kaiser, M. E. Lallo, M. D.; Lennon, D.; Niemi, S.; Osten, R. A.; Pascucci, I.; Smith, E.; Wolfe, M. A.; York, B.; Zheng, W.; Gull, T. R.; Lindler, D. J. Woodgate, B. E., 2010, HST Calibration Workshop Proceedings
\color{blue}\href{https://ui.adsabs.harvard.edu/#abs/2010hstc.workE...6P/abstract}{(ADS link)}\color{black}\\
%\\
\clearpage
\textsc{The On-Orbit Performance of the Cosmic Origins Spectrograph}\\
Aloisi, A.; Ake, T.; \textbf{Bostroem, K. A.}; Bohlin, R.; Cox, C.; Diaz, R.; Dixon, V.; Ghavamian, P.; Goudfrooij, P.; Hartig, G.; Hodge, P.; Keyes, C.; Kriss, G.; Lallo, M.; Lennon, D.; Massa, D.; Niemi, S.; Oliveira, C.; Osten, R.; Proffitt, C. R.; Sahnow, D.; Smith, E.; Wheeler, T.; Wolfe, M.; York, B.; Zheng, W.; Green, J.; Froning, C.; Beland, S.; Burgh, E.; France, K.; Osterman, S.; Penton, S.; McPhate, J.; Delker, T., 2010, HST Calibration Workshop Proceedings 
\color{blue}\href{https://ui.adsabs.harvard.edu/#abs/2010hstc.workE...3A/abstract}{(ADS link)}\color{black}\\
%16 technical reports with {\bf Bostroem} as co-author; see \color{blue}\url{http://bit.ly/2iUeoax }\color{black}\hspace*{0pt} for details
%-------------------------------------------------------------
%-----------------------POSTERS-----------------------
%-------------------------------------------------------------
\clearpage
\vspace{-0.1in}   
\sectiontitle{Conference \\Posters}
\textsc{Do Type IIP/IIL Supernovae Experience Interaction?}\\ 
{\bf Bostroem, K. A.}; Valenti, S.; Horesh, A.; Morozova, V.; Kuin, P.; Wyatt, S.; Jerkstrand, A.; Sand, D., American Astronomical Society, AAS Meeting \#233, 2018\\
\\
\textsc{Spectral Types and Wind Velocities for Massive Stars in R136}\\
\textbf{Bostroem, K. A.}; Ma�z Apell�niz, J.; Caballero-Nieves, S. M.; Walborn, N. R.; Crowther, P. A., American Astronomical Society, AAS Meeting \#223, 2014\\
\\
\textsc{New HST/STIS Spectroscopy of Massive Members of R136 in 30 Doradus}\\ 
{\bf Bostroem, K. A.}; Walborn, N.; Crowther, P.; Caballero-Nieves, S.; Lennon, D.; Ma\'{i}z Apell\'{a}niz, J., et al., Massive Stars for $\rm{\alpha}$ to $\rm{\Omega}$, 2013\\
\\
\textsc{Long-Slit Spectroscopy for 30 Doradus}\\ 
{\bf Bostroem, K. A.}; Walborn, N.; Crowther, P.; ; Lennon, D.; Ma\'{i}z Apell\'{a}niz, J.,  American Astronomical Society, AAS Meeting \#221, 2013\\
\\
\textsc{An Update on the Performance of the Space Telescope Imaging Spectrograph}\\
\textbf{Bostroem, K. Azalee}; Aloisi, A.; Bohlin, R. C.; Cox, C.; Diaz, R.; Dixon, W.; Duval, J.; Ely, J.; Mason, E.; Osten, R.; Proffitt, C.; Sonnentrucker, P.; Wolfe, M. A.; York, B.; Zheng, W., American Astronomical Society, AAS Meeting \#219, 2012\\

%
%-------------------------------------------------------------
%----------SERVICE & OUTREACH-----------------
%-------------------------------------------------------------
\sectiontitle{Service \\\& Outreach}
\textsc{Code Review Leader}, Davis, CA\hfill 2015-Present\\
Organize, and lead weekly meetings attended by graduate students, post-docs, research staff, and faculty to improve and share coding knowledge and best practices.\\
\\
\textsc{Diversity and Inclusion in Physics Group Member and Co-Leader}, Davis, CA\hfill 2014-Present\\
Bring discussions, programs, and activities to the department to improve the departmental culture surrounding diversity, equity, and inclusion.\\
\\
\textsc{Python in Astronomy SOC member} \hfill 2017-2018\\
Organized and coordinated the 2018 Python in Astronomy conference.\\
\\
\textsc{Python Lesson Maintainer}, Software Carpentry Lessons\hfill 2014-2016\\
Provide feedback on and incorporate suggested improvements to the Python lesson using git and GitHub.\\
\\
\textsc{Scipy Conference Proceedings Editor}\hfill 2015\\
Edited conference proceedings via GitHub for conference on scientific computing with Python.\\
\\
\textsc{HST Spectroscopic Legacy Working Group}, STScI\hfill2013-2014\\
Worked to define tools to improve spectroscopic archival products and interface for the HST spectrographs.\\
\\
\textsc{STScI Summer Student Selection Committee}, Baltimore, MD\hfill2012\\
Evaluated applications for the Space Astronomy Summer Program at STScI.\\
\\
\textsc{HST Calibration Workshop Organizing Committee}\hfill 2012\\
Organized and Coordinated the HST Calibration Workshop.\\
\\
\textsc{Project Astro Astronomer}, San Diego, CA\hfill 2007-2008\\
Brought hands on astronomy projects to a third grade class.\\
%-------------------------------------------------------------
%-----------TEACHING EXPERIENCE--------------
%-------------------------------------------------------------
\vspace{-0.1in}   
\sectiontitle{Teaching\\Experience}
\textsc{Lead Instructor, Software Carpentry Workshops}\hfill 2012-present\\
Lead two day workshops at institutions and conferences internationally to enable scientists to work more efficiently and reproducibly by improving their coding skills. Workshops include Python, Git, Unix, test driven development, object oriented and functional programming. 
%\begin{itemize}
%\item []American Astronomical Society 233rd Meeting, 2019, Seattle, WA 
%\item []American Astronomical Society 231st Meeting, 2018, National Harbor,  MD 
%\item []American Astronomical Society 229th Meeting, 2017, Grapevine, TX 
%\item []American Astronomical Society 227th Meeting, 2016, Kissammee, FL 
%\item []Women in Science and Engineering, 2015, Davis, CA 
%\item []American Astronomical Society 225th Meeting, 2015, Seattle, WA 
%\item []Stanford University, 2014, Palo Alto, CA 
%\item []Women in Science and Engineering, 2014, Davis, CA 
%\item []European Space Astronomy Centre, 2014, Cebreros, Avila, Spain 
%\item []Space Telescope Science Institute, 2013, Baltimore, MD
%\item []University of California, Davis, 2013, Davis, CA
%\item []Lawrence Berkeley National Laboratory, 2013, Berkeley, CA
%\item []University of Maryland, 2013, Baltimore, MD
%\item []University of Edinburgh, 2012, Edinburgh, Scotland, UK
%\item []George Mason University, 2012, Fairfax, VA
%\end{itemize}
\\
\textsc{Teaching Assistant, University of California} (PHYS 7A, 7B) \hfill 2014-2018\\
Guided students through lab work to understand thermodynamics, mechanics, electrical circuits, and fluid dynamics with lecture, small group work, and whole class discussions.\\
%TODO: Add course evaluations
\\
\textsc{Lead Teaching Assistant, San Diego State University}\hfill Fall 2008\\
Supervised astronomy lab instructors and prepared them to teach each week. \\
\\
\textsc{Teaching Assistant, San Diego State University}\hfill 2006-2009\\
Enabled a better understanding of general astronomy through hand on applications of topics covered in lecture. Prepared labs, developed lesson plans and curriculum.\\
\\
\textsc{Mathematics Teacher, Ross Valley Summer School}\hfill Summer 2006\\
Designed the curriculum and lesson plans for and taught first through sixth grade mathematics.\\
\\
\textsc{Mathematics Student Teacher, Poughkeepsie High School}\hfill Fall 2006\\
Created lesson plans and taught a full course load (five periods) of high school mathematics.\\
\\
\textsc{Science Teacher and Curriculum Designer, Crossroads Summer Camp}\hfill Summer 2003, 2004 \\
Created a curriculum and lesson plans for and taught sixth, seventh, and eighth grade science courses.\\
%
%-------------------------------------------------------------
%-----------------REFERENCES-----------------------
%-------------------------------------------------------------
%\sectiontitle{References}
%Stefano Valenti \\
%Alessandra Aloisi \\
%Ori Fox \\
\end{llist}
\end{document}






























